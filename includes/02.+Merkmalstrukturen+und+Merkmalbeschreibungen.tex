
\begin{frame}
  {Ziele}
  \onslide<+->
  \onslide<+->
  Worum geht es heute?\\
  \Zeile
  \begin{itemize}[<+->]
    \item Systematisierung von Merkmalen in Grammatiken
    \item Strukturierte\slash hierarchische Mermale
    \item Unifikationvon Merkmalstrukturen
    \item Merkmalstrukturen vs.\ Merkmalbeschreibungen
  \end{itemize}
  \Zeile
  \centering
  \onslide<+->
  \grau{\citet[Kapitel~2]{MuellerLehrbuch}}
\end{frame}


\begin{frame}
  {Vorteil von Merkmalstrukturen}
  \onslide<+->
  \onslide<+->
  Problem mit einfachen \alert{Phrasenstrukturgrammatiken}\\
  \Halbzeile
  \begin{itemize}[<+->]
    \item \alert{Symbolinflation} | Selbst für einfachte Valenz-\slash Kongruenzphänomene
    \item Viele Regeln und viele Kategorien
  \end{itemize}
  \onslide<+->
  \Zeile
  \alert{Merkmalstrukturen} wie in HPSG\\
  \Halbzeile
  \begin{itemize}[<+->]
    \item \alert{Komplexe Symbole}, dadurch weniger Symbole
    \item Extrem einfache \alert{Regeln} (Kombinatorik)
  \end{itemize}
\end{frame}

\begin{frame}
  {Merkmalstrukturen und Merkmalbeschreibungen}
  \onslide<+->
  \onslide<+->
  \alert{Merkmalstrukturen} modellieren linguistische Objekte.\\
  \Halbzeile
  \begin{itemize}[<+->]
    \item Merkmal-Wert-Struktur
    \item Attribut-Wert-Struktur
    \item \emph{Feature structure}
  \end{itemize}
  \onslide<+->
  \Zeile
  Wir nutzen \alert{Merkmalsbeschreibungen}, um über Merkmalstrukturen zu sprechen.\\
  \Halbzeile
  \begin{itemize}[<+->]
    \item \emph{Attribute-value matrix}
    \item \emph{Feature matrix}
  \end{itemize}
  \onslide<+->
  \Zeile
  \centering 
  \grau{\footnotesize\citet{Shieber86a}, \citet{ps}, \citet{Johnson88},\citet{Carpenter92a}, \citet{King94a}, \citet{Richter2004a-u,Richter2021a}}
\end{frame}

\begin{frame}
  {AVM-Format}
  \onslide<+->
  \onslide<+->
  Einfache Merkmalbeschreibung\\
  \onslide<+->
  \Viertelzeile
  \alert{\begin{avm}
    \[ attribut & wert \]
  \end{avm}}\\
  \onslide<+->
  \Halbzeile
  Mehrere Attribut-Wert-Paare in einer Struktur\\
  \onslide<+->
  \Viertelzeile
  \alert{\begin{avm}
    \[ attribut1 & wert1 \\
      attribut2 & wert2 \\
      \ldots & \ldots
    \]
    \end{avm}}\\
    \onslide<+->
    \Halbzeile
    AVMs können wieder Werte von Attributen sein!\\
    \alert{\begin{avm}
      \[ attribut1a & wert1a \\
         attribut1b & 
                     \[ attribut2a & wert2a \\
                        attribut2b & wert2b \]
      \]
    \end{avm}
    }
\end{frame}

\begin{frame}
  {Wörter in Merkmalen beschreiben | Phone und Graphen}
  \onslide<+->
  \onslide<+->
  \textsc{phone} oder \textsc{graphen} | Aussprache bzw. Schreibung\\
  \onslide<+->
  \Halbzeile
  \alert{\begin{avm}
    \[ graphen & \textit{Tisch} \]
  \end{avm}}\\
  \onslide<+->
  \Zeile
  Aber reicht diese Datenstruktur?\\
  \Viertelzeile
  \begin{itemize}[<+->]
    \item \textit{Tisch} | Wieder nur ein \alert{Symbol}
    \item Phonetik\slash Phonologie | \alert{Ketten} Phonen\slash Phonemen\\
      \grau{Bei \citet{Schaefer2018a} und anderen: Segmente}
    \item Phonologische Grammatik | Zugriff auf einzelne Segmente\\
      \grau{Auslautverhärtung | Zugriff auf letztes Segment einer Silbe}
  \end{itemize}
\end{frame}

\begin{frame}
  {Listen}
  \onslide<+->
  \onslide<+->
  Lösung für \textsc{graph(en)} oder \textsc{phon(e)} | \alert{Geordnete Listen}\\
  \onslide<+->
  \Halbzeile
  \alert{\begin{avm}
    \[ graph & \<\it T,i,s,c,h\> \]
  \end{avm}}\\
  \onslide<+->
  \Zeile
  Auf einer Liste stehen eigentlich auch Merkmalbeschreibungen.\\
  \onslide<+->
  \Viertelzeile
  \scalebox{0.8}{\begin{avm}
    \[ phon &  \<\[ art & plosiv \\ ort & alveolar \],
    \[ art & vokal \\ position & vorn \\ höhe & hoch \\ rundung & nein \],
    \[ art & frikativ \\ ort & alveolar \]
  \> \]
  \end{avm}}\\
  \onslide<+->
  \Zeile
  Falsche Kurzschreibweise in typischer HPSG\\
  \onslide<+->
  \Viertelzeile
  \orongsch{\begin{avm}
    \[ phon & \textit{Tisch} \]
  \end{avm}}
\end{frame}
