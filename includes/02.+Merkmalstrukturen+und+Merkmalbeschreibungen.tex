
\begin{frame}
  {Ziele}
  \onslide<+->
  \onslide<+->
  Worum geht es heute?\\
  \Zeile
  \begin{itemize}[<+->]
    \item Repräsentation von Merkmalen und ihren Werten in Grammatiken
    \item Strukturierte\slash hierarchische Merkmalstrukturen
    \item Unifikation von Merkmalstrukturen
    \item Merkmalstrukturen vs.\ Merkmalbeschreibungen
  \end{itemize}
  \Zeile
  \centering
  \onslide<+->
  \grau{\citet[Kapitel~2]{MuellerLehrbuch}}
\end{frame}


\begin{frame}
  {Vorteil von Merkmalstrukturen}
  \onslide<+->
  \onslide<+->
  Problem mit einfachen \alert{Phrasenstrukturgrammatiken}\\
  \Halbzeile
  \begin{itemize}[<+->]
    \item \alert{Symbolinflation} | Selbst für einfachete Valenz-\slash Kongruenzphänomene
    \item Viele Regeln und viele Kategorien
  \end{itemize}
  \onslide<+->
  \Zeile
  \alert{Merkmalstrukturen} wie in HPSG\\
  \Halbzeile
  \begin{itemize}[<+->]
    \item \alert{Komplexe Symbole}, dadurch weniger Symbole
    \item Extrem einfache \alert{Regeln} (Kombinatorik)
  \end{itemize}
\end{frame}

\begin{frame}
  {Merkmalstrukturen und Merkmalbeschreibungen}
  \onslide<+->
  \onslide<+->
  \alert{Merkmalstrukturen} modellieren linguistische Objekte.\\
  \Halbzeile
  \begin{itemize}[<+->]
    \item Merkmal-Wert-Struktur
    \item Attribut-Wert-Struktur
    \item \emph{Feature structure}
  \end{itemize}
  \onslide<+->
  \Zeile
  Wir nutzen \alert{Merkmalsbeschreibungen}, um über Merkmalstrukturen zu sprechen.\\
  \Halbzeile
  \begin{itemize}[<+->]
    \item \emph{Attribute-value matrix}
    \item \emph{Feature matrix}
  \end{itemize}
  \onslide<+->
  \Zeile
  \centering 
  \grau{\footnotesize\citet{Shieber86a}, \citet{ps}, \citet{Johnson88},\citet{Carpenter92a}, \citet{King94a}, \citet{Richter2004a-u,Richter2021a}}
\end{frame}

\begin{frame}
  {AVM-Format}
  \onslide<+->
  \onslide<+->
  Einfache Merkmalbeschreibung\\
  \onslide<+->
  \Viertelzeile
  \alert{\begin{avm}
    \[ attribut & wert \]
  \end{avm}}\\
  \onslide<+->
  \Zeile
  Mehrere Attribut-Wert-Paare in einer Struktur\\
  \onslide<+->
  \Viertelzeile
  \alert{\begin{avm}
    \[ attribut1 & wert1 \\
      attribut2 & wert2 \\
      \ldots & \ldots
    \]
    \end{avm}}\\
    \onslide<+->
    \Zeile
    Komplexe Merkmale können Werte von Attributen sein!\\
    \alert{\begin{avm}
      \[ attribut1a & wert1a \\
         attribut1b & 
                     \[ attribut2a & wert2a \\
                        attribut2b & wert2b \]
      \]
    \end{avm}
    }
\end{frame}

\begin{frame}
  {Wörter in Merkmalen beschreiben | Phone und Graphen}
  \onslide<+->
  \onslide<+->
  \textsc{phone} oder \textsc{graphen} | Aussprache bzw. Schreibung\\
  \onslide<+->
  \Halbzeile
  \alert{\begin{avm}
    \[ graphen & \textit{Tisch} \]
  \end{avm}}\\
  \onslide<+->
  \Zeile
  Aber reicht diese Datenstruktur?\\
  \Viertelzeile
  \begin{itemize}[<+->]
    \item \textit{Tisch} | Sieht aus wie ein \alert{Symbol} ohne Struktur
    \item Phonetik\slash Phonologie | \alert{Ketten} Phonen\slash Phonemen\\
      \grau{Bei \citet{Schaefer2018a} und anderen: Segmente}
    \item Phonologische Grammatik | Zugriff auf einzelne Segmente\\
      \grau{Auslautverhärtung | Zugriff auf letztes Segment einer Silbe}
  \end{itemize}
\end{frame}

\begin{frame}
  {Listen}
  \onslide<+->
  \onslide<+->
  Lösung für \textsc{graph(en)} oder \textsc{phon(e)} | \alert{Geordnete Listen}\\
  \onslide<+->
  \Halbzeile
  \alert{\begin{avm}
    \[ graph & \<\it T,i,s,c,h\> \]
  \end{avm}}\\
  \onslide<+->
  \Zeile
  Auf einer Liste stehen eigentlich auch Merkmalbeschreibungen.\\
  \onslide<+->
  \Viertelzeile
  \scalebox{0.8}{\begin{avm}
    \[ phon &  \<\[ art & plosiv \\ ort & alveolar \],
    \[ art & vokal \\ position & vorn \\ höhe & hoch \\ rundung & nein \],
    \[ art & frikativ \\ ort & alveolar \]
  \> \]
  \end{avm}}\\
  \onslide<+->
  \Zeile
  Strenggenommen falsche Kurzschreibweisen in typischer HPSG\\
  \onslide<+->
  \Viertelzeile
  \orongsch{\begin{avm}
    \[ phon & \textit{Tisch} \]
  \end{avm}}\\
  \Viertelzeile
  \onslide<+->
  \orongsch{\begin{avm}
    \[ phon & \<\rm\it Tisch\> \]
  \end{avm}}
\end{frame}

\begin{frame}
  {Morpholosyntaktische Merkmale}
  \onslide<+->
  \onslide<+->
  Lösung für Probleme mit Genus usw.\ in PSGs von letzter Woche\\
  \onslide<+->
  \Viertelzeile
  \scalebox{0.8}{\begin{avm}
    \[ graphen & \textit{Tisch} \\
      \alert{genus} & \alert{maskulin} \\
      \alert{numerus} & \alert{singular} \\
      \alert{kasus} & \alert{nominativ}
    \]
  \end{avm}}\\
  \onslide<+->
  \Halbzeile
  Andere Merkmalausstattungen = andere sprachliche Zeichen\\
  \onslide<+->
  \Viertelzeile
  \scalebox{0.6}{\begin{avm}
    \[ graphen & \textit{Tisch} \\
      wortart & nomen \\
      genus & maskulin \\
      numerus & singular \\
      \orongsch{kasus} & \orongsch{akkusativ}
    \]
  \end{avm}}\onslide<+->
  \scalebox{0.6}{\begin{avm}
    \[ graphen & \textit{Tisch} \\
      wortart & nomen \\
      genus & maskulin \\
      numerus & singular \\
      \orongsch{kasus} & \orongsch{dativ}
    \]
  \end{avm}}\onslide<+->
  \scalebox{0.6}{\begin{avm}
    \[ graphen & \textit{Tisch} \\
      wortart & nomen \\
      genus & maskulin \\
      numerus & singular \\
      \orongsch{kasus} & \orongsch{genitiv}
    \]
  \end{avm}}\\
  \Halbzeile
  \onslide<+->
  Abgekürzte Schreibweise mit \gruen{\textit{oder} bzw. $\vee$}\\
  \onslide<+->
  \Viertelzeile
  \scalebox{0.8}{\begin{avm}
    \[ graphen & \textit{Tisch} \\
      wortart & nomen \\
      genus & maskulin \\
      numerus & singular \\
      \gruen{kasus} & \gruen{nominativ $\vee$ akkusativ $\vee$ dativ $\vee$ genitiv}
    \]
  \end{avm}}
\end{frame}

\begin{frame}
  {Dasselbe für eine Verbform}
  \onslide<+->
  \onslide<+->
  Verben | Teilweise dieselben, teilweise andere Merkmale verglichen mit Nomina\\
  \onslide<+->
  \Viertelzeile
  \begin{avm}
    \[ graphen & \textit{sieht} \\
      wortart & verb \\
      numerus & singular \\
      person & 3 
    \]
  \end{avm}\\
  \onslide<+->
  \Zeile
  Syntaktisch relevant auch \alert{Finitheit} bzw.\ \gruen{Status} \\
  \Viertelzeile
  \onslide<+->
  \begin{avm}
    \[ graphen & \textit{sieht} \\
      wortart & verb \\
      numerus & singular \\
      person & 3 \\
      \alert{finit} & \alert{ja} \\
    \]
  \end{avm}
  \onslide<+->
  \begin{avm}
    \[ graphen & \textit{gesehen} \\
      wortart & verb \\
      \alert{finit} & \alert{nein} \\
      \gruen{status} & \gruen{3} \\
    \]
  \end{avm}
\end{frame}

\begin{frame}
  {Getypte Strukturen}
  \onslide<+->
  \onslide<+->
  Nicht alle Wörter haben alle Merkmale. | \alert{Typen} und \alert{Beschränkungen} über Typen
  \onslide<+->
  \Zeile
  \begin{avm}
    \[ \asort{\alert{nomen}}
    graphen & \textit{Tischs} \\
    genus & maskulin \\
    numerus & singular \\
    kasus & genitiv \\
  \]
  \end{avm}\\
  \onslide<+->
  \begin{avm}
    \[ \asort{\alert{finites-verb}}
    graphen & \textit{sieht} \\
    numerus & singular \\
    person & 3 \\
    tempus & präsens \\
    modus & indikativ
  \]
  \end{avm}
  \onslide<+->
  \begin{avm}
    \[ \asort{\alert{infinites-verb}}
    graphen & \textit{gesehen} \\
    status & 1 
  \]
  \end{avm}
\end{frame}

\begin{frame}
  {Typenhierarchien}
  \onslide<+->
  \onslide<+->
  Typen sind sehr wichtig in HPSG und bilden \alert{Hierachien}. Denkbares Beispiel:\\
  \onslide<+->
  \Zeile
  \centering 
  \scalebox{0.7}{\begin{forest}
    [ wort
      [nomen
        [eigenname]
        [appellativum
          [zählsubstantiv]
          [stoffsubstantiv]
        ]
      ]
      [verb
        [finites-verb]
        [infinites-verb]
      ]
    ]
  \end{forest}}

  \Zeile
  \raggedright
  \begin{itemize}[<+->]
    \item Typen sind die eigentlichen \alert{Wortarten} in HPSG.
    \item \alert{Monotonizität} | \alert{Untertypen} erbt alle Merkmale\slash Beschränkungen ihrer \alert{Obertypen}.
    \item \alert{Mehrfachvererbung} | Ein Typ kann \alert{mehrere Obertypen} haben.
  \end{itemize}
\end{frame}

\begin{frame}
  {Valenz}
  \onslide<+->
  \onslide<+->
  Letzte Woche in PSGs | Valenz doppelt in \alert{Kategorien} und \orongsch{Regeln} kodiert\\
  \Halbzeile
  \begin{itemize}[<+->]
    \item[ ] \orongsch{Regel} für Satz mit intransitivem Verb\\
      \footnotesize S \goesto NP(Per, Num, nom) \alert{V\_itr}(Per, Num)
    \item[ ] \orongsch{Regel} für Satz mit transivitem Verb\\
      \footnotesize S \goesto NP(Per1, Num1, nom) NP(Per2, Num2, akk) \alert{V\_tr}(Per1, Num1)
    \item[ ] \orongsch{Regel} für Satz mit ditransitivem Verb\\
      \footnotesize S \goesto NP(Per1, Num1, nom) NP(Per2, Num2, dat) NP(Per3, Num3, akk) \alert{V\_dtr}(Per1, Num1)
  \end{itemize}
  \onslide<+->
  \Zeile
  \alert{Typische Definition von Valenz allerdings}\\
  \onslide<+->
  Die \alert{Liste} der Ergänzungen eines Worts.\\
\end{frame}

\begin{frame}
  {Valenz als Liste}
  \onslide<+->
  \onslide<+->
  Valenz | \alert{Liste von Merkmalsbeschreibungen}\\
  \onslide<+->
  \Halbzeile
  \begin{avm}
    \[ \asort{finites-verb}
      graphen & \textit{sieht} \\
      numerus & singular \\
      person & 3 \\
      tempus & präsens \\
      modus & indikativ \\
      \alert{valenz} & \alert{\< \[ \asort{nomen} \], \[ \asort{nomen} \]\>}
    \]
  \end{avm}
\end{frame}

\begin{frame}
  {Hinreichende Beschreibung}
  \onslide<+->
  \onslide<+->
  Valenzliste | Hinreichend eingrenzende Beschreibung der Ergänzungen des Verbs\\
  \onslide<+->
  \Halbzeile
  \begin{avm}
    \[ \asort{finites-verb}
      graphen & \textit{sieht} \\
      numerus & singular \\
      person & 3 \\
      tempus & präsens \\
      modus & indikativ \\
      \alert{valenz} & \alert{\< \[ \asort{nomen}
                                    numerus & singular \\
                                    person & 3 \\
                                    kasus & nom \], \[ \asort{nomen}
                                                        kasus & akkusativ \]\>}
    \]
  \end{avm}
\end{frame}


\begin{frame}
  {Subjekt-Verb-Kongruenz und Strukturteilung}
  \onslide<+->
  \onslide<+->
  \alert{Übereinstimmung von Merkmalen} | Hart verdrahtet mit \gruen{Strukturteilung}\\
  \onslide<+->
  \Halbzeile
  \scalebox{0.8}{\begin{avm}
    \[ \asort{finites-verb}
      graphen & \textit{sieht} \\
      numerus & \alt<1-3>{\alert{singular}}{\gruen{\@1 singular}} \\
      person & \alt<1-3>{\alert{3}}{\gruen{\@2 3}} \\
      tempus & präsens \\
      modus & indikativ \\
      valenz & \< \[ \asort{nomen}
        numerus & \alt<1-3>{\alert{singular}}{\gruen{\@1}} \\
        person & \alt<1-3>{\alert{3}}{\gruen{\@2}} \\
                     kasus & nom \], \[ \asort{nomen}
                                         kasus & akkusativ \]\>
    \]
  \end{avm}}\\
  \onslide<+->
  \Zeile 
  \gruen{Strukturteilung bedeutet Token-Identität von Werten,} \rot{nicht Kopie}!
\end{frame}

\begin{frame}
  {Beispiel für Valenz einer Präposition}
  \onslide<+->
  \onslide<+->
  Valenz von Präpositionen | NP in einem bestimmten Kasus\\
  \onslide<+->
  \Halbzeile
  \begin{avm}
    \[ \asort{präposition} 
    graphen & \textit{wegen} \\
    valenz & \< \[ \asort{nomen} 
                kasus & genitiv \] \>
  \]
  \end{avm}\\
  \Zeile
  \begin{itemize}[<+->]
    \item Was ist mit \alert{argumentmarkierenden Präpositionen}\slash Präpositionalobjekten?\\
      \grau{\textit{leiden unter}, \textit{abhängen von}, \textit{glauben an} usw.}
    \item Was ist mit \alert{Wechselpräpositionen} mit Akkusativ oder Dativ?\\
      \grau{\textit{unter}, \textit{neben}, \textit{über} usw.}
  \end{itemize}
\end{frame}

\begin{frame}
  {Beispieleintrag für einen Determinierer}
  \onslide<+->
  \onslide<+->
  Kongruenzmerkmale in der NP auch beim Determinierer\\
  \onslide<+->
  \Zeile 
  \begin{avm}
    \[ \asort{determinierer} 
    graphen & \textit{des} \\
    genus & maskulin \\
    numerus & singular \\
    kasus & genitiv \\
  \]
  \end{avm}
\end{frame}

\begin{frame}
  {Determinierer in der NP}
  \onslide<+->
  \onslide<+->
  DP oder NP? | \alert{Für Deutsch ist eine NP-Analyse näherliegend.}\\
  \onslide<+->
  \Zeile
  \begin{avm}
    \[ \asort{nomen}
    graphen & \textit{Tischs} \\
    genus & \alt<1-5>{maskulin}{\gruen{\@1 maskulin}} \\
    numerus & \alt<1-5>{singular}{\gruen{\@2 singular}} \\
    kasus & \alt<1-5>{genitiv}{\gruen{\@3 genitiv}} \\
    \alt<5->{\gruen{valenz} & \gruen{\<\[ \asort{determinierer} 
                             genus & \ \@1 \\
                             numerus & \@2 \\
                             kasus & \ \@3
                           \]\>}}{}
  \]
  \end{avm}\\
  \onslide<+->
  \Zeile
  Wie kann man \alert{Notwendigkeit von} und \alert{Kongruenz mit Determinierern} kodieren?
  \onslide<+->
  \onslide<+->
\end{frame}
