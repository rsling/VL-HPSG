
\section{Einleitung}

\begin{frame}
  {Ziele}
  \onslide<+->
  \onslide<+->
  Worum geht es heute?\\
  \Zeile
  \begin{itemize}[<+->]
    \item Repräsentation von Merkmalen und ihren Werten in Grammatiken
    \item Strukturierte\slash hierarchische Merkmalstrukturen
    \item Unifikation von Merkmalstrukturen
    \item Merkmalstrukturen vs.\ Merkmalbeschreibungen
  \end{itemize}
  \Zeile
  \centering
  \onslide<+->
  \grau{\citet[Kapitel~2]{MuellerLehrbuch}}
\end{frame}

\begin{frame}
  {Warnung}
  \onslide<+->
  \onslide<+->
  \centering 
  \LARGE
  \rot{Merken Sie sich die Strukturen von heute\\nicht als "`korrekte Modellierung"'\\
  des Deutschen in HPSG!}\\
  \Zeile
  \onslide<+->
  \normalsize
  Wir nehmen heute einige Vereinfachungen und Didaktisierungen vor,\\
  denn es geht darum, grundlegende Repräsentationen\slash Prinzipien einzuführen.\\
  \Zeile
  \onslide<+->
  Völlig abwegig sind die Strukturen dieser Lektion aber auch nicht.\\
  \Zeile
  \onslide<+->
  Generell haben Sie mehr davon, wenn Sie in jeder Woche zu verstehen versuchen,\\
  warum sich bestimmte Repräsentationen wieder ändern, als wenn Sie von Anfang an\\
  nur wissen wollen, wie das Endergebnis in den Prüfungen aussehen wird.
\end{frame}

\section{Merkmalstrukturen}

\begin{frame}
  {Vorteil von Merkmalstrukturen}
  \onslide<+->
  \onslide<+->
  Problem mit einfachen \alert{Phrasenstrukturgrammatiken}\\
  \Halbzeile
  \begin{itemize}[<+->]
    \item \alert{Symbolinflation} | Selbst für einfachete Valenz-\slash Kongruenzphänomene
    \item Viele Regeln und viele Kategorien
  \end{itemize}
  \onslide<+->
  \Zeile
  \alert{Merkmalstrukturen} wie in HPSG\\
  \Halbzeile
  \begin{itemize}[<+->]
    \item \alert{Komplexe Symbole}, dadurch weniger Symbole
    \item Extrem einfache \alert{Regeln} (Kombinatorik)
  \end{itemize}
\end{frame}

\begin{frame}
  {Merkmalstrukturen und Merkmalbeschreibungen}
  \onslide<+->
  \onslide<+->
  \alert{Merkmalstrukturen} modellieren linguistische Objekte.\\
  \Halbzeile
  \begin{itemize}[<+->]
    \item Merkmal-Wert-Struktur
    \item Attribut-Wert-Struktur
    \item \emph{Feature structure}
  \end{itemize}
  \onslide<+->
  \Zeile
  Wir nutzen \alert{Merkmalsbeschreibungen}, um über Merkmalstrukturen zu sprechen.\\
  \Halbzeile
  \begin{itemize}[<+->]
    \item \emph{Attribute-value matrix}
    \item \emph{Feature matrix}
  \end{itemize}
  \onslide<+->
  \Zeile
  \centering 
  \grau{\footnotesize\citet{Shieber86a}, \citet{ps}, \citet{Johnson88},\citet{Carpenter92a}, \citet{King94a}, \citet{Richter2004a-u,Richter2021a}}
\end{frame}

\begin{frame}
  {AVM-Format}
  \onslide<+->
  \onslide<+->
  Einfache Merkmalbeschreibung\\
  \onslide<+->
  \Viertelzeile
  \alert{\begin{avm}
    \[ attribut & wert \]
  \end{avm}}\\
  \onslide<+->
  \Zeile
  Mehrere Attribut-Wert-Paare in einer Struktur\\
  \onslide<+->
  \Viertelzeile
  \alert{\begin{avm}
    \[ attribut1 & wert1 \\
      attribut2 & wert2 \\
      \ldots & \ldots
    \]
    \end{avm}}\\
    \onslide<+->
    \Zeile
    Komplexe Merkmale können Werte von Attributen sein!\\
    \alert{\begin{avm}
      \[ attribut1a & wert1a \\
         attribut1b & 
                     \[ attribut2a & wert2a \\
                        attribut2b & wert2b \]
      \]
    \end{avm}
    }
\end{frame}

\begin{frame}
  {Wörter in Merkmalen beschreiben | Phone und Graphen}
  \onslide<+->
  \onslide<+->
  \textsc{phone} oder \textsc{graphen} | Aussprache bzw. Schreibung\\
  \onslide<+->
  \Halbzeile
  \alert{\begin{avm}
    \[ graphen & \textit{Tisch} \]
  \end{avm}}\\
  \onslide<+->
  \Zeile
  Aber reicht diese Datenstruktur?\\
  \Viertelzeile
  \begin{itemize}[<+->]
    \item \textit{Tisch} | Sieht aus wie ein \alert{Symbol} ohne Struktur
    \item Phonetik\slash Phonologie | \alert{Ketten} Phonen\slash Phonemen\\
      \grau{Bei \citet{Schaefer2018a} und anderen: Segmente}
    \item Phonologische Grammatik | Zugriff auf einzelne Segmente\\
      \grau{Auslautverhärtung | Zugriff auf letztes Segment einer Silbe}
  \end{itemize}
\end{frame}

\begin{frame}
  {Listen}
  \onslide<+->
  \onslide<+->
  Lösung für \textsc{graph(en)} oder \textsc{phon(e)} | \alert{Geordnete Listen}\\
  \onslide<+->
  \Halbzeile
  \alert{\begin{avm}
    \[ graph & \<\it T,i,s,c,h\> \]
  \end{avm}}\\
  \onslide<+->
  \Zeile
  Auf einer Liste stehen eigentlich auch Merkmalbeschreibungen.\\
  \onslide<+->
  \Viertelzeile
  \scalebox{0.8}{\begin{avm}
    \[ phon &  \<\[ art & plosiv \\ ort & alveolar \],
    \[ art & vokal \\ position & vorn \\ höhe & hoch \\ rundung & nein \],
    \[ art & frikativ \\ ort & alveolar \]
  \> \]
  \end{avm}}\\
  \onslide<+->
  \Zeile
  Strenggenommen falsche Kurzschreibweisen für \textsc{phon} in typischer HPSG\\
  \onslide<+->
  \Viertelzeile
  \orongsch{\begin{avm}
    \[ phon & \textit{Tisch} \]
  \end{avm}}\\
  \Viertelzeile
  \onslide<+->
  \orongsch{\begin{avm}
    \[ phon & \<\rm\it Tisch\> \]
  \end{avm}}
\end{frame}

\begin{frame}
  {Morpholosyntaktische Merkmale}
  \onslide<+->
  \onslide<+->
  Lösung für Probleme mit Genus usw.\ in PSGs von letzter Woche\\
  \onslide<+->
  \Viertelzeile
  \scalebox{0.8}{\begin{avm}
    \[ graphen & \textit{Tisch} \\
      \alert{genus} & \alert{maskulin} \\
      \alert{numerus} & \alert{singular} \\
      \alert{kasus} & \alert{nominativ}
    \]
  \end{avm}}\\
  \onslide<+->
  \Halbzeile
  Andere Merkmalausstattungen = andere sprachliche Zeichen\\
  \onslide<+->
  \Viertelzeile
  \scalebox{0.6}{\begin{avm}
    \[ graphen & \textit{Tisch} \\
      wortart & nomen \\
      genus & maskulin \\
      numerus & singular \\
      \orongsch{kasus} & \orongsch{akkusativ}
    \]
  \end{avm}}\onslide<+->
  \scalebox{0.6}{\begin{avm}
    \[ graphen & \textit{Tisch} \\
      wortart & nomen \\
      genus & maskulin \\
      numerus & singular \\
      \orongsch{kasus} & \orongsch{dativ}
    \]
  \end{avm}}\onslide<+->
  \scalebox{0.6}{\begin{avm}
    \[ graphen & \textit{Tisch} \\
      wortart & nomen \\
      genus & maskulin \\
      numerus & singular \\
      \orongsch{kasus} & \orongsch{genitiv}
    \]
  \end{avm}}\\
  \Halbzeile
  \onslide<+->
  Abgekürzte Schreibweise mit \gruen{\textit{oder} bzw. $\vee$}\\
  \onslide<+->
  \Viertelzeile
  \scalebox{0.8}{\begin{avm}
    \[ graphen & \textit{Tisch} \\
      wortart & nomen \\
      genus & maskulin \\
      numerus & singular \\
      \gruen{kasus} & \gruen{nominativ $\vee$ akkusativ $\vee$ dativ $\vee$ genitiv}
    \]
  \end{avm}}
\end{frame}

\begin{frame}
  {Dasselbe für eine Verbform}
  \onslide<+->
  \onslide<+->
  Verben | Teilweise dieselben, teilweise andere Merkmale verglichen mit Nomina\\
  \onslide<+->
  \Viertelzeile
  \begin{avm}
    \[ graphen & \textit{sieht} \\
      wortart & verb \\
      person & dritte \\
      numerus & singular \\
    \]
  \end{avm}\\
  \onslide<+->
  \Zeile
  Syntaktisch relevant auch \alert{Finitheit} bzw.\ \gruen{Status} \\
  \Viertelzeile
  \onslide<+->
  \begin{avm}
    \[ graphen & \textit{sieht} \\
      wortart & verb \\
      person & dritte \\
      numerus & singular \\
      \alert{finit} & \alert{ja} \\
    \]
  \end{avm}
  \onslide<+->
  \begin{avm}
    \[ graphen & \textit{gesehen} \\
      wortart & verb \\
      \alert{finit} & \alert{nein} \\
      \gruen{status} & \gruen{3} \\
    \]
  \end{avm}
\end{frame}

\section{Typen}

\begin{frame}
  {Getypte Strukturen}
  \onslide<+->
  \onslide<+->
  Nicht alle Wörter haben alle Merkmale. | \alert{Typen} und \alert{Beschränkungen} über Typen
  \onslide<+->
  \Zeile
  \begin{avm}
    \[ \asort{\alert{nomen}}
    graphen & \textit{Tischs} \\
    genus & maskulin \\
    numerus & singular \\
    kasus & genitiv \\
  \]
  \end{avm}\\
  \onslide<+->
  \begin{avm}
    \[ \asort{\alert{finites-verb}}
    graphen & \textit{sieht} \\
    person & dritte \\
    numerus & singular \\
    tempus & präsens \\
    modus & indikativ
  \]
  \end{avm}
  \onslide<+->
  \begin{avm}
    \[ \asort{\alert{infinites-verb}}
    graphen & \textit{gesehen} \\
    status & 1 
  \]
  \end{avm}
\end{frame}

\begin{frame}
  {Typenhierarchien}
  \onslide<+->
  \onslide<+->
  Typen sind sehr wichtig in HPSG und bilden \alert{Hierachien}. Denkbares Beispiel:\\
  \onslide<+->
  \Zeile
  \centering 
  \scalebox{0.7}{\begin{forest}
    [ wort
      [nomen
        [eigenname]
        [appellativum
          [zählsubstantiv]
          [stoffsubstantiv]
        ]
      ]
      [verb
        [finites-verb]
        [infinites-verb]
      ]
    ]
  \end{forest}}

  \Zeile
  \raggedright
  \begin{itemize}[<+->]
    \item Typen sind die eigentlichen \alert{Wortarten} in HPSG.
    \item \alert{Monotonizität} | \alert{Untertypen} erbt alle Merkmale\slash Beschränkungen ihrer \alert{Obertypen}.
    \item \alert{Mehrfachvererbung} | Ein Typ kann \alert{mehrere Obertypen} haben.
    \item \grau{Keine Sorge! Dazu kommen wir noch im Detail.}
  \end{itemize}
\end{frame}

\section{Strukturteilung}

\begin{frame}
  {Valenz}
  \onslide<+->
  \onslide<+->
  Letzte Woche in PSGs | Valenz doppelt in \alert{Kategorien} und \orongsch{Regeln} kodiert\\
  \Halbzeile
  \begin{itemize}[<+->]
    \item[ ] \orongsch{Regel} für Satz mit intransitivem Verb\\
      \footnotesize S \goesto NP(Per, Num, nom) \alert{V\_itr}(Per, Num)
    \item[ ] \orongsch{Regel} für Satz mit transivitem Verb\\
      \footnotesize S \goesto NP(Per1, Num1, nom) NP(Per2, Num2, akk) \alert{V\_tr}(Per1, Num1)
    \item[ ] \orongsch{Regel} für Satz mit ditransitivem Verb\\
      \footnotesize S \goesto NP(Per1, Num1, nom) NP(Per2, Num2, dat) NP(Per3, Num3, akk) \alert{V\_dtr}(Per1, Num1)
  \end{itemize}
  \onslide<+->
  \Zeile
  \alert{Typische Definition von Valenz allerdings}\\
  \onslide<+->
  Die \alert{Liste} der Ergänzungen eines Worts.\\
\end{frame}

\begin{frame}
  {Valenz als Liste}
  \onslide<+->
  \onslide<+->
  Valenz | \alert{Liste von Merkmalsbeschreibungen}\\
  \onslide<+->
  \Halbzeile
  \begin{avm}
    \[ \asort{finites-verb}
      graphen & \textit{sieht} \\
      person & dritte \\
      numerus & singular \\
      tempus & präsens \\
      modus & indikativ \\
      \alert{valenz} & \alert{\< \[ \asort{nomen} \], \[ \asort{nomen} \]\>}
    \]
  \end{avm}
\end{frame}

\begin{frame}
  {Hinreichende Beschreibung}
  \onslide<+->
  \onslide<+->
  Valenzliste | Hinreichend eingrenzende Beschreibung der Ergänzungen des Verbs\\
  \onslide<+->
  \Halbzeile
  \begin{avm}
    \[ \asort{finites-verb}
      graphen & \textit{sieht} \\
      person & dritte \\
      numerus & singular \\
      tempus & präsens \\
      modus & indikativ \\
      \alert{valenz} & \alert{\< \[ \asort{nomen}
                                    person & dritte \\
                                    numerus & singular \\
                                    kasus & nom \], \[ \asort{nomen}
                                                        kasus & akkusativ \]\>}
    \]
  \end{avm}
\end{frame}

\begin{frame}
  {Subjekt-Verb-Kongruenz und Strukturteilung}
  \onslide<+->
  \onslide<+->
  \alert{Übereinstimmung von Merkmalen} | Hart verdrahtet durch \gruen{Strukturteilung}\\
  \onslide<+->
  \Halbzeile
  \scalebox{0.8}{\begin{avm}
    \[ \asort{finites-verb}
      graphen & \textit{sieht} \\
      person & \alt<1-3>{\alert{dritte}}{\gruen{\@1 dritte}} \\
      numerus & \alt<1-3>{\alert{singular}}{\gruen{\@2 singular}} \\
      tempus & präsens \\
      modus & indikativ \\
      valenz & \< \[ \asort{nomen}
        person & \alt<1-3>{\alert{dritte}}{\gruen{\@1}} \\
        numerus & \alt<1-3>{\alert{singular}}{\gruen{\@2}} \\
                     kasus & nom \], \[ \asort{nomen}
                                         kasus & akkusativ \]\>
    \]
  \end{avm}}\\
  \onslide<+->
  \Zeile 
  \gruen{Strukturteilung bedeutet Token-Identität von Werten,} \rot{nicht Kopie}!\\
  Man kann sich die Nummern als \alert{Zeiger} auf dieselbe Datenstruktur vorstellen.
\end{frame}

\begin{frame}
  {Beispiel für Valenz einer Präposition}
  \onslide<+->
  \onslide<+->
  Valenz von Präpositionen | NP in einem bestimmten Kasus\\
  \onslide<+->
  \Halbzeile
  \begin{avm}
    \[ \asort{präposition} 
    graphen & \textit{wegen} \\
    valenz & \< \[ \asort{nomen} 
                kasus & genitiv \] \>
  \]
  \end{avm}\\
  \Zeile
  \begin{itemize}[<+->]
    \item Was ist mit \alert{argumentmarkierenden Präpositionen}\slash Präpositionalobjekten?\\
      \grau{\textit{leiden unter}, \textit{abhängen von}, \textit{glauben an} usw.}
    \item Was ist mit \alert{Wechselpräpositionen} mit Akkusativ oder Dativ?\\
      \grau{\textit{unter}, \textit{neben}, \textit{über} usw.}
  \end{itemize}
\end{frame}

\section{Phrasen und Kopfmerkmale}

\begin{frame}
  {Beispieleintrag für einen Determinierer}
  \onslide<+->
  \onslide<+->
  Kongruenzmerkmale in der NP auch beim Determinierer\\
  \onslide<+->
  \Zeile 
  \begin{avm}
    \[ \asort{determinierer} 
    graphen & \textit{des} \\
    genus & maskulin \\
    numerus & singular \\
    kasus & genitiv \\
  \]
  \end{avm}
\end{frame}

\begin{frame}
  {Determinierer in der NP}
  \onslide<+->
  \onslide<+->
  DP oder NP? | \alert{Für Deutsch ist eine NP-Analyse näherliegend.}\\
  \onslide<+->
  \Zeile
  \begin{avm}
    \[ \asort{nomen}
    graphen & \textit{Tischs} \\
    person & dritte \\
    genus & \alt<1-5>{maskulin}{\gruen{\@1 maskulin}} \\
    numerus & \alt<1-5>{singular}{\gruen{\@2 singular}} \\
    kasus & \alt<1-5>{genitiv}{\gruen{\@3 genitiv}} \\
    \alt<5->{\gruen{valenz} & \gruen{\<\[ \asort{determinierer} 
                             genus & \ \@1 \\
                             numerus & \@2 \\
                             kasus & \ \@3
                           \]\>}}{}
  \]
  \end{avm}\\
  \onslide<+->
  \Zeile
  Wie kann man \alert{Notwendigkeit von} und \alert{Kongruenz mit Determinierern} kodieren?
  \onslide<+->
  \onslide<+->
\end{frame}

\newcommand{\AvmA}{%
  \begin{avm}
    \[ \asort{nomen}
    graphen & \textit{Tischs} \\
    person & \gruen{\@1 dritte} \\
    genus & \gruen{\@2 maskulin} \\
    numerus & \gruen{\@3 singular} \\
    kasus & \gruen{\@4 genitiv} \\
    valenz & \<\[ \asort{determinierer} 
      genus & \ \gruen{\@2} \\
      numerus & \gruen{\@3} \\
      kasus & \ \gruen{\@4}
                           \]\>
  \]
  \end{avm}%
}

\newcommand{\AvmB}{%
  \begin{avm}
    \[ \asort{determinierer}
      graphen & \textit{des} \\
       genus & \gruen{\@2} \\
       numerus & \gruen{\@3} \\
       kasus & \gruen{\@4} \\
       valenz & \<\>
    \]
  \end{avm}
}

\newcommand{\AvmC}{%
  \begin{avm}
    \[ \asort{nomen} 
      graphen & \textit{des Tischs} \\
      person & \gruen{\@1} \\ 
      genus & \gruen{\@2} \\
      numerus & \gruen{\@3} \\
      kasus & \gruen{\@4} \\ 
  \]
  \end{avm}
}

\begin{frame}
  {NP mit Kongruenz als Baum}
  \onslide<+->
  \onslide<+->
  \orongsch{In HPSG gibt es eigentlich keine Bäume.} Zur Illustration aber hilfreich:\\
  \onslide<+->
  \Zeile
  \centering
  \begin{forest}
    [ \scalebox{0.6}{\AvmC}
      [ \scalebox{0.6}{\AvmB} ]
      [ \scalebox{0.6}{\AvmA} ]
    ]
  \end{forest}
\end{frame}

\begin{frame}
  {Offene Probleme}
  \onslide<+->
  \onslide<+->
  Wir haben jetzt so getan, \orongsch{als hätten wir schon eine Syntax}!\\
  \Zeile 
  \begin{itemize}[<+->]
    \item Eigentlich \alert{nur Lexikoneinträge}
    \item Fehlende \alert{Regeln für Kombinationsmechanismus}
    \item NP auf der letzten Folie | Nur eine grobe Idee, wo wir hin wollen
    \item \alert{Projektionsebenen} (N vs.\ NP) nicht unterscheidbar
    \item Also auch keine Identifikation von \alert{Köpfen}
    \item Identifikation der \alert{Merkmale, die vom Kopf zur Phrase projizieren}
    \item Zusammenbau von \textit{des Tischs} aus \textit{des} und \textit{Tischs}
  \end{itemize}
\end{frame}

\newcommand{\AvmAb}{%
  \begin{avm}
    \[ \asort{nomen}
    graphen & \textit{Tischs} \\
    \alert{kopf} & \alert{\[ 
      person & \@1 dritte \\
      genus & \@2 maskulin \\
      numerus & \@3 singular \\
      kasus & \@4 genitiv
    \]} \\
    valenz & \<\[ \asort{determinierer}
      kopf & \[ genus & \ \@2 \\
        numerus & \@3 \\
        kasus & \ \@4
      \]
      \]\>
  \]
  \end{avm}%
}

\newcommand{\AvmBb}{%
  \begin{avm}
    \[ \asort{determinierer}
      graphen & \textit{des} \\
      \alert{kopf} & \alert{\[ 
         genus & \@2 \\
         numerus & \@3 \\
         kasus & \@4
       \]} \\
       valenz & \<\>
    \]
  \end{avm}
}

\newcommand{\AvmCb}{%
  \begin{avm}
    \[ \asort{nomen} 
      graphen & \textit{des Tischs} \\
      \alert{kopf} & \alert{\[
        person & \@1 \\
        genus & \@2 \\
        numerus & \@3 \\
        kasus & \@4 
      \]}
  \]
  \end{avm}
}


\begin{frame}
  {Kopfmerkmale}
  \onslide<+->
  \onslide<+->
  \alert{Bündel der Merkmale}, die vom Kopf zur Phrase projizieren\\
  \onslide<+->
  \Halbzeile
  \centering
  \begin{forest}
    [ \alt<1-3>{\scalebox{0.6}{\AvmC}}{\scalebox{0.6}{\AvmCb}}
      [\alt<1-3>{\scalebox{0.6}{\AvmB}}{\scalebox{0.6}{\AvmBb}}]
      [\alt<1-3>{\scalebox{0.6}{\AvmA}}{\scalebox{0.6}{\AvmAb}}]
    ]
  \end{forest}
  \onslide<+->
\end{frame}

\newcommand{\AvmAc}{%
  \begin{avm}
    \[ \asort{nomen}
    graphen & \textit{Tischs} \\
    kopf & \gruen{\@5} \[ 
      person & \@1 dritte \\
      genus & \@2 maskulin \\
      numerus & \@3 singular \\
      kasus & \@4 genitiv
    \] \\
    valenz & \<\[ \asort{determinierer}
      kopf & \gruen{\@6} 
      \]\>
  \]
  \end{avm}%
}

\newcommand{\AvmBc}{%
  \begin{avm}
    \[ \asort{determinierer}
      graphen & \textit{des} \\
      kopf & \gruen{\@6} \[ genus & \ \@2 \\
        numerus & \@3 \\
        kasus & \ \@4
      \]\\
       valenz & \<\>
    \]
  \end{avm}
}

\newcommand{\AvmCc}{%
  \begin{avm}
    \[ \asort{nomen} 
      graphen & \textit{des Tischs} \\
      kopf & \gruen{\@5} 
  \]
  \end{avm}
}

\begin{frame}
  {Projizierte Kopfmerkmale}
  \onslide<+->
  \onslide<+->
  Durch Merkmalbündel | Optimale Struktur finden\slash \alert{Generalisierungen abbilden}\\
  \onslide<+->
  \Halbzeile
  \centering
  \begin{forest}
    [ \alt<1-3>{\scalebox{0.6}{\AvmCb}}{\scalebox{0.6}{\AvmCc}}
      [\alt<1-3>{\scalebox{0.6}{\AvmBb}}{\scalebox{0.6}{\AvmBc}}]
      [\alt<1-3>{\scalebox{0.6}{\AvmAb}}{\scalebox{0.6}{\AvmAc}}]
    ]
  \end{forest}
  \onslide<+->
\end{frame}

\newcommand{\AvmAd}{%
  \begin{avm}
    \[ \asort{\orongsch{wort}}
    graphen & \textit{Tischs} \\
    kopf & \gruen{\@5} \[ 
      \asort{\orongsch{nomen}}
      person & \@1 dritte \\
      genus & \@2 maskulin \\
      numerus & \@3 singular \\
      kasus & \@4 genitiv
    \] \\
    valenz & \<\[ \asort{\orongsch{wort}}
      kopf & \gruen{\@6} 
      \]\>
  \]
  \end{avm}%
}

\newcommand{\AvmBd}{%
  \begin{avm}
    \[ \asort{\orongsch{wort}}
      graphen & \textit{des} \\
      kopf & \gruen{\@6} \[
        \asort{\orongsch{determinierer}}
        genus & \ \@2 \\
        numerus & \@3 \\
        kasus & \ \@4
      \]\\
       valenz & \<\>
    \]
  \end{avm}
}

\newcommand{\AvmCd}{%
  \begin{avm}
    \[ \asort{\orongsch{phrase}} 
      graphen & \textit{des Tischs} \\
      kopf & \gruen{\@5} 
  \]
  \end{avm}
}


\begin{frame}
  {\textsc{head}-Typen}
  \onslide<+->
  \onslide<+->
  Wortartenspezifisch sind die \orongsch{\textsc{head}-Bündel}, nicht die \orongsch{Wörter\slash Phrasen}.\\
  \onslide<+->
  \Halbzeile
  \centering
  \begin{forest}
    [ \alt<1-3>{\scalebox{0.6}{\AvmCc}}{\scalebox{0.6}{\AvmCd}}
      [\alt<1-3>{\scalebox{0.6}{\AvmBc}}{\scalebox{0.6}{\AvmBd}}]
      [\alt<1-3>{\scalebox{0.6}{\AvmAc}}{\scalebox{0.6}{\AvmAd}}]
    ]
  \end{forest}
  \onslide<+->
\end{frame}

\section{Unifikation}

\begin{frame}
  {Zusammenlegen von Informationen}
  \onslide<+->
  \onslide<+->
  Beispiel | Lexikalische Spezifikation der Valenz einer Präposition\\
  \Viertelzeile
  \onslide<+->
  \scalebox{0.5}{\begin{avm}
    \[ \asort{wort} 
    graphen & \textit{wegen} \\
    kopf & \[ \asort{präposition} \] \\
    valenz & \< \gruen{\[ 
      kopf & \@1 \[ \asort{nomen}
          kasus & genitiv 
        \]
      \]} \>
  \]
  \end{avm}}\\
  \onslide<+->
  \Zeile
  Die NP kommt mit viel mehr Information daher.\\
  \onslide<+->
  \Viertelzeile
  \scalebox{0.5}{\begin{avm}
    \[ \asort{phrase} 
      graphen & \textit{des Tischs} \\
      kopf    & \gruen{\@1} \[ \asort{\gruen{nomen}}
                   person & dritte \\
                   genus & maskulin \\
                   numerus & singular \\
                   \gruen{kasus} & \gruen{genitiv}
                 \] \\
      valenz & \<\> \\
    \]
  \end{avm}}\\
  \onslide<+->
  \Zeile
  Die Informationen unter \mybox{1} sind aber kompatibel und \alert{unifizieren} daher.
\end{frame}

\begin{frame}
  {Unifikation}
  \onslide<+->
  \onslide<+->
  Bedingungen für Unifikation von zwei Merkmalstrukturen A und B\\
  \Zeile
  \begin{itemize}[<+->]
    \item A und B enthalten \alert{keine widersprüchlichen Informationen}.\\
      \Viertelzeile
      \onslide<+->
      \grau{%
        \scalebox{0.8}{\begin{avm}
          \[ kasus & nominativ \]
        \end{avm}} und %
        \scalebox{0.8}{\begin{avm}
          \[ kasus & akkusativ \]
        \end{avm}} unifizieren nicht.}
      \Halbzeile
    \item Aus nicht widersprüchlichen Informationen wird die \alert{Vereinigungsmenge} gebildet.
      \Halbzeile
    \item A kann mehr Informationen enthalten als B oder umgekehrt.\\
      \Viertelzeile
      \onslide<+->
      \grau{%
        \scalebox{0.8}{\begin{avm}
          \[ kasus & nominativ \\ person & dritte \]
        \end{avm}} und %
        \scalebox{0.8}{\begin{avm}
          \[ kasus & nominativ \]
        \end{avm}} unifizieren zu %
        \scalebox{0.8}{\begin{avm}
          \[ kasus & nominativ \\ person & dritte \]
        \end{avm}}}
      \Halbzeile
    \item A und B können beide mehr Informationen enthalten als die jeweils andere.\\
      \Viertelzeile
      \onslide<+->
      \grau{%
        \scalebox{0.8}{\begin{avm}
          \[ kasus & nominativ \\ person & dritte \]
        \end{avm}} und %
        \scalebox{0.8}{\begin{avm}
          \[ kasus & nominativ \\ numerus & singular \]
        \end{avm}} unifizieren zu %
        \scalebox{0.8}{\begin{avm}
          \[ kasus & nominativ \\ person & dritte \\ numerus & singular \]
        \end{avm}}}
  \end{itemize}
\end{frame}

\section{Nächste Woche}

\begin{frame}
  {Vorbereitung}
  \onslide<+->
  \onslide<+->
  \centering 
  \large
  Nächste Woche geht es um Valenz und Valenzabbindung.\\
  \onslide<+->
  \Zeile
  \rot{Sie sollten dringend vorher aus dem HPSG-Buch\\
  Abschnitt 3.1 und Kapitel 4 lesen!}\\
    \onslide<+->
    \Viertelzeile
  Das sind gerade mal 15 Seiten.
\end{frame}
