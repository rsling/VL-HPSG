\section{Einleitung}

\begin{frame}
  {Struktur des Lexikons und Lexikonregeln}
  \onslide<+->
  \onslide<+->
  Lexikalistische Theorien lösen so viel wie möglich im Lexikon\\
  \Zeile
  \begin{itemize}[<+->]
    \item Welche Information ist bei Wörtern wirklich idiosynkratisch (= individuell)?
    \item Wie kann man Generalisierungen im Lexikon erfassen (Typen)?
    \item Wie helfen Typenhierarchien, die sehr komplexen Lexikoneinträge zu strukturieren.
      \Halbzeile
    \item Wie funktionieren Lexikonregeln für das Passiv?
    \item Wie modelliert man Flexion und Wortbildung in HPSG? (Kapitel~19)
  \end{itemize}
  \Zeile
  \onslide<+->
  \centering 
  \grau{\citet[Kapitel~6]{MuellerLehrbuch3}}\\
\end{frame}

\section{Idiosynkrasie und Typenbildung}

\begin{frame}
  {Welche Informationen sind wirklich idiosynkratisch?}
  \onslide<+->
  \onslide<+->
  \begin{itemize}[<+->]
    \item \alert{\textit{Buch}}
      \begin{itemize}[<+->]
        \item \textsc{phon} und \textit{buch-rel}
        \item Ansonsten \alert{neutrales Zählsubstantiv}
      \end{itemize}
      \Halbzeile
    \item \alert{\textit{Zement}}
      \begin{itemize}[<+->]
        \item \textsc{phon} und \textit{zement-rel}
        \item Ansonsten \alert{maskulines Stoffsubstantiv}
      \end{itemize}
      \Halbzeile
    \item \alert{\textit{zerlegen}}
      \begin{itemize}[<+->]
        \item \textsc{phon} und \textit{zerlegen-rel}
        \item Ansonsten \alert{schwaches transitives telisches Agens\slash Patiens-Verb}
      \end{itemize}
      \Halbzeile
    \item \alert{\textit{sehen}}
      \begin{itemize}[<+->]
        \item \textsc{phon} und \textit{sehen-rel}
        \item Und die \alert{Stammformen}
        \item Ansonsten \alert{transitives atelisches Agens\slash Thema-Verb}
      \end{itemize}
  \end{itemize}
\end{frame}

\begin{frame}
  {Nomen-Typen I}
  \onslide<+->
  \onslide<+->
  Was entspricht der \alert{traditionellen Wortklasse \textit{Nomen}}?\\
  \grau{\footnotesize Wir schreiben jetzt reine Typen ohne eckige Klammern.}\\
  \onslide<+->
  \centering 
  \Zeile
  \begin{avm}
    \[ \asort{noun-sign}
      cat|head & \ty{noun}\\
      cont & \ty{nom-obj}
    \]
  \end{avm}\\
  \Doppelzeile
  \raggedright
  Die Typen \textit{noun} und \textit{nom-obj} sind dann anderswo in der Hierarchie zu spezifizieren.\\
  \grau{\footnotesize Hier nur der Illustration halber. Alle Werte haben wiederum Typen.}\\
  \onslide<+->
  \Halbzeile
  \centering 
  \raisebox{-1.3\baselineskip}{\begin{avm}
    \[ \asort{noun} cas & \ty{case} \]
  \end{avm}}\hspace{2em}\begin{avm}
    \[ \asort{nom-obj} 
      ind & \ty{index} \\
      restr & \<\[ \asort{nom-psoa} inst & \ty{index} \]\>
    \]
  \end{avm}
\end{frame}

\begin{frame}
  {Nomen-Typen II}
  \onslide<+->
  \onslide<+->
  Achtung! Die Aussagen auf der letzten Folie zu \textit{nom-obj} gelten z.\,B.\ \alert{nicht für Pronomina}.\\
  \centering
  \Halbzeile
  \scalebox{0.7}{\begin{avm}
    \[ \asort{count-appellative-noun-sign}
      cat|subcat & \<\[ cat|head & \ty{det} \]\> \\
      cont & \[ \asort{nom-obj} 
        ind & \@1 \[ per & 3 \]\\
        restr & \<\[ \asort{nom-psoa} inst & \@1 \]\>
      \]
    \]
  \end{avm}}\\
  \onslide<+->
  \Zeile
  \raggedright
  Für Feminina gilt zusätzlich:\\
  \Halbzeile
  \centering 
  \scalebox{0.7}{\begin{avm}
    \[ \asort{fem-noun-sign}
      cont|ind|gen & fem \\
    \]
  \end{avm}}\\
\end{frame}

\begin{frame}
  {Mehrfachvererbung}
  \onslide<+->
  \onslide<+->
  Aus Typen, die Teilinformationen kodieren, werden terminale spezifische Typen gebildet.\\
  \grau{\footnotesize Hier nur beispielhafte Toy-Hierarhie. Eine größere Hierarchie weiter unten und später in \textit{Trale}.}\\
  \centering 
  \Zeile
  \scalebox{0.6}{\begin{forest}
    typehierarchy,
    for tree={
      calign=fixed angles,
      calign angle=60
    }
    [ sign
      [noun-sign
        [count-appellative-noun-sign
          [\alert{fem-count-appellative-noun-sign}]
        ]
        [fem-noun-sign, calign=last
          [, identify=!r111]
          [\ldots]
        ]
        [masc-noun-sign]
        [neut-noun-sign]
      ]
    ]
  \end{forest}}%
  \onslide<+->\hspace{1em}%
  \raisebox{-2\baselineskip}{\scalebox{0.6}{\begin{avm}
    \[ \asort{\alert{fem-count-appellative-noun-sign}}
      cat & \[
        head & \ty{noun} cas & case \\
        subcat & \< \[ cat|head & det \]\> \\
      \] \\
      cont & \[ \asort{nom-obj}
        ind & \@1 \[ gen & fem \\ per & 3 \]\\
        restr & \< \[ \asort{psoa} inst & \@1 \]\>\\
      \]
    \]
  \end{avm}}}
\end{frame}

\begin{frame}
  {Verb-Typen}
  \onslide<+->
  \onslide<+->
  \begin{minipage}{0.45\textwidth}
  Verben an sich\\
  \onslide<+->
  \Viertelzeile 
  \scalebox{0.75}{\begin{avm}
    \[
      cat|head & verb \\
      cont|restr & psoa \\
    \]
  \end{avm}}\\

  \onslide<+->
  \Zeile
  Dativverben\\
  \onslide<+->
  \Viertelzeile
  \scalebox{0.75}{\begin{avm}
    \[
      cat|subcat & \< NP\Sub{nom}, NP\Sub{dat} \> \\
    \]
  \end{avm}}\\
  
  \onslide<+->
  \Zeile
  Agentivische Experiencerverben\\
  \onslide<+->
  \Viertelzeile
  \scalebox{0.75}{\begin{avm}
    \[
      cat|subcat & \< \[ cont|ind & \@1 \], \[ cont|ind & \@2 \] \> \\
      cont|restr & \[ \asort{agens-exp-rel} agens & \@1 \\ exp & \@2 \]
    \]
  \end{avm}}\\
  \end{minipage}
  \onslide<+->
  \begin{minipage}{0.5\textwidth}
    Im Ergebnis:\\

    \Halbzeile
    \centering 
    \scalebox{0.65}{%
      \begin{avm}
        \[ cat & 
          \[
            head & verb \\
            subcat & \<
              \[ cat & \[
                head & \[ \asort{noun} cas & nom \]\\
                subcat & \<\> \\
                \] \\
                cont|ind & \@1 \\
              \], \\
              \[ cat & \[
                head & \[ \asort{noun} cas & dat \]\\
                subcat & \<\> \\
                \] \\
                cont|ind & \@2 \\
              \]
            \>\\
          \] \\
          cont|restr & \< \[ \asort{agens-exp-rel} agens & \@1 \\ exp & \@2 \]\>
        \]
      \end{avm}
    }
  \end{minipage}
\end{frame}


\begin{frame}
  {Möglicher größerer Ausschnitt der Typhierarchie}
  \centering 
  \scalebox{0.5}{\begin{forest}
  typehierarchy
  [sign, l sep*=3
    [root,name=root]
    [{\begin{tabular}[t]{@{}c@{}}
     verb-sign\\
     \upshape [\head \type{verb}]\end{tabular}}, l sep+=8\baselineskip
      [nom-dat-verb-root, edge to= root, edge to=nom-dat-arg, edge to=agens-exp, l sep+=2\baselineskip
        [helf, instance]]]
    [{\begin{tabular}[t]{@{}c@{}}
     noun-sign\\
     \upshape [\head \type{noun}]\end{tabular}}, l sep+=8\baselineskip
      [count-noun-root, edge to=root, edge to=det-sc, edge to=3rd, l sep+=2\baselineskip
        [Frau, instance]]]
    [{\begin{tabular}[t]{@{}c@{}}
      saturated\\
     \upshape [\subcat \eliste]\end{tabular}}]
    [{\begin{tabular}[t]{@{}c@{}}
      unsaturated\\
     \upshape [\subcat \sliste{ [], \ldots }]\end{tabular}}
      [{\begin{tabular}[t]{@{}c@{}}
       det-sc\\
       \upshape [\subcat \sliste{ Det }]\end{tabular}},name=det-sc]
      [{\begin{tabular}[t]{@{}c@{}}
       nom-arg\\
       \upshape [\subcat \sliste{ NP[\type{nom}] }]\end{tabular}}]
      [{\begin{tabular}[t]{@{}c@{}}
       nom-dat-arg\\
       \upshape [\subcat \sliste{ NP[\type{nom}], NP[\type{acc}] }]\end{tabular}},name=nom-dat-arg]]
    [nominal-sem-sign
      [1]
      [2]
      [{\begin{tabular}[t]{@{}c@{}}
       3\\
       \upshape [\textsc{ind|per} \type{3}]\end{tabular}},name=3rd]]
    [{\begin{tabular}[t]{@{}c@{}}
      verbal-sem-sign\\
      \upshape [\cont \type{psoa}]\end{tabular}}
      [{\begin{tabular}[t]{@{}c@{}}
       agens\\
      \upshape [\cont \type{agens-rel}]\end{tabular}}
      [agens-exp, edge to=!un, name=agens-exp %before drawing tree={x/.option=!r.x}
      ]]
      [{\begin{tabular}[t]{@{}c@{}}
       experiencer\\
      \upshape [\cont \type{exp-rel}]\end{tabular}}
        ]]]
  \end{forest}}
\end{frame}

\begin{frame}
  {Maximal spezifische Typen und "`Wortklassen"'}
\end{frame}

\section{Lexikonregeln und Passiv}

\begin{frame}
  {Unäre Regeln}
\end{frame}

\begin{frame}
  {Lexikonregeln (DLR)}
\end{frame}

\begin{frame}
  {Passivregel}
\end{frame}

\begin{frame}
  {Satz mit Verb im Passiv}
\end{frame}

\section{Flexion und Wortbildung mit Lexikonregeln}

\begin{frame}
  {Plural von Nomina}
  % Selbstmachen
\end{frame}

\begin{frame}
  {Verbflexion}
  % S. 380
\end{frame}

\begin{frame}
  {Derivation mit \textit{-bar}}
  % S. 382
\end{frame}

\begin{frame}
  {Konversion mit Lexikonregel}
  % Selbstmachen
\end{frame}

\section{Nächste Woche}

\begin{frame}
  {Vorbereitung}
  \onslide<+->
  \onslide<+->
  \centering 
  \large
  Nächste Woche reden wir über Konstituentenstellung und V1-Sätze.\\
  \onslide<+->
  \Zeile
  \rot{Sie sollten dringend vorher aus dem HPSG-Buch\\
  von Kapitel 9 die Seiten 129--148 lesen!}\\
  \onslide<+->
  \Viertelzeile
  Das sind \gruen{20} Seiten.\\
  Etwas mehr als sonst, aber durchaus machbar.\\
  \onslide<+->
  \Zeile
  \rot{Achtung! In der Woche darauf sind die Seiten~163--147 dran.\\Lesen Sie ggf.\ im Voraus!}
\end{frame}
