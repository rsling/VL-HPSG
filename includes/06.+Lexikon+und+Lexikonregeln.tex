\section{Einleitung}

\begin{frame}
  {Struktur des Lexikons und Lexikonregeln}
  \onslide<+->
  \onslide<+->
  Lexikalistische Theorien lösen so viel wie möglich im Lexikon\\
  \Zeile
  \begin{itemize}[<+->]
    \item Welche Information ist bei Wörtern wirklich idiosynkratisch (= individuell)?
    \item Wie kann man Generalisierungen im Lexikon erfassen (Typen)?
    \item Wie funktionieren Lexikonregeln für das Passiv?
    \item Wie modelliert man Flexion und Wortbildung in HPSG? (Kapitel~19)
  \end{itemize}
  \Zeile
  \onslide<+->
  \centering 
  \grau{\citet[Kapitel~6]{MuellerLehrbuch3}}\\
\end{frame}

\section{Idiosynkrasie und Typenbildung}

\begin{frame}
  {Welche Informationen sind wirklich idiosynkratisch?}
  \onslide<+->
  \onslide<+->
  \begin{itemize}[<+->]
    \item \alert{\textit{Buch}}
      \begin{itemize}[<+->]
        \item \textsc{phon} und \textit{buch-rel}
        \item Ansonsten \alert{neutrales Zählsubstantiv}
      \end{itemize}
      \Halbzeile
    \item \alert{\textit{Zement}}
      \begin{itemize}[<+->]
        \item \textsc{phon} und \textit{zement-rel}
        \item Ansonsten \alert{maskulines Stoffsubstantiv}
      \end{itemize}
      \Halbzeile
    \item \alert{\textit{zerlegen}}
      \begin{itemize}[<+->]
        \item \textsc{phon} und \textit{zerlegen-rel}
        \item Ansonsten \alert{schwaches transitives telisches Agens\slash Patiens-Verb}
      \end{itemize}
      \Halbzeile
    \item \alert{\textit{sehen}}
      \begin{itemize}[<+->]
        \item \textsc{phon} und \textit{sehen-rel}
        \item Und die \alert{Stammformen}
        \item Ansonsten \alert{transitives atelisches Agens\slash Thema-Verb}
      \end{itemize}
  \end{itemize}
\end{frame}

\begin{frame}
  {Nomen-Typen}
\end{frame}

\begin{frame}
  {Verb-Typen}
\end{frame}

\begin{frame}
  {Typhierarchie}
\end{frame}

\begin{frame}
  {Maximal spezifische Typen und "`Wortklassen"'}
\end{frame}

\section{Lexikonregeln und Passiv}

\begin{frame}
  {Unäre Regeln}
\end{frame}

\begin{frame}
  {Lexikonregeln (DLR)}
\end{frame}

\begin{frame}
  {Passivregel}
\end{frame}

\begin{frame}
  {Satz mit Verb im Passiv}
\end{frame}

\section{Flexion und Wortbildung mit Lexikonregeln}

\begin{frame}
  {Plural von Nomina}
  % Selbstmachen
\end{frame}

\begin{frame}
  {Verbflexion}
  % S. 380
\end{frame}

\begin{frame}
  {Derivation mit \textit{-bar}}
  % S. 382
\end{frame}

\begin{frame}
  {Konversion mit Lexikonregel}
  % Selbstmachen
\end{frame}

\section{Nächste Woche}

\begin{frame}
  {Vorbereitung}
  \onslide<+->
  \onslide<+->
  \centering 
  \large
  Nächste Woche reden wir über Konstituentenstellung und V1-Sätze.\\
  \onslide<+->
  \Zeile
  \rot{Sie sollten dringend vorher aus dem HPSG-Buch\\
  von Kapitel 9 die Seiten 129--148 lesen!}\\
  \onslide<+->
  \Viertelzeile
  Das sind \gruen{20} Seiten.\\
  Etwas mehr als sonst, aber durchaus machbar.\\
  \onslide<+->
  \Zeile
  \rot{Achtung! In der Woche darauf sind die Seiten~163--147 dran.\\Lesen Sie ggf.\ im Voraus!}
\end{frame}
