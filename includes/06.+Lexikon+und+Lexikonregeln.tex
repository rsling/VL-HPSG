\section{Einleitung}

\begin{frame}
  {Struktur des Lexikons und Lexikonregeln}
  \onslide<+->
  \onslide<+->
  Lexikalistische Theorien lösen so viel wie möglich im Lexikon\\
  \Zeile
  \begin{itemize}[<+->]
    \item Welche Information ist bei Wörtern wirklich idiosynkratisch (= individuell)?
    \item Wie kann man Generalisierungen im Lexikon erfassen (Typen)?
    \item Wie helfen Typhierarchien, die sehr komplexen Lexikoneinträge zu strukturieren.
      \Halbzeile
    \item Wie funktionieren Lexikonregeln für das Passiv?
    \item Wie modelliert man Flexion und Wortbildung in HPSG? (Kapitel~19)
  \end{itemize}
  \Zeile
  \onslide<+->
  \centering 
  \grau{\citet[Kapitel~6]{MuellerLehrbuch3}}\\
\end{frame}

\section{Idiosynkrasie und Typbildung}

\begin{frame}
  {Welche Informationen sind wirklich idiosynkratisch?}
  \onslide<+->
  \begin{itemize}[<+->]
    \item \alert{\textit{Buch}}
      \begin{itemize}[<+->]
        \item \textsc{phon} und \textit{book-rel}
        \item Ansonsten \alert{neutrales Zählsubstantiv}
      \end{itemize}
      \Halbzeile
    \item \alert{\textit{Zement}}
      \begin{itemize}[<+->]
        \item \textsc{phon} und \textit{cement-rel}
        \item Ansonsten \alert{maskulines Stoffsubstantiv}
      \end{itemize}
      \Halbzeile
    \item \alert{\textit{zerlegen}}
      \begin{itemize}[<+->]
        \item \textsc{phon} und \textit{disassemble-rel}
        \item Ansonsten \alert{schwaches transitives telisches Agens\slash Patiens-Verb}
      \end{itemize}
      \Halbzeile
    \item \alert{\textit{sehen}}
      \begin{itemize}[<+->]
        \item \textsc{phon} und \textit{see-rel}
        \item Und die \alert{Stammformen}
        \item Ansonsten \alert{transitives atelisches Agens\slash Thema-Verb}
      \end{itemize}
  \end{itemize}
\end{frame}

\begin{frame}
  {Nomen-Typen I}
  \onslide<+->
  \onslide<+->
  Was entspricht der \alert{traditionellen Wortklasse \textit{Nomen}}?\\
  \grau{\footnotesize Wir schreiben jetzt reine Typangaben in AVMs ohne eckige Klammern.}\\
  \onslide<+->
  \centering 
  \Zeile
  \begin{avm}
    \[ \asort{noun-sign}
      cat|head & \ty{noun}\\
      cont & \ty{nom-obj}
    \]
  \end{avm}\\
  \Doppelzeile
  \raggedright
  \onslide<+->
  Die Typen \textit{noun} und \textit{nom-obj} sind dann anderswo in der Hierarchie zu spezifizieren.\\
  \grau{\footnotesize Hier nur der Illustration halber. Alle Werte haben wiederum Typen.}\\
  \onslide<+->
  \Halbzeile
  \centering 
  \raisebox{-1.3\baselineskip}{\begin{avm}
    \[ \asort{noun} cas & \ty{case} \]
  \end{avm}}\hspace{2em}\begin{avm}
    \[ \asort{nom-obj} 
      ind & \ty{index} \\
      restr & \<\[ \asort{nom-psoa} inst & \ty{index} \]\>
    \]
  \end{avm}
\end{frame}

\begin{frame}
  {Nomen-Typen II}
  \onslide<+->
  \onslide<+->
  Achtung! Die Aussagen auf der letzten Folie zu \textit{nom-obj} gelten z.\,B.\ \alert{nicht für Pronomina}.\\
  \grau{\footnotesize Pronomina führen keine \textsc{rel} ein. Substantive sind dafür immer dritte Person.}\\
  \centering
  \Halbzeile
  \onslide<+->
  \scalebox{0.7}{\begin{avm}
    \[ \asort{count-appellative-noun-sign}
      cat|subcat & \<\[ cat|head & \ty{det} \]\> \\
      cont & \[ \asort{nom-obj} 
        ind & \@1 \[ per & 3 \]\\
        restr & \<\[ \asort{nom-psoa} inst & \@1 \]\>
      \]
    \]
  \end{avm}}\\
  \onslide<+->
  \Zeile
  \raggedright
  Für Feminina gilt zusätzlich:\\
  \onslide<+->
  \Halbzeile
  \centering 
  \scalebox{0.7}{\begin{avm}
    \[ \asort{fem-noun-sign}
      cont|ind|gen & fem \\
    \]
  \end{avm}}\\
\end{frame}

\begin{frame}
  {Mehrfachvererbung}
  \onslide<+->
  \onslide<+->
  Aus Typen, die Teilinformationen kodieren, werden terminale spezifische Typen gebildet.\\
  \grau{\footnotesize Hier nur beispielhafte Toy-Hierarhie. Eine größere Hierarchie weiter unten und später in \textit{Trale}.}\\
  \centering 
  \Zeile
  \onslide<+->
  \scalebox{0.6}{\begin{forest}
    typehierarchy,
    for tree={
      calign=fixed angles,
      calign angle=60
    }
    [ sign
      [noun-sign
        [count-appellative-noun-sign
          [\alert{fem-count-appellative-noun-sign}]
        ]
        [fem-noun-sign, calign=last
          [, identify=!r111]
          [\ldots]
        ]
        [masc-noun-sign]
        [neut-noun-sign]
      ]
    ]
  \end{forest}}%
  \onslide<+->\hspace{1em}%
  \raisebox{-2\baselineskip}{\scalebox{0.6}{\begin{avm}
    \[ \asort{\alert{fem-count-appellative-noun-sign}}
      cat & \[
        head & \ty{noun} cas & case \\
        subcat & \< \[ cat|head & det \]\> \\
      \] \\
      cont & \[ \asort{nom-obj}
        ind & \@1 \[ gen & fem \\ per & 3 \]\\
        restr & \< \[ \asort{psoa} inst & \@1 \]\>\\
      \]
    \]
  \end{avm}}}
\end{frame}

\begin{frame}
  {Verb-Typen}
  \onslide<+->
  \onslide<+->
  \begin{minipage}{0.45\textwidth}
  Verben an sich\\
  \onslide<+->
  \Viertelzeile 
  \scalebox{0.75}{\begin{avm}
    \[
      cat|head & verb \\
      cont|restr & psoa \\
    \]
  \end{avm}}\\

  \onslide<+->
  \Zeile
  Dativverben\\
  \onslide<+->
  \Viertelzeile
  \scalebox{0.75}{\begin{avm}
    \[
      cat|subcat & \< NP\Sub{nom}, NP\Sub{dat} \> \\
    \]
  \end{avm}}\\
  
  \onslide<+->
  \Zeile
  Agentivische Experiencerverben\\
  \onslide<+->
  \Viertelzeile
  \scalebox{0.75}{\begin{avm}
    \[
      cat|subcat & \< \[ cont|ind & \@1 \], \[ cont|ind & \@2 \] \> \\
      cont|restr & \[ \asort{agens-exp-rel} agens & \@1 \\ exp & \@2 \]
    \]
  \end{avm}}\\
  \end{minipage}
  \onslide<+->
  \begin{minipage}{0.5\textwidth}
    Im Ergebnis:\\

    \Halbzeile
    \centering 
    \scalebox{0.65}{%
      \begin{avm}
        \[ cat & 
          \[
            head & verb \\
            subcat & \<
              \[ cat & \[
                head & \[ \asort{noun} cas & nom \]\\
                subcat & \<\> \\
                \] \\
                cont|ind & \@1 \\
              \], \\
              \[ cat & \[
                head & \[ \asort{noun} cas & dat \]\\
                subcat & \<\> \\
                \] \\
                cont|ind & \@2 \\
              \]
            \>\\
          \] \\
          cont|restr & \< \[ \asort{agens-exp-rel} agens & \@1 \\ exp & \@2 \]\>
        \]
      \end{avm}
    }
  \end{minipage}
\end{frame}


\begin{frame}
  {Möglicher größerer Ausschnitt der Typhierarchie}
  \onslide<+->
  \onslide<+->
  \centering 
  \scalebox{0.5}{\begin{forest}
  typehierarchy
  [sign, l sep*=3
    [root,name=root]
    [{\begin{tabular}[t]{@{}c@{}}
     verb-sign\\
     \upshape [\head \type{verb}]\end{tabular}}, l sep+=8\baselineskip
      [nom-dat-verb-root, edge to= root, edge to=nom-dat-arg, edge to=agens-exp, l sep+=2\baselineskip
        [helf, instance]]]
    [{\begin{tabular}[t]{@{}c@{}}
     noun-sign\\
     \upshape [\head \type{noun}]\end{tabular}}, l sep+=8\baselineskip
      [count-noun-root, edge to=root, edge to=det-sc, edge to=3rd, l sep+=2\baselineskip
        [Frau, instance]]]
    [{\begin{tabular}[t]{@{}c@{}}
      saturated\\
     \upshape [\subcat \eliste]\end{tabular}}]
    [{\begin{tabular}[t]{@{}c@{}}
      unsaturated\\
     \upshape [\subcat \sliste{ [], \ldots }]\end{tabular}}
      [{\begin{tabular}[t]{@{}c@{}}
       det-sc\\
       \upshape [\subcat \sliste{ Det }]\end{tabular}},name=det-sc]
      [{\begin{tabular}[t]{@{}c@{}}
       nom-arg\\
       \upshape [\subcat \sliste{ NP[\type{nom}] }]\end{tabular}}]
      [{\begin{tabular}[t]{@{}c@{}}
       nom-dat-arg\\
       \upshape [\subcat \sliste{ NP[\type{nom}], NP[\type{acc}] }]\end{tabular}},name=nom-dat-arg]]
    [nominal-sem-sign
      [1]
      [2]
      [{\begin{tabular}[t]{@{}c@{}}
       3\\
       \upshape [\textsc{ind|per} \type{3}]\end{tabular}},name=3rd]]
    [{\begin{tabular}[t]{@{}c@{}}
      verbal-sem-sign\\
      \upshape [\cont \type{psoa}]\end{tabular}}
      [{\begin{tabular}[t]{@{}c@{}}
       agens\\
      \upshape [\cont \type{agens-rel}]\end{tabular}}
      [agens-exp, edge to=!un, name=agens-exp %before drawing tree={x/.option=!r.x}
      ]]
      [{\begin{tabular}[t]{@{}c@{}}
       experiencer\\
      \upshape [\cont \type{exp-rel}]\end{tabular}}
        ]]]
  \end{forest}}
\end{frame}

\begin{frame}
  {Typen und "`Wortarten"'}
  \onslide<+->
  \onslide<+->
  Platitüde aus der Morphologie- oder Syntax-Einführung: \alert{Wortarten sind immer\\
  nur ein Konstrukt. Wir teilen Wörter grob so ein, wie wir es für die Grammatik brauchen.}\\
  \Zeile
  \begin{itemize}[<+->]
    \item Solche Äußerungen treffen auf \rot{nicht-formalisierte Grammatiken} zu.
    \Viertelzeile
    \item In der \alert{Formalisierung} verschwinden diese Einschränkungen:
      \begin{itemize}[<+->]
        \item \alert{Typen} erfassen Generalisierungen über Wörter und Wortformen.
        \item \alert{Konkrete Wörter} erben von diversen Typen und haben einen maximal spezifischen Typ.
        \item Wörter bringen zusätzlich \alert{idiosynkratische} Informationen mit.\\
          \grau{\footnotesize\textit{Frau} ist ein \textit{noun-sign}, \textit{det-sc}, nominal-sem-sign\slash 3 mit \textit{Frau} als \textsc{phon}-Wert.}
        \item Wortarten erfassen brutal vereinfacht Teilaspekte dieser Typhierarchie.
      \end{itemize}
      \Viertelzeile
    \item Wortarten sind Konstrukte, Typen (und Typhierarchien) sind real.
    \item Wenn Sie sonst nichts aus dieser Vorlesung übrig behalten:\\
      \alert{Daran} sollten Sie sich erinnern, wenn Sie Wortarten unterrichten.
  \end{itemize}
\end{frame}

\section{Lexikonregeln und Passiv}

\begin{frame}
  {Unäre Phrasen}
  \onslide<+->
  \onslide<+->
  Nichts verbietet unäre Projektionen in HPSG. Analog zu X-Bar-Syntax:\\
  \grau{\footnotesize Aus Kontexten wie: \textit{Wir brauchen dringend Zement.}}\\
  \onslide<+->
  \Halbzeile
  \begin{minipage}{0.45\textwidth}
    \centering 
    \begin{forest}
      [NP
        [N$'$
          [N
            [\it Zement]
          ]
        ]
      ]
    \end{forest}
  \end{minipage}%
  \onslide<+->
  \begin{minipage}{0.5\textwidth}
    Wir brauchen solche Projektionen nicht.\\

    \Halbzeile
    \centering
    \begin{avm}
      \[
        phon & \phon{Zement} \\
        cat & \[
          head & noun \\
          subcat \<\> \\
        \] \\
      \]
    \end{avm}\\

    \onslide<+->
    \Zeile
    \raggedright
    Das Wort kommt als NP aus dem Lexikon.
  \end{minipage}
\end{frame}

\begin{frame}
  {Unäre Syntaxregeln}
  \onslide<+->
  \onslide<+->
  Man kann aber \alert{unäre Regeln} einführen und daran beliebige Funktionen aufhängen.\\
  Hypothetisches Schema, das ein Stoffsubstantiv zu einem sortalen Nomen macht.\\
  \grau{\footnotesize (ein bisschen) Zement → (ein) Zement}\\
  \onslide<+->
  \Zeile
  \textit{sortal-noun-unary-phrase} $\Rightarrow$\\
  \Viertelzeile
  \begin{minipage}{0.425\textwidth}
    \scalebox{0.8}{\begin{avm}
      \[
        phon & \@2 \\
        cat & \[
          head & \@1 \\
          subcat & \< \[ cat|head & det \] \> \\
        \] \\
        cont|restr & \[ \asort{sortal-psoa-op} psoa-arg & \@3 \] \\
        single-dtr & \[ 
          phon & \@2 \\
          cat & \[ 
            head & \@1 \\
            subcat & \<\> \\
          \] \\
          cont|restr & \@3 \[ \asort{mass-rel} \] \\
        \]
      \]
    \end{avm}}
  \end{minipage}%
  \begin{minipage}{0.55\textwidth}
    \begin{itemize}[<+->]\footnotesize
      \item Die einzige Tochter ist ein Stoffsubstantiv.
      \item Es kommt ein sortales Nomen heraus (\textsc{cont}-Magie).
      \item Das sortale Nomen braucht einen Determinierer\\
        (im Gegensatz zum Stoffsubstantiv).
      \item \textsc{phon} und \textsc{head} bleiben gleich.
        \Halbzeile
      \item Das könnten wir so machen und hätten damit\\
        eine Art \alert{syntaktischer Konversion}.
      \item Probleme gäbe es, \rot{wenn das Nomen}\\
        \rot{bereits Adjunkte zu sich genommen hat}.
      \item \rot{Man vermeidet solche Regeln möglichst\\
        in der Syntax.}
    \end{itemize}
  \end{minipage}
\end{frame}

\begin{frame}
  {Description-Level-Lexical Rules (DLR)}
  \onslide<+->
  \onslide<+->
  \alert{Lexikonregeln} funktionieren ähnlich. Aber \alert{ihre Töchter sind immer Lexikoneinträge}.\\
  \onslide<+->
  \Zeile
  \centering 
  \scalebox{0.8}{%
    \begin{avm}
      \[ \asort{acc-passive-lexical-rule}
        cat & \[ 
          head|vform & passive-participle \\
          subcat & \<
            \[
              cat|head & \[ \asort{noun} cas & nom \] \\
              cont|ind & \@1 \\
            \]
          \>\ $\oplus$\ \@2 \\
        \]\\
        lex-dtr & \[ \asort{stem}
          cat & \[
            head & verb \\
            subcat & \<
              \[
                cat|head & \[ \asort{noun} cas & nom \]
              \],
              \[
                cat|head & \[ \asort{noun} cas & acc \] \\
                cont|ind & \@1 \\
              \]
            \>\ $\oplus$\ \@2 \\
          \]
        \] \\
      \]
    \end{avm}
  }\\
  \Halbzeile
  \onslide<+->
  \grau{\footnotesize Deswegen erkläre ich in \citet{Schaefer2018a}, dass Passiv lexikalisch ist.\\
    Vollständige Argumentation: \citet{AW98a}}
\end{frame}

\begin{frame}
  {Lexikonregeln}
  \onslide<+->
  \onslide<+->
  Wir betrachten hier nur DLR-Lexikonregeln.\\
  \grau{\footnotesize Alternativen s.\ \citet[Kapitel~6]{MuellerLehrbuch3}.}\\
  \Halbzeile
  \begin{itemize}[<+->]
    \item Es gibt keinen formalen Unterschied zwischen Morphologie und Syntax.
    \item Lexikonregeln sind Teil des Formalismus.
    \item Sie sind unäre Regeln, die auf Lexikoneinträge beschränkt sind.
    \item Die \textsc{lex-dtr} ist der lexikalische Input.
    \item Alles, worüber die Regel nichts aussagt, wird übernommen.
      \Halbzeile
    \item So funktionieren Passiv, Flexion, Wortbildung usw. in HPSG.
    \item Theorien wie HPSG sind \alert{Theorien des gesamten Sprachsystems} inkl.\ Lexikon,\\
      sie sind \rot{keine reinen Syntaxen im engen Sinn}.
  \end{itemize}
\end{frame}

\begin{frame}
  {Passiv mit Morphonolomagie}
  \onslide<+->
  \onslide<+->
  Um die Form kümmert sich $f$\,!\\
  \onslide<+->
  \Halbzeile
  \centering 
  \scalebox{0.7}{%
    \begin{avm}
      \[ \asort{acc-passive-lexical-rule}
        \alert{phon} & \alert{f(\@1)} \\
        cat & \[ 
          head|vform & passive-participle \\
          subcat & \<
            \[
              cat|head & \[ \asort{noun} cas & nom \] \\
              cont|ind & \@1 \\
            \]
          \>\ $\oplus$\ \@2 \\
        \]\\
        lex-dtr & \[ \asort{stem}
          \alert{phon} & \alert{\@1} \\
          cat & \[
            head & verb \\
            subcat & \<
              \[
                cat|head & \[ \asort{noun} cas & nom \]
              \],
              \[
                cat|head & \[ \asort{noun} cas & acc \] \\
                cont|ind & \@1 \\
              \]
            \>\ $\oplus$\ \@2 \\
          \]
        \] \\
      \]
    \end{avm}
  }\\
  \onslide<+->
  \Halbzeile
  Die Funktion \alert{$f$} baut die Form \textit{gekauft} zu \textit{kauf} usw.\\
  \onslide<+->
  \rot{Und starke Verben?}
\end{frame}

\begin{frame}
  {Starke Verben}
  \onslide<+->
  \onslide<+->
  Wenn man nicht $f$ noch mehr externes Wissen mitgeben will, muss man irgendwo\\
  die \alert{Information über die Stammallomorphie} in \textit{stem}-Typen repräsentieren.\\
  \onslide<+->
  \Halbzeile
  \small Starke Verben:\\
  \scalebox{0.6}{\begin{avm}
    \[ \asort{verb-stem-phon}
      pres-stem & \phon{geb} \\
      pret-stem & \phon{gab} \\
      part-stem & \phon{geb} \\
    \]
  \end{avm}}\\
  \onslide<+->
  \Zeile
  \small Alternativ die Information über das Ablautmuster für $f$ hinterlegen:\\
  \scalebox{0.6}{
    \begin{avm}
      \[ \asort{stem}
        phon & \[ \asort{verb-ablaut-eae-phon} \phon{g\_b} \]
      \]
    \end{avm}}\\
  \onslide<+->
  \Zeile
  \small Nomina:\\
  \scalebox{0.6}{\begin{avm}
    \[ \asort{noun-stem-phon}
      sg-stem & \phon{Haus} \\
      pl-stem & \phon{Häus} \\
    \]
  \end{avm}}\hspace{1em}%
  \scalebox{0.6}{\begin{avm}
    \[ \asort{noun-stem-phon}
      sg-alloform & \phon{Haus} \\
      pl-alloform & \phon{Häuser} \\
    \]
  \end{avm}}\\
  \raggedleft
  \grau{\footnotesize Siehe \citet{Crysmann2021a} für richtige Morphologie in HPSG.}
\end{frame}

\section{Flexion und Wortbildung mit Lexikonregeln}

\begin{frame}
  {Plural von Nomina}
  \onslide<+->
  \onslide<+->
  Worin besteht Pluralbildung bei Nomina? \onslide<+-> -- Formänderung und \textsc{num}:\\
  \onslide<+->
  \Zeile
  \centering 
  \scalebox{0.7}{%
    \begin{avm}
      \[ \asort{noun-plural-lex-rule} 
        phon & \@1 \\
        cat & \@2 \[
          subcat & \< \[ cat|head|numform & pl \] \>
        \] \\
        cont & \@3 \[ ind|num & pl \] \\
        lex-dtr & \[ \asort{stem}
          phon & \[ pl-alloform & \@1 \] \\
          cat & \@2 \\
          cont & \@3 \\
        \]
      \]
    \end{avm}
  }\\
  \raggedright
  \Zeile
  \begin{itemize}[<+->]
    \item \textsc{phon} ist hier für \textit{stem} komplex.
    \item Die eigentliche Quantifikation macht der Quantor (Artikel\slash Determinierer).
    \item Der Quantor muss aber ein pluralischer (\textit{zwei}, \textit{mehrere}, \ldots) sein.\\
      \grau{Das wurde hier behelfsmäßig mit \textsc{numform} implementiert.}
  \end{itemize}
\end{frame}

\begin{frame}
  {Verbflexion}
  \onslide<+->
  \onslide<+->
  Worin besteht Verbflexion? \onslide<+-> -- Formänderung, Tempus, Person, Numerus\grau{(, Modus)}\\
  \grau{\footnotesize Vereinfachung für unsere Zwecke nach \citet[380]{MuellerLehrbuch3}.}\\
  \onslide<+->
  \Halbzeile
  \centering 
  \scalebox{0.75}{%
    \begin{avm}
      \[ \asort{fin-verb-infl-lexical-rule}
        phon & f(\@1, \phon{st}) \\
        cat & \@2 \[
          head|vform & fin\\
          subcat & \< \[
            cat|head & \[ \asort{noun} cas & nom \] \\
            cont|ind & \[ per & 2 \\ num & sg \] \\
          \]\>\ $\oplus$\ \@5 \\
        \] \\
        cont & \[ \asort{present-psoa-op} psoa-arg & \@4 \] \\
      lex-dtr & \[ \asort{stem}
        phon & \@1 \\
        cat & \@2 \[ head & verb \] \\
        cont|restr & \@4 \\
      \]
      \] 
    \end{avm}
  }
\end{frame}

\begin{frame}
  {Derivation mit \textit{-bar}}
  \onslide<+->
  \onslide<+->
  \centering 
  \scalebox{0.65}{%
    \begin{avm}
      \[ \asort{reg-bar-adj-stem}
        phon & \@1\ $\oplus$\ \phon{bar} \\
        cat & \[
          head & \[ \asort{adj} mod & \[
            cat|head & noun \\
            cont|ind & \@2 \\
            \]
          \]\\
          subcat & \< \[ cat|head & \[ \asort{prep} pform & von $\vee$\ durch\] \\ cont|ind & \@5 \]\>\ $\oplus$\ \@3\\
        \] \\
        cont|restr & \[ \asort{possible-psoa-op}
          psoa-arg & \@4 \\
        \] \\
      lex-dtr & \[ \asort{stem}
        phon & \@1 \\
        cat & \[
          head & verb \\
          subcat & \<\[ cat|head & \[ \asort{noun} cas & nom\] \\ cont|ind & \@5\], \[cat|head & \[ \asort{noun} cas & acc\] \\ cont|ind & \@2\]\>\ $\oplus$\ \@3 \\
        \] \\
        cont|restr & \@4 \\
      \]
    \]
    \end{avm}
  }\\
  \Viertelzeile
  \onslide<+->
  \grau{\footnotesize Die Version in \citet[382]{MuellerLehrbuch3} ist allgemeiner, aber dadurch schwerer nachvollziehbar.}
\end{frame}


\begin{frame}
  {Ein Beispiel}
  \onslide<+->
  \onslide<+->
  Beispielkontext: \textit{die durch mich \alert{lösbare} Gleichung}\\
  \onslide<+->
  \centering 
  \Zeile 
  \scalebox{0.6}{%
    \begin{avm}
      \[
        \asort{reg-bar-adj-stem}
        phon & \phon{lös,bar} \\
        cat & \[
          head & \[
            \asort{adj}
            mod & \[
              cat|head & noun \\
              cont|ind & \@2 \\
            \]
          \] \\
          subcat & \<\[
            cat|head & \[
              \asort{prep}
              pform & durch \\
            \] \\
            cont|ind & \@5 \\
          \]\> \\
        \] \\
        cont|restr & \[ \asort{possible-psoa-op} psoa-arg & \@4 \] \\
        lex-dtr & \@{100} \\
      \]
    \end{avm}
  }\hspace{1em}\onslide<+->\raisebox{-1.5\baselineskip}{\scalebox{0.6}{%
    \begin{avm}
      \@{100} \[
        \asort{verb-stem} 
        phon & \phon{lös} \\
        cat|subcat & \<
            \[
              cat|head|cas & nom \\
              cont|ind & \@5 \\
            \],
            \[
              cat|head|cas & acc \\
              cont|ind & \@2 \\
            \]
          \> \\
        cont|restr & \@4 \<\[
            \asort{solve-rel}
            agent & \@5 \\
            other &  \@2 \\
          \]\> \\
      \]
    \end{avm}
  }}
\end{frame}

\section{Nächste Woche}

\begin{frame}
  {Vorbereitung}
  \onslide<+->
  \onslide<+->
  \centering 
  \large
  Nächste Woche reden wir über Konstituentenstellung und V1-Sätze.\\
  \onslide<+->
  \Zeile
  \rot{Sie sollten dringend vorher aus dem HPSG-Buch\\
  von Kapitel 9 die Seiten 129--148 lesen!}\\
  \onslide<+->
  \Viertelzeile
  Das sind \gruen{20} Seiten.\\
  Etwas mehr als sonst, aber durchaus machbar.\\
  \onslide<+->
  \Zeile
  \rot{Achtung! In der Woche darauf sind die Seiten~163--147 dran.\\Lesen Sie ggf.\ im Voraus!}
\end{frame}
