
\begin{frame}
  {Ziele}
  \onslide<+->
  \onslide<+->
  Worum geht es heute?\\
  \Zeile
  \begin{itemize}
    \item Vermittlung grundlegender Vorstellungen über deutsche Syntax
    \item Vorstellung für die Daten, Zusammenhänge und Komplexität
    \item Einführung in Grundannahmen in der HPSG
    \item Befähigung zum Schreiben formaler Grammatiken
  \end{itemize}
  \Zeile
  \centering 
  \grau{\citet[Kapitel~1]{MuellerLehrbuch} bzw.\ \citet[Kapitel~1]{MuellerGTBuch}\\
    Englische Version des Grammatiktheoriebuches: \citet[Kapitel~1]{MuellerGT-Eng}}
\end{frame}



\section{Wozu (formale) Syntax?}

\begin{frame}
  {Wozu Syntax?}
  \onslide<+->
  \begin{itemize}[<+->]
    \item \alert{Zeichen} | Form-Bedeutungs-Paare \citep{Saussure16a-Fr}
    \item Wörter, Wortgruppen, Sätze
    \item Sprache | \rot{keine} (endliche) \rot{Aufzählung} von Wortfolgen\\
      \grau{Endlichkeit von Sprache bei Annahme einer maximalen Satzlänge}
      \begin{exe}
        \ex Dieser Satz geht weiter und weiter und weiter und weiter \ldots
        \ex {}[Ein Satz ist ein Satz] ist ein Satz.
      \end{exe}
      \Halbzeile
    \item Auf jeden Fall \alert{sehr viele Sätze}, Unendlichkeitsproblem als Scheinfrage
    \item \alert{Kompetenz} | (implizites) Wissen um grammatische Regularitäten
    \item \alert{Performanz} | Nutzung des Wissens, Sprachproduktion
      \Halbzeile
    \item \alert{Kreativität} | Sätze bilden, die man nie zuvor gehört hat
  \end{itemize}
\end{frame}
 
 
\begin{frame}
  {Die Kinder im Rumpedorf}
  \onslide<+->
  \onslide<+->
  Schon Kindern kann man ein Spiel um Kompetenz und Performanz zumuten!\\
  \Zeile
  \onslide<+->
  \begin{quote}
    Und wir beeilten uns, den Jungen zu erzählen, wir hätten von Anfang an gewußt, daß es nur eine
    Erfindung von Lasse gewesen sei. Und da sagte Lasse, die Jungen hätten gewußt, daß wir gewußt
    hätten, es sei nur eine Erfindung von ihm. Das war natürlich gelogen, aber vorsichtshalber sagten
    wir, wir hätten gewußt, die Jungen hätten gewußt, daß wir gewußt hätten, es sei nur eine Erfindung
    von Lasse. Und da sagten die Jungen -- ja -- jetzt schaffe ich es nicht mehr aufzuzählen, aber es
    waren so viele "`gewußt"', daß man ganz verwirrt davon werden konnte, wenn man es hörte.
  \end{quote}
  \Zeile
  \begin{itemize}[<+->]
    \item \alert{Grammatikalität} der Sätze | Einwandfrei feststellbar
    \item \alert{Akzeptabilität} der Sätze | Vermindert durch \rot{Performanzeffekte}
  \end{itemize}
\end{frame}

\begin{frame}
  {Wozu Syntax? Bedeutung aus Bestandteilen ermitteln}
  \onslide<+->
  \onslide<+->
  Bedeutung einer Äußerung aus den Bedeutungen ihrer Teile bestimmen\\
  \Viertelzeile
  \begin{exe}
    \ex Der Mann kennt den Kollegen.
  \end{exe}
  \Halbzeile
  \alert{Syntax} | Art und Weise der Kombination, Strukturierung\\
  \Viertelzeile
  \begin{exe}
    \ex
    \begin{xlist}
      \ex Die Frau kennt die Kolleginnen.
      \ex Die Frau kennen die Kolleginnen.
    \end{xlist}
    \ex
    \begin{xlist}
      \ex Die Frau schläft.
      \ex Die Kolleginnen schlafen.
    \end{xlist}
  \end{exe}
  \Halbzeile
  \onslide<+->
  \begin{block}
    {Das Frege-Prinzip (Gottlob Frege}
    Die Bedeutung eines Satzes ergibt sich aus der Bedeutung seiner Konstituenten\\
    und der Art ihrer Kombination.
  \end{block}
\end{frame}
 

\begin{frame}
  {Warum formal?}
  \begin{quote}\small
    Precisely constructed models for linguistic structure can play an
    important role, both negative and positive, in the process of discovery 
    itself. By pushing a precise but inadequate formulation to
    an unacceptable conclusion, we can often \alert{expose the exact source
    of this inadequacy and, consequently, gain a deeper understanding}
    of the linguistic data. More positively, a formalized theory may 
    \alert{automatically provide solutions for many problems other than those
    for which it was explicitly designed}. Obscure and intuition-bound
    notions can neither lead to absurd conclusions nor provide new and
    correct ones, and hence they fail to be useful in two important respects. 
    I think that some of those linguists who have questioned
    the value of precise and technical development of linguistic theory
    have failed to recognize the productive potential in the method
    of rigorously stating a proposed theory and applying it strictly to
    linguistic material with no attempt to avoid unacceptable conclusions by ad hoc adjustments or loose formulation.
\citep[S.\,5]{Chomsky57a}
  \end{quote}
  \onslide<+->
  \Halbzeile
  \begin{quote}\small
    As is frequently pointed out but cannot be overemphasized, an important goal
    of formalization in linguistics is to \alert{enable subsequent researchers to see the defects
    of an analysis as clearly as its merits}; only then can progress be made efficiently.
    \citep[S.\,322]{Dowty79a}
  \end{quote}
\end{frame}


\section{Konstituenz}

\begin{frame}
  {Einteilung in Einheiten}
  \onslide<+->
  \onslide<+->
  \alert{Parataxe} | Einbettung von ganzen Satzstrukturen\\
  \Viertelzeile
  \begin{exe}
    \ex dass Max glaubt, [dass Julius weiß, [dass Otto behauptet, [dass Karl vermutet, [dass Richard bestätigt, [dass Friederike lacht]]]]]
  \end{exe}
  \onslide<+->
  \Zeile
  Parataxe als Spezialfall | \alert{Konstitueten in Konstituenten}\\
  \Viertelzeile
  \begin{exe}
    \ex {}[das Haus [des Autors [von Zettel Traum [den ich 1993 gelesen habe]]]]
    \ex {}[[den][ich][1993][[gelesen]habe]]
  \end{exe}
\end{frame}

\begin{frame}
  {Naive Konstituenzanalyse}
  \onslide<+->
  \onslide<+->
  \centering
  \scalebox{0.8}{\begin{forest}
    [NP
      [Artikel
        [das]
      ]
      [N$'$
        [N
          [Haus]
        ]
        [NP
          [Artikel
            [des]
          ]
          [N$'$
            [N
              [Autors]
            ]
            [PP
              [P
                [von]
              ]
              [NP
                [NP
                  [Zettels Traum]
                ]
                [Relativsatz
                  [den \ldots\ habe]
                ]
              ]
            ]
          ]
        ]
      ]
    ]
  \end{forest}}
\end{frame}


\begin{frame}
  {Konstituententests}
  \onslide<+->
  \onslide<+->
  Welche \alert{Konstituententests} kennen Sie?\\
  \Zeile 
  \begin{itemize}[<+->]
    \item Substituierbarkeit\slash Pronominalisierungstest\slash Fragetest
    \item Weglaßtest
    \item Verschiebetest (Umstelltest)\slash Vorfeldtest
    \item Koordinationstest
  \end{itemize}
\end{frame}


\begin{frame}
  {Konstituententests I}
  \begin{description}
  \item[Substituierbarkeit]
    Ausstauschbare Wortfolgen als potenzielle Konstituenten
    \begin{exe}
      \ex Er kennt \orongsch{den Mann}.
      \ex Er kennt \orongsch{eine Frau}.
    \end{exe}
    \Zeile
    \onslide<+->
  \item[Pronominalisierungstest]
    Dasselbe, aber spezifisch mit pronominalen Ein-Wort-Folgen
    \begin{exe}
      \ex \orongsch{Der Mann} schläft.
      \ex \orongsch{Er} schläft.
    \end{exe}
  \end{description}
\end{frame}

\begin{frame}
  {Konstituententests II}
  \begin{description}
    \item[Fragetest]
    Erfragbarkeit von Konstituenten
      \begin{exe}
        \ex \orongsch{Der Mann} arbeitet.
        \ex \orongsch{Wer} arbeitet?
      \end{exe}
      \onslide<+->
      \Halbzeile
  \item[Verschiebetest] 
   Umstellbarkeit von Konstituenten 
      \begin{exe}
        \ex weil \orongsch{keiner} \gruen{diese Frau} kennt.
        \ex weil \gruen{diese Frau} \orongsch{keiner} kennt.
      \end{exe}
      \onslide<+->
      \Halbzeile
  \item[Koordinationstest]
    Konstituenten als koordinierbar
      \begin{exe}
        \ex \alert{[}\orongsch{[Der Mann]} \alert{und} \orongsch{[die Frau]}\alert{]} arbeiten.
      \end{exe}
    \end{description}
\end{frame}


\section{Köpfe}



\begin{frame}
  {Köpfe}
  \onslide<+->
  \onslide<+->
  \alert{Kopf} | Festlegung der syntaktisch relevanten \alert{kategorialen Merkmale der Phrase}\\
  \onslide<+->
  \Halbzeile
  \begin{exe}
    \ex \alert{Träumt} er?
    \ex \alert{Erwartet} er einen dreiprozentigen Anstieg?
    \ex \alert{in} diesem Haus
    \ex ein \alert{Mann}
  \end{exe}
  \onslide<+->
  \Halbzeile
  \begin{itemize}[<+->]
    \item \alert{Projektion} | Kombination eines Kopfes mit anderem Material
    \item \alert{Maximalprojektion} | Vollständige Projektion
    \item \alert{Satz} | Maximalprojektion eines finiten Verbs
  \end{itemize}
\end{frame}

\begin{frame}
  {Naive Konstituenzanalyse mit Markierung der Köpfe}
  \onslide<+->
  \onslide<+->
  \centering
  \scalebox{0.8}{\begin{forest}
    [NP, name=NP2, blue
      [Artikel
        [das]
      ]
      [N$'$, name=N2b, blue
        [N, blue, name=N2a
          [Haus, blue]
        ]
        [NP, name=NP1, green
          [Artikel
            [des]
          ]
          [N$'$, green, name=N1b
            [N, green, name=N1a
              [Autors, green]
            ]
            [PP, name=PP, orange
              [P, name=P, orange
                [von, orange]
              ]
              [NP, red, name=NP3b
                [NP, red, name=NP3a
                  [Zettels Traum, red]
                ]
                [Relativsatz
                  [den \ldots\ habe]
                ]
              ]
            ]
          ]
        ]
      ]
      {\draw [->, bend left=45, orange, thick] (P.north) to (PP.west);}
      {\draw [->, bend left=45, green, thick] (N1a.north) to (N1b.west);}
      {\draw [->, bend right=45, green, thick] (N1b.north) to (NP1.east);}
      {\draw [->, bend left=45, blue, thick] (N2a.north) to (N2b.west);}
      {\draw [->, bend right=45, blue, thick] (N2b.north) to (NP2.east);}
      {\draw [->, bend left=45, red, thick] (NP3a.north) to (NP3b.west);}
    ]
  \end{forest}}
\end{frame}


\begin{frame}
  {Generalisierung durch Phrasenbildung}
  \onslide<+->
  Der \alert{interne Aufbau} einer Phrase ist für den Kontext \alert{irrelevant}:\\
  \onslide<+->
  \Viertelzeile
  \begin{exe}
    \ex er
    \ex der Mann
    \ex der Mann aus Stuttgart
    \ex der Mann aus Stuttgart, den wir kennen
  \end{exe}
  \onslide<+->
  \Halbzeile
  Bestimmte \alert{Merkmale} des Kopfs sind aber \alert{kontextrelevant}:\\
  \begin{exe}
   \ex[]{Der Kollege liest einen Aufsatz.} 
   \ex[*]{Die Kollegen liest einen Aufsatz.} 
   \ex[*]{ Des Kollegen liest einen Aufsatz.} 
  \end{exe} 
\end{frame}


\begin{frame}
  {Naive Konstituenzanalyse mit Projektion von Kopfmerkmalen}
  \onslide<+->
  \onslide<+->
  \centering
  \scalebox{0.6}{\begin{forest}
    [NP\\\footnotesize Nom Sg Neut, name=NP2, blue
      [Artikel
        [das]
      ]
      [N$'$\\\footnotesize Nom Sg Neut, name=N2b, blue
        [{N\\\footnotesize Nom Sg Neut}, blue, name=N2a
          [Haus, blue]
        ]
        [{NP\\\footnotesize Gen Sg Mask}, name=NP1, green
          [Artikel
            [des]
          ]
          [{N$'$\\\footnotesize Gen Sg Mask}, green, name=N1b
            [{N\\\footnotesize Gen Sg Mask}, green, name=N1a
              [Autors, green]
            ]
            [{PP\\\footnotesize ?}, name=PP, orange
              [{P\\\footnotesize ?}, name=P, orange
                [von, orange]
              ]
              [{NP\\\footnotesize Dat Sg Mask}, red, name=NP3b
                [{NP\\\footnotesize Dat Sg Mask}, red, name=NP3a
                  [Zettels Traum, red]
                ]
                [Relativsatz
                  [den \ldots\ habe]
                ]
              ]
            ]
          ]
        ]
      ]
      {\draw [->, bend left=45, orange, thick] (P.north) to (PP.west);}
      {\draw [->, bend left=45, green, thick] (N1a.north) to (N1b.west);}
      {\draw [->, bend right=45, green, thick] (N1b.north) to (NP1.east);}
      {\draw [->, bend left=45, blue, thick] (N2a.north) to (N2b.west);}
      {\draw [->, bend right=45, blue, thick] (N2b.north) to (NP2.east);}
      {\draw [->, bend left=45, red, thick] (NP3a.north) to (NP3b.west);}
    ]
  \end{forest}}
\end{frame}


\section{Argumente und Adjunkte}

\begin{frame}
  {Valenz und logische Argumente}
  \onslide<+->
  \onslide<+->
  Nicht alle Phrasen, die vom Verb abhängen, stehen in derselben Art Relation zu ihm.\\
  \Zeile
  \begin{itemize}[<+->]
    \item Konstituenten | Verschiedenartige Beziehungen zu ihrem Kopf
    \item Semantische Beteiligte -- \alert{Aktanten} -- als \alert{feste Teile der Verbbedeutung}
    \item Semantik von \textit{sehen} | Immer ein \alert{Sehender}, ein \alert{Gesehenes}
      \Viertelzeile
      \begin{exe}
       \ex Dani sieht den Chaoten.
      \end{exe}
    \item \alert{Logische Argumente von \textit{sehen}} | Dani und der Chaot
    \item Valenz | Abbildung logischer Argumente auf grammatische Argumente
  \end{itemize}
\end{frame}

\begin{frame}
  {Optionale Argumente}
  \onslide<+->
  \onslide<+->
  Semantische Argumente | Nicht immer syntaktisch erforderlich\\
  \onslide<+->
  \Halbzeile
    \begin{exe}
      \ex Er wartet \alert{auf den Installateur}.
      \ex Er wartet.
    \end{exe}
  \onslide<+->
  \Zeile
  Bei \alert{Nominalisierung} | Alle Argumente optional\\
  \onslide<+->
  \Halbzeile
    \begin{exe}
      \ex \gruen{Arno} \alert{liest} \orongsch{diese Bücher}.
      \ex \alert{das Lesen} \orongsch{dieser Bücher} \gruen{durch Arno}
      \ex \alert{das Lesen} \orongsch{dieser Bücher} 
      \ex \alert{das Lesen} 
    \end{exe}
\end{frame}



\begin{frame}
  {Syntaktische Argumente, die keine logischen sind}
  \onslide<+->
  \onslide<+->
  Oben waren alle \alert{syntaktischen Argumente} auch \gruen{logische Argumente}.\\
  \Halbzeile
  \begin{exe}
    \ex \alert{Dani} sieht \alert{den Chaoten}.
  \end{exe}
  \onslide<+->
  \Zeile
  \orongsch{Syntaktische Argumente}, die keine logischen sind:\\
  \Halbzeile
  \onslide<+->
  \begin{exe}
    \ex \orongsch{Es} regnet.
    \ex Conny erholt \orongsch{sich}.
  \end{exe}
\end{frame}

\begin{frame}
  {Adjunkte}
  \onslide<+->
  \onslide<+->
  \alert{Adjunkte} | Keine verbgebundene, sondern \alert{selbst mitgebrachte} Rolle\\
  \onslide<+->
  \Halbzeile
  \begin{exe}
    \ex \alert{Dani} sieht \alert{den Chaoten} \orongsch{bellend} \rot{auf der Brücke}.
  \end{exe}
  \onslide<+->
  \Zeile
  Deutliche Unterschiede zwischen Argumenten und Adjunkten\\
  \Halbzeile
  \begin{itemize}[<+->]
    \item \alert{Sehende und Gesehener} | Fester Teil einer \textit{sehen}-Situation
    \item \rot{Ort} | Teil so ziemlich jedes Geschehens, nicht \textit{sehen}-spezifisch
    \item \orongsch{Verhalten des Beteiligten} | Erst recht nicht \textit{sehen}-spezifisch
  \end{itemize}
\end{frame}
 
\begin{frame}
  {Andere Bezeichnungen}
  \onslide<+->
  \onslide<+->
  Üblicher Terminologie-Wildwuchs in der Linguistik\\
  \Zeile
  \begin{itemize}[<+->]
    \item Argument = \alert{Ergänzung}
    \item Adjunkt = \alert{(freie) Angabe}
      \Halbzeile
    \item Argumente | Beim Verb aufgeteilt in \alert{Subjekte} und \alert{Komplemente} 
    \item \alert{Aktant} Subjekte und Objekte (nicht Prädikative und Adverbiale)
      \Halbzeile
    \item \alert{Adverbial} | Angabe beim Verb
      \begin{itemize}
        \item Raum (Lage, Richtung/Ziel, Herkunft, Weg)
        \item Zeit (Zeitpunkt, Anfang, Ende, Dauer)
        \item Grund (inkl.\ Gegengrund, Bedingung)
        \item Art und Weise
      \end{itemize}
  \end{itemize}
\end{frame}


\section{Grammatische Funktionen}

\begin{frame}
  {Grammatische Funktionen (eigentlich Relationen)}
  \onslide<+->
  \onslide<+->
  Grammatische Funktionen\slash Relationen sind oft nicht unabhängig definierbar!\\
  \Zeile
  \begin{itemize}[<+->]
    \item Typen von Argumenten\slash Adjunkten mit spezifischen Eigenschaften
      \Halbzeile
    \item \alert{Subjekt} | Siehe nächste Folien
    \item \alert{Objekt}\slash \alert{Komplement} | Nicht-Nominativ-Argumente
    \item \alert{Adverb}\slash \alert{Adverbiale Bestimmung} | Angabe des Verbs
  \end{itemize}
\end{frame}

\subsection{Subjekt}

\begin{frame}
  {Subjekt}
  \onslide<+->
  \onslide<+->
  Für \alert{deutsche Subjekte} benannte definitorische Kriterien:\\
  \Halbzeile
  \begin{enumerate}[<+->]
    \item \alert{Kongruenz} mit dem finiten Verb
    \item \alert{Nominativ} in nichtkopulativen Sätzen
    \item Weglassbarkeit in \alert{Infinitivkonstruktionen} (Kontrolle)
    \item Weglassbarkeit in \alert{Imperativsätzen}
  \end{enumerate}
  \Zeile
  \onslide<+->
  \citet{Reis82} | Nur (2) relevant!
\end{frame}


\begin{frame}
  {Dative sind keine Subjekte}
  \onslide<+->
  \onslide<+->
  Kongruenz:\\
  \onslide<+->
  \Viertelzeile
  \begin{exe}
    \ex[]{Er hilft den Männern.}
    \ex[]{Den Männern wurde geholfen.}
    \ex[*]{Den Männern wurden geholfen.}
  \end{exe}
  \onslide<+->
  \Halbzeile
  Keine Kontrolle in Infinitivkonstruktionen:\\
  \onslide<+->
  \Viertelzeile
  \begin{exe}
   \ex[]{Klaus behauptet, den Männern zu helfen.}
   \ex[]{Klaus behauptet, dass er den Männern hilft.}
   \ex[]{Klaus behauptet, seine Familie zu lieben.}
   \ex[]{Seine Familie behauptet, geliebt zu werden.}
   \ex[*]{Die Männer behaupten, geholfen zu werden.}
   \ex[*]{Die Männer behaupten, elegant getanzt zu werden.}
  \end{exe}
\end{frame}


\begin{frame}
  {Dative sind keine Subjekte}
  \onslide<+->
  \onslide<+->
  Weglassbarkeit in Imperativen:\\
  \Halbzeile
  \onslide<+->
  \begin{exe}
    \ex[]{Fürchte dich nicht!}
    \ex[*]{Graue nicht!}
    \ex[]{Werd einmal unterstützt und \ldots}
    \ex[*]{Werd einmal geholfen und \ldots}
  \end{exe}
\end{frame}

\section{Phrasenstrukturgrammatiken}

\begin{frame}
  {Phrasenstrukturen}
  \centering 
  \begin{tabular}{@{}l@{\hspace{1cm}}l@{}}
  \scalebox{.75}{%
  \begin{forest}
    [S
      [NP [er] ]
      [NP
        [Det [das] ]
        [N [Buch] ] 
      ]
      [NP
        [Det [dem] ]
        [N [Mann] ] 
      ]
      [V [gibt] ]
    ]
  \end{forest}} &
  \scalebox{.75111111}{%
  \begin{forest}
    [V
      [NP [er] ]
      [V
        [NP
          [Det [das] ]
          [N [Buch] ] ]
        [V
          [NP
            [Det [dem] ]
            [N [Mann] ] ]
          [V [gibt] ] ] ] ]
  \end{forest}}
  \\
  \\
  \onslide<+->
    \begin{tabular}{@{~}l@{ }l@{}}
      NP & $\to$ Det, N            \\
      S  & $\to$ NP, NP, NP, V  \\
    \end{tabular} & \begin{tabular}{@{~}l@{ }l@{}}
      NP & $\to$ Det, N  \\
      V  & $\to$ NP, V\\
    \end{tabular}\\
  \end{tabular}
\end{frame}


% \begin{frame}
%   {Beispielableitung bei Annahme flacher Strukturen}
% 
%   \vfill
% 
%   \bigskip
%   \parskip0pt
%   \begin{tabular}[t]{@{}l@{ }l}
%   \highlight{NP}<5,8> & \highlight{$\to$ Det N}<5,8>\\          
%   \highlight{S}<10>  & \highlight{$\to$ NP NP NP V}<10>
%   \end{tabular}\hspace{2cm}%
%   \begin{tabular}[t]{@{}l@{ }l}
%   \highlight{NP}<2> & \highlight{$\to$ er}<2>\\
%   \highlight{Det}<3>  & \highlight{$\to$ das}<3>\\
%   \highlight{Det}<6>  & \highlight{$\to$ dem}<6>\\
%   \end{tabular}\hspace{8mm}
%   \begin{tabular}[t]{@{}l@{ }l}
%   \highlight{N}<4> & \highlight{$\to$ Buch}<4>\\
%   \highlight{N}<7> & \highlight{$\to$ Mann}<7>\\
%   \highlight{V}<9> & \highlight{$\to$ gibt}<9>\\
%   \end{tabular}
%   \vfill
% 
%   \begin{tabular}{@{}llllll@{\hspace{2.5cm}}l}
%   er            & das          & Buch          & dem          & Mann & gibt                \onslide<+->\\
%   \highlight{NP}<2> & das          & Buch          & dem          & Mann & gibt & \only<handout>{NP $\to$ er}  \onslide<+->\\
%   NP            & \highlight{Det}<3> & Buch          & dem          & Mann & gibt & \only<handout>{Det $\to$ das}  \onslide<+->\\
%   NP            & Det            & \highlight{N}<4>  & dem          & Mann & gibt & \only<handout>{N $\to$ Buch} \onslide<+->\\
%   NP            &              & \highlight{NP}<5> & dem          & Mann & gibt & \only<handout>{NP $\to$ Det N}\onslide<+->\\
%   NP            &              & NP            & \highlight{Det}<6> & Mann & gibt & \only<handout>{Det $\to$ dem}  \onslide<+->\\
%   NP            &              & NP            & Det            & \highlight{N}<7>    & gibt & \only<handout>{N $\to$ Mann} \onslide<+->\\
%   NP            &              & NP            &              & \highlight{NP}<8>       & gibt & \only<handout>{NP $\to$ Det N}\onslide<+->\\
%   NP            &              & NP            &              & NP       & \highlight{V}<9>   & \only<handout>{V $\to$ gibt}  \onslide<+->\\
%                 &              &               &              &      & \highlight{S}<10>   & \only<handout>{S $\to$ NP NP NP V}\\
%   \end{tabular}
% 
%   \vfill
% \end{frame}
% 
% \begin{frame}
%   {Von der Grammatik beschriebene Sätze}
%   \begin{itemize}
%   \item die Grammatik ist zu ungenau:\\
%   \begin{tabular}{@{}l@{ }l}
%   NP & $\to$ Det N\\
%   S  & $\to$ NP NP NP V\\
%   \end{tabular}
%   \begin{exe}
%     \ex[]{
%   er das Buch dem Mann gibt
%   \end{frame}
%     \ex[*]{
%   ich das Buch dem Mann gibt\\
%   \onslide<+->
%   (Subjekt"=Verb"=Kongruenz {\em ich\/}, {\em gibt\/})}
%   \onslide<+->
%     \ex[*]{
%   er das Buch den Mann gibt\\\onslide<+->
%   (Kasusanforderungen des Verbs {\em gibt\/} verlangt Dativ)
%   \end{frame}
%   \onslide<+->
%     \ex[*]{
%   er den Buch dem Mann gibt\\\onslide<+->
%   (Determinator"=Nomen"=Kongruenz {\em den\/}, {\em Buch\/})
%   \end{frame}
%   \end{exe}
%   \end{itemize}
% \end{frame}
% 
% 
% \begin{frame}
%   {Subjekt"=Verb"=Kongruenz (I)}
%   \begin{itemize}
%   \item Übereinstimmung in Person (1, 2, 3) und Numerus (sg, pl)
%   \begin{exe}
%     \ex Ich schlafe. (1, sg)
%     \ex Du schläfst.  (2, sg)
%     \ex Er schläft.  (3, sg)
%     \ex Wir schlafen. (1, pl)
%     \ex Ihr schlaft.  (2, pl)
%     \ex Sie schlafen. (3,pl)
%   \end{exe}
%   \item Wie drückt man das in Regeln aus?
%   \end{itemize}
% \end{frame}
% 
% 
% \begin{frame}
%   {Subjekt"=Verb"=Kongruenz (II)}
%   \begin{itemize}
%   \item Verfeinerung der verwedenten Symbole\\
%               aus S $\to$ NP NP NP V wird\\[2ex]
%   \begin{tabular}{@{}l@{ }l}
%   S  & $\to$ NP\_1\_sg NP NP V\_1\_sg\\
%   S  & $\to$ NP\_2\_sg NP NP V\_2\_sg\\
%   S  & $\to$ NP\_3\_sg NP NP V\_3\_sg\\
%   S  & $\to$ NP\_1\_pl NP NP V\_1\_pl\\
%   S  & $\to$ NP\_2\_pl NP NP V\_2\_pl\\
%   S  & $\to$ NP\_3\_pl NP NP V\_3\_pl\\
%   \end{tabular}
% 
%   \item sechs Symbole für Nominalphrasen, sechs für Verben
%   \item sechs Regeln statt einer
%   \end{itemize}
% \end{frame}
% 
% 
% \begin{frame}
%   {Kasuszuweisung durch das Verb}
%   \begin{itemize}
%   \item Kasus muß repräsentiert sein:
%   \begin{tabular}{@{}l@{ }l}
%   S  & $\to$ NP\_1\_sg\_nom NP\_dat NP\_acc V\_1\_sg\_ditransitiv\\
%   S  & $\to$ NP\_2\_sg\_nom NP\_dat NP\_acc V\_2\_sg\_ditransitiv\\
%   S  & $\to$ NP\_3\_sg\_nom NP\_dat NP\_acc V\_3\_sg\_ditransitiv\\
%   S  & $\to$ NP\_1\_pl\_nom NP\_dat NP\_acc V\_1\_pl\_ditransitiv\\
%   S  & $\to$ NP\_2\_pl\_nom NP\_dat NP\_acc V\_2\_pl\_ditransitiv\\
%   S  & $\to$ NP\_3\_pl\_nom NP\_dat NP\_acc V\_3\_pl\_ditransitiv\\
%   \end{tabular}
%   \item insgesamt 3 * 2 * 4 = 24 neue Kategorien für NP
%   \item 3 * 2 * x  Kategorien für V (x = Anzahl der Valenzmuster)
%   \end{itemize}
% \end{frame}
% 
% 
% \begin{frame}
%   {Determinator"=Nomen"=Kongruenz}
%   \begin{itemize}
%   \item Übereinstimmung in Genus (fem, mas, neu), Numerus (sg, pl) und\\
%         Kasus (nom, gen, dat, acc)
%   \begin{exe}
%     \ex der Mann, die Frau, das Buch (Genus)
%     \ex das Buch, die Bücher (Numerus)
%     \ex des Buches, dem Buch (Kasus)
%   \end{exe}
%   \onslide<+->
%   \item aus NP $\to$ Det N wird\\[2ex]
%   \resizebox{\linewidth}{!}{
%   \begin{tabular}{@{}l@{ }l@{\hspace{4mm}}l@{ }l}
%   NP\_3\_sg\_nom  & $\to$ Det\_fem\_sg\_nom N\_fem\_sg\_nom & NP\_gen  & $\to$ Det\_fem\_sg\_gen N\_fem\_sg\_gen\\
%   NP\_3\_sg\_nom  & $\to$ Det\_mas\_sg\_nom N\_mas\_sg\_nom & NP\_gen  & $\to$ Det\_mas\_sg\_gen N\_mas\_sg\_gen\\
%   NP\_3\_sg\_nom  & $\to$ Det\_neu\_sg\_nom N\_neu\_sg\_nom & NP\_gen  & $\to$ Det\_neu\_sg\_gen N\_neu\_sg\_gen\\
%   NP\_3\_pl\_nom  & $\to$ Det\_fem\_pl\_nom N\_fem\_pl\_nom & NP\_gen  & $\to$ Det\_fem\_pl\_gen N\_fem\_pl\_gen\\
%   NP\_3\_pl\_nom  & $\to$ Det\_mas\_pl\_nom N\_mas\_pl\_nom & NP\_gen  & $\to$ Det\_mas\_pl\_gen N\_mas\_pl\_gen\\
%   NP\_3\_pl\_nom  & $\to$ Det\_neu\_pl\_nom N\_neu\_pl\_nom & NP\_gen  & $\to$ Det\_neu\_pl\_gen N\_neu\_pl\_gen\\[2mm]
%   \ldots & \hspaceThis{$\to$} Dativ                                                             & \ldots & \hspaceThis{$\to$} Akkusativ\\[2mm]
%   \end{tabular}
%   }
%   \item 24 Symbole für Determinatoren, 24 Symbole für Nomen
%   \item 24 Regeln statt einer
%   \end{itemize}
% \end{frame}
% 
% \begin{frame}
%   {Probleme dieses Ansatzes}
%   \begin{itemize}
%   \item Gernalisierungen werden nicht erfaßt.
%   \item weder in Regeln noch in Kategoriesymbolen
%         \begin{itemize}
%         \item Wo kann eine NP oder NP\_nom stehen?\\
%               Nicht wo kann eine NP\_3\_sg\_nom stehen?
%         \item Gemeinsamkeiten der Regeln sind nicht offensichtlich.
%         \end{itemize}
%   \onslide<+->
%   \item Lösung: Merkmale mit Werten und Identität von Werten\\
%         Kategoriesymbol: NP Merkmal: Per, Num, Kas, \ldots\\
% 
%   Wir erhalten \zb die Regeln:\\
% 
%   \begin{tabular}{@{}l@{ }l}
%   NP(3,sg,nom)  & $\to$ Det(fem,sg,nom) N(fem,sg,nom)\\
%   NP(3,sg,nom)  & $\to$ Det(mas,sg,nom) N(mas,sg,nom)\\
%   \end{tabular}
%   \end{itemize}
% \end{frame}
% 
% 
% \begin{frame}
%   {Merkmale und Regelschemata (I)}
%   \begin{itemize}
%   \item Regeln mit speziellen Werten zu Regelschemata verallgemeinern:
%   \medskip
%   \begin{tabular}{@{}l@{ }l@{ }l}
%   NP(\blau<3>{3},\blau<2>{Num},\blau<2>{Kas}) & $\to$ & Det(\gruen<2>{Gen},\blau<2>{Num},\blau<2>{Kas}) N(\gruen<2>{Gen},\blau<2>{Num},\blau<2>{Kas})\\
%   \end{tabular}
%   \onslide<+->
%   \item Gen-, Num- und Kas-Werte sind egal,\\
%         Hauptsache sie stimmen überein (identische Werte)
%   \onslide<+->
%   \item Der Wert des Personenmerkmals (erste Stelle in NP(3,Num,Kas))\\
%    ist durch die Regel festgelegt: 3.
%   \end{itemize}
% \end{frame}
% 
% 
% \begin{frame}
%   {Merkmale und Regelschemata (II)}
%   \begin{itemize}
%   \item Regeln mit speziellen Werten zu Regelschemata verallgemeinern:
%   \medskip
%   \begin{tabular}{@{}l@{ }l@{ }l}
%   NP({3},{Num},{Kas}) & $\to$ & Det(Gen,{Num},{Kas}) N(Gen,{Num},{Kas})\\
%   S  & $\to$ & NP(\blau<1>{Per1},\blau<1>{Num1},\blau<3>{nom})\\
%      &       & NP(Per2,Num2,\blau<3>{dat})\\
%      &       & NP(Per3,Num3,\blau<3>{akk})\\
%      &       & V(\blau<1>{Per1},\blau<1>{Num1})\\\\
%   \end{tabular}
%   \item Per1 und Num1 sind beim Verb und Subjekt gleich.
%   \onslide<+->
%   \item Bei anderen NPen sind die Werte egal.\\
%         (Schreibweise für irrelevante Werte: `\_')
%   \onslide<+->
%   \item Die Kasus der NPen sind in der zweiten Regel festgelegt.
%   \end{itemize}
% \end{frame}
% 
% 
% \begin{frame}
%   {\xbar-Theorie}
% 
% \xbar-Theorie wird in vielen verschiedenen Frameworks angenommen:\\
% \begin{itemize}
% \item Government \& Binding (GB): \citew*{Chomsky81a}
% \item Lexical Functional Grammar (LFG): \citew{Bresnan82a-ed,Bresnan2001a}
% \item Generalized Phrase Structure Grammar (GPSG):\\
%       \citew*{GKPS85a}
% \end{itemize}
% 
% \end{frame}

