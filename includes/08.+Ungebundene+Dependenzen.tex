\section{Einleitung}

\begin{frame}
  {Vorfeldbestzung usw.}
  \onslide<+->
  \onslide<+->
  Es gibt zwei Arten von Bewegung im Deutschen (und anderen Sprachen).\\
  \Zeile
  \begin{itemize}[<+->]
    \item Was bedeutet lokale und nicht-lokale Bewegung?
    \item Warum führen wir \textsc{synsem} in die Merkmalgeometrie ein?
      \Halbzeile
    \item Wie modelliert man Vorfeldbesetzung in HPSG?
    \item Warum kann man nicht wie bei \textsc{dsl} ein Kopfmerkmal nehmen?
    \item Wie funktionieren alternative Ansätze ohne Spuren?
  \end{itemize}
  \Zeile
  \onslide<+->
  \centering 
  \grau{\citet[Abschnitt~10.1--10.2, 12.1]{MuellerLehrbuch3}}\\
\end{frame}

\begin{frame}
  {Was macht Abhängigkeiten nicht-lokal?}
  \onslide<+->
  \onslide<+->
  Bei \textsc{dsl}-\alert{Bewegung} wird der \alert{Kopf} an seine \alert{Phrasengrenze} bewegt.\\
  \onslide<+->
  \Viertelzeile
  \begin{exe}
    \ex[ ]{\gruen{Hustet\Sub{1}} [Matthias \gruen{t\Sub{1}}]?}
    \ex[ ]{\gruen{Gibt\Sub{1}} [Doro Matthias den Wagen in einem Stück zurück \gruen{t\Sub{1}}]?}
    \ex[ ]{\gruen{Glaubt\Sub{1}} [Doro \gruen{t\Sub{1}}, [dass Matthias gut Auto fährt]]?}
    \ex[*]{\gruen{Fährt\Sub{1}} [Doro weiß, [dass Matthias gut Auto \gruen{t\Sub{1}}]]?}
  \end{exe}
  \onslide<+->
  \Halbzeile
  Andere Bewegungen gehen (potenziell) über Phrasen- und Clause-Grenzen hinweg.\\
  \onslide<+->
  \Viertelzeile
  \begin{exe}
    \ex[ ]{[Doro hat Matthias [das Buch \orongsch{t\Sub{1}}] gegeben], \orongsch{[das er suchte]\Sub{1}}.}
    \ex[ ]{[Matthias hat \orongsch{t\Sub{1}} gedacht], \orongsch{[die Hupe zu hören]\Sub{1}}.}
    \ex[ ]{Matthias hat [das Buch [des Linguisten [aus der Stadt \orongsch{t\Sub{1}}]]]] gelesen,\\
    \orongsch{[die keine Autobahnanbindung hat]\Sub{1}}.}
    \ex[ ]{\orongsch{Wen\Sub{1}} [hat Otje behauptet, [dass Carlos \orongsch{t\Sub{1}} gesehen hat]]?}
  \end{exe}
\end{frame}

\begin{frame}
  {Warum reicht \textsc{dsl} hier nicht?}
  \onslide<+->
  \onslide<+->
  \textsc{dsl} ist ein \alert{Kopfmerkmal}!\\
  \Zeile
  \begin{itemize}[<+->]
    \item Auf \textsc{head|dsl} ist das extrahierte Element registriert (als \textsc{loc}-Wert).
    \item Am Phrasenknoten sind zuletzt die Kopfmerkmale des Kopfs repräsentiert.
    \item In größeren Strukturen ist ein anderes Wort der Kopf, und \ldots
    \item \ldots\ das \textsc{dsl}-Merkmal des eingebetteten Kopfs ist nicht mehr zugänglich.
    \item \rot{Mit \textsc{dsl} kann man nur Kopf-an-Phrase-Bewegung modellieren!}
      \Halbzeile
    \item Mit \textsc{nonlocal} (z.\,B.\ \textsc{nonloc|slash}) führen wir neue Listen ein.
    \item Diese registrieren \alert{nicht-phrasengebundene extrahierte Elemente}: \alert{Gaps}.
    \item Ihr Inhalt wird von Köpfen und Nicht-Köpfen konkateniert und weitergegeben.
    \item Irgendwo muss ein passender \alert{Filler} (= bewegtes Element) für jede Gap stehen.
    \item Dafür gibt es einen neuen Phrasentyp: \alert{\textit{filler-gap-phrase}}.
  \end{itemize}
\end{frame}

% S. 165:
% Die englische Bezeichnung für Fernabhängigkeiten, die mehrere Satzgrenzen überschreiten können, ist unbounded dependencies. Im Unter- schied dazu gibt es Fernabhängigkeiten, die begrenzt sind, aber dennoch nicht lokal. Diese werden auch long distance dependencies genannt. Ein Beispiel für solch eine Fernabhän- gigkeit ist die Extraposition. In (4a) ist ein Relativsatz extraponiert worden, der Frau im Mittelfeld modifiziert. In (4b) ist ein Infinitivkomplement ins Nachfeld gestellt worden.
% (4) a. Der Mann hat [der Frau _i] den Apfel gegeben, [die er am schönsten fand]i. b. DerMannhat_ibehauptet,[einerFraudenApfelgegebenzuhaben]i.


\section{Extraktion mit Spur}

\begin{frame}
  {Neue Merkmalgeometrie für Zeichen}
  \onslide<+->
  \onslide<+->
  Wir haben letzte Woche bereits \textsc{nonloc} und \textsc{synysem} eingeführt.\\
  \onslide<+->
  \Zeile
  \centering 
  \begin{avm}
    \[
      \asort{sign}
      phon & \textit{\sl list of phoneme strings} \\
      \alert{synsem} & \[
        loc & \[
          cat & \[
            head & \textit{\sl head}\\
            subcat & \textit{\sl list of signs} \\
          \] \\
          cont & \textit{\sl cont} \\
        \]\\
        \alert{nonloc} & \alert{\[ \asort{nonloc}
          que   & \textit{\sl list of npros} \\
          rel   & \textit{\sl list of indices} \\
          slash & \textit{\sl list of local phrases} \\
        \]} \\
      \]
    \]
  \end{avm}\\
  \onslide<+->
  \Zeile
  Über \alert{\textsc{nonloc}} werden \alert{Fernabhängigkeiten} modelliert.\\
  \onslide<+->
  \Viertelzeile
  Nur die Merkmale auf \alert{\textsc{synsem}} dürfen lokal selegiert werden!
\end{frame}

\begin{frame}
  {Revidiertes Kopf-Argument-Schema}
  \onslide<+->
  \onslide<+->
  Es steht nicht das ganze Zeichen, sondern nur sein \textsc{synsem} auf der \textsc{subcat}.\\
  \onslide<+->
  \Zeile
  \centering 
  \raisebox{-1.5\baselineskip}{\textit{hd-arg-phr}$\Rightarrow$}\scalebox{1}{%
    \begin{avm}
      \[ 
      synsem|loc|cat|subcat        & \@1 $\oplus$ \@3 \\
      hd-dtr|synsem|loc|cat|subcat & \@1 $\oplus$ \<\@2\> $\oplus$ \@3 \\
      \alert{nhd-dtr|synsem}       & \@2 \\
    \]
    \end{avm}
  }
\end{frame}

\begin{frame}
  {Spur für die Vorfeldbesetzung}
  \onslide<+->
  \onslide<+->
  Egal, ob Argumente oder Adjunkte extrahiert werden \ldots\\
  \onslide<+->
  \Zeile
  \centering 
  \scalebox{1}{%
    \begin{avm}
      \[ \asort{word}
        phon & \<\> \\
        synsem & \[
          loc & \@1 \\
          nonloc|slash & \<\@1\> \\
        \]
      \]
    \end{avm}
  }\\
  \onslide<+->
  \Zeile
  \raggedright
  Ähnlich wie bei \textsc{dsl}:\\
  \Halbzeile
  \begin{itemize}[<+->]
    \item Phonologisch ist die \alert{Spur\slash Gap} leer.
    \item Ihr \textsc{synsem|loc}-Wert \mybox1 kommt vom \alert{Filler}.
    \item Die Gap wird auf der \alert{\textsc{synsem|nonloc|slash}-Liste} registriert.
    \item Anders als \textsc{dsl} ist \textsc{slash} \alert{nicht lokal\slash kein \textsc{head}-Merkmal}.\\
      Sonst könnte nicht über Phrasengrenzen hinaus bewegt werden!
  \end{itemize}
\end{frame}

\newcommand{\AvmHa}{%
  \scalebox{0.4}{\gruen{%
    \begin{avm}
      \[
        phon & \<\> \\
        snynsem & \[
          loc & \gruen{\@1} \[ cat|head & \@2 \[ dsl & \gruen{\@1} \]\] \\
          nonloc & \[slash & \<\> \] \\
        \] \\
      \]
    \end{avm}
  }}
}

\newcommand{\AvmHb}{%
  \scalebox{0.4}{\orongsch{%
    \begin{avm}
      \[
        phon & \<\> \\
        synsem & \rot{\@5} \[
          loc & \orongsch{\@3} \\
          nonloc|slash & \<\orongsch{\@3}\> \\
        \]
      \]
    \end{avm}
  }}
}

\newcommand{\AvmHc}{%
  \scalebox{0.4}{%
    \begin{avm}
      \[ \asort{hd-arg-phr} 
        phon & \<\> \\
        synsem & \@4 \[
          loc|cat|head & \@2  \[ dsl & \gruen{\@1} \] \\
          nonloc|slash & \<\orongsch{\@3}\> \\
        \]
      \]
    \end{avm}
  }
}

\newcommand{\AvmHd}{%
  \scalebox{0.4}{%
    \begin{avm}
      \[ \asort{v1-lex-rule}
        phon & \phon{hustet} \\
        synsem & \[ loc|cat & \[
          head & \@6 \[ \asort{verb} initial & $+$ \] \\
          subcat & \<
          \@4 \[loc|cat|head|dsl & \gruen{\@1} \]
          \> \\
        \] \\
        nonloc|slash & \<\> \\
        \] \\
        lex-dtr & \[synsem|loc & \gruen{\@1} \[
            cat & \[
              head & \[ \asort{verb} initial & $-$ \] \\
              subcat & \<\rot{\@5 NP\Up{\sl Nom}}\> \\
            \] \\
          \]\\
        \]
      \]
    \end{avm}
  }
}

\newcommand{\AvmHe}{%
  \scalebox{0.4}{%
    \begin{avm}
      \[ 
        phon & \phon{Matthias} \\
        synsem & \rot{\@5} \[
          loc & \orongsch{\@3} \\
        \] \\
      \]
    \end{avm}
  }
}

\newcommand{\AvmHf}{%
  \scalebox{0.4}{%
    \begin{avm}
      \[ \asort{hd-filler-phr}
        phon & \phon{Matthias, hustet} \\
        synsem & \[
          loc|cat & \[
            head & \@6 \[ \alert{dsl} & \alert{none} \] \\
            \alert{subcat} & \alert{\<\>} 
          \]\\
          nonloc & \[ \alert{slash} & \alert{\<\>} \] \\
        \]
      \]
    \end{avm}
  }
}

\newcommand{\AvmHg}{%
  \scalebox{0.4}{%
    \begin{avm}
      \[ \asort{hd-arg-phr}
        phon & \phon{hustet} \\
        synsem & \[
          loc|cat & \[
            head & \@6 \\
            subcat & \<\>
          \]\\
          nonloc & \[ slash & \<\orongsch{\@3}\> \] \\
        \]
      \]
    \end{avm}
  }
}

\begin{frame}
  {Verb- und Vorfeldbewegung | \textit{Matthias hustet.}}
  \onslide<+->
  \onslide<+->
  \centering
  \vspace{-1.75\baselineskip}
  \begin{forest}
    [\AvmHf
      [\AvmHe, edge label={node[midway,below,font=\tiny]{\textsc{nhd-dtr}}}]
      [\AvmHg, edge label={node[midway,below,font=\tiny]{\textsc{hd-dtr}}}
        [\AvmHd, edge label={node[midway,below,font=\tiny]{\textsc{hd-dtr}}}]
        [\AvmHc, edge label={node[midway,below,font=\tiny]{\textsc{nhd-dtr}}}
          [\AvmHb, edge label={node[midway,below,font=\tiny]{\textsc{nhd-dtr}}}]
          [\AvmHa, edge label={node[midway,below,font=\tiny]{\textsc{hd-dtr}}}]
        ]
      ]
    ]
  \end{forest}
\end{frame}

\begin{frame}
  {Filler-Gap-Konstruktionen in HPSG}
  \onslide<+->
  \onslide<+->
  \alert{Filler-Gap-Konstruktionen} modellieren unbegrenzte Dependenzen.\\
  \Zeile
  \begin{itemize}[<+->]
    \item Die Spur führt einen zu ihrem \textsc{loc} identischen \alert{\textsc{nonloc|slash}} ein \alert{(Gap)}.
    \item Alle Listen auf \textsc{nonloc} von Köpfen und Nicht-Köpfen werden weitergegeben.
    \item An eine abgeschlossene Clause-Struktur werden \alert{Filler} quasi adjungiert.
    \item Jede Kombination mit einem Filler reduziert die entsprechende \textsc{nonloc}-Liste.
    \item Über die Token-Identität mit dem \textsc{loc}-Wert der Gap pumpt der Filler \\
      alle relevanten Informationen an die Spur-Position.
      \Halbzeile
    \item \rot{Fehlt:} Mechanismus, der die \textsc{synsem|nonloc}-Listen aufsammelt.
    \item \rot{Fehlt:} Schema für die \textsc{head-filler-phrase}.
  \end{itemize}
\end{frame}

\begin{frame}
  {Aufsammeln von Informationen über Gaps}
  \onslide<+->
  \onslide<+->
  \alert{Nonlocal Feature Principle} \grau{\citep[162]{ps2}}\\
  \Zeile
  \centering 
  \onslide<+->
  \Large Der Wert jedes \textsc{nonlocal}-Merkmals einer Phrase ist die Vereinigung der entsprechenden \textsc{nonlocal}-Werte der Töchter.\\
  \onslide<+->
  \Zeile
  \footnotesize
  \grau{Eigentlich komplexer, weil das \textsc{nonlocal}-Merkmal in \citet{ps2} komplexer ist.}
\end{frame}

\begin{frame}
  {Schema für \textit{head-filler-phrase}}
  \onslide<+->
  \onslide<+->
  Filler kombinieren mit \gruen{Sätzen}, die ihre \orongsch{Gap} enthalten. Aus Fillern wird \alert{nie extrahiert}.\\
  \onslide<+->
  \Doppelzeile
  \centering 
  \raisebox{-5\baselineskip}{\textit{head-filler-phrase}$\Rightarrow$}%
  \begin{avm}
    \[ 
      synsem & \[ nonloc|slash & \<\> \] \\
      head-dtr & \[ synsem & \[
        \gruen{loc|cat} & \gruen{\[
          head & \[
            vform & fin \\
            initial & $+$ \\
          \] \\
          subcat & \<\> \\
        \]} \\
        nonloc & \[ \orongsch{slash} & \orongsch{\<\@1\>} \] \\
      \]
      \]\\
      nhd-dtr  & \[
        synsem & \[
          \orongsch{loc} & \orongsch{\@1} \\
          \alert{nonloc|slash} & \alert{\<\>} \\
        \]
      \]\\
    \]
  \end{avm}
\end{frame}

\begin{frame}
  {Typhierarchie für \textit{sign}}
  \onslide<+->
  \onslide<+->
  Das sind die Zeichentypen unserer Grammatik.\\
  \grau{\footnotesize Erinnerung | \textit{signs} modellieren tatsächliche sprachliche Zeichen.}\\
  \onslide<+->
  \centering 
  \Viertelzeile
  \scalebox{0.75}{\begin{forest}
    type hierarchy,
    for tree={
      calign=fixed angles,
      calign angle=50,
      align=center
    }
    [sign
      [word]
      [{phrase\\\alert{\footnotesize Nonlocal Feature Principle}}
        [{non-headed-phrase\\\ }]
        [{headed-phrase\\\alert{\footnotesize Head Feature Principle}}
          [{head-non-adjunct-phrase\\\alert{\footnotesize Semantics Principle A}}
            [{head-argument-phrase\\\gruen{\footnotesize Schema}\\\alert{\footnotesize Subcategorisation Principle A}\\\ }]
            [{head-filler-phrase\\\gruen{\footnotesize Schema}\\\ }]]
          [{head-non-argument-phrase\\\alert{\footnotesize Subcategorisation Principle B}\\\alert{\footnotesize Specifier Principle}}
            [,identify=!r2212]
            [{head-adjunct-phrase\\\gruen{\footnotesize Schema}\\\alert{\footnotesize Semantics Principle B}}]
          ]
        ]
      ]
    ]
  \end{forest}}
\end{frame}


\section{Extraktion ohne Spur}

% \begin{frame}
%   {Spuren}
%   \onslide<+->
%   \onslide<+->
%   Es gibt verschiedene Möglichkeiten, Fernabhängigkeiten zu modellieren.\\
%   \Zeile
%   \begin{itemize}[<+->]
%     \item \alert{Spuren} | Wort mit leerem \textsc{phon}-Wert
%     \item \alert{Unäre Regel} | "`Verschiebt"' Eintrag von \textsc{subcat} auf \textsc{slash}.
%     \item \alert{Lexikonregel} | Wie unäre Regel, nur im Lexikon
%     \item \alert{Unterspezifizierte} Lexikoneinträge und \alert{relationale Beschänkungen}
%   \end{itemize}
% \end{frame}

\begin{frame}
  {Alternativen | Unäre Regel}
  \onslide<+->
  \onslide<+->
  Unäre Regel, die einen Eintrag von \textsc{subcat} zu \textsc{slash} "`verschiebt"'.\\
  \Zeile
  \centering
  \onslide<+->
  \raisebox{-3\baselineskip}{\textit{\footnotesize hd-comp-slash-phr}$\Rightarrow$}%
  \scalebox{0.8}{%
  \begin{avm}
    \[
      synysem & \[
        loc|cat|subcat & \@1 \\
        nonloc|slash & \<\@4\>$\oplus$\@5 \\
      \] \\
    head-dtr & \[
      synsem & \[
        loc|cat|subcat & \@1$\oplus$\<\[
          synsem & \[
            loc & \@4 \\
            nonloc|slash & \@4 \\
          \] \\
        \]\> \\
        nonloc|slash & \@5 \\
      \] \\
    \] \\
    \]
  \end{avm}
  }
\end{frame}

\begin{frame}
  {Alternativen | Lexikonregel}
  \onslide<+->
  \onslide<+->
  Ganz ähnlich wie die unäre Regel \ldots\\
  \Zeile
  \onslide<+->
  \centering 
  \scalebox{0.7}{%
    \begin{avm}
      \[
        \asort{slash-lex-rule}
        synsem & \[
          loc|cat|subcat & \@1$\oplus$\@3 \\
          nonloc|slash & \<\@4\> \\
        \] \\
        lex-dtr & \[
          synsem & \[
            loc|cat & \[
              head|mod & none \\
              subcat & \@1$\oplus$\<\[
                synsem & \[
                  loc & \@4 \\
                  nonloc|slash & \<\@4\>
                \]
              \]\>$\oplus$\@3
            \] \\
            nonloc & \[ slash & \<\> \] \\
          \]
        \]
      \]
    \end{avm}
  }
\end{frame}

\begin{frame}
  {Alternativen | Unterspezifikation}
  \onslide<+->
  \onslide<+->
  Das \alert{2000er-System}! \grau{\citep{GSag2000a-u,BMS2001a}}\\
  \grau{\footnotesize Das Werk von Bouma, Malouf \& Sag umgangssprachlich auch "`BouMS"' \ldots}\\
  \Zeile
  \begin{itemize}[<+->]
    \item Ähnliche Idee wie bei der einfachen lexikalischen Regel
    \item Parallel zur \textsc{subcat} (\textsc{arg-st}) eine Liste \textsc{deps}, auf der auch Adjunkte stehen
    \item Ein Teil von \textsc{deps} wird geslasht (\textsc{local}=\textsc{slash}) und von \textsc{deps} entfernt
    \item Auf \textsc{nonloc|slash} dann Komplemente und Adjunkte möglich
  \end{itemize}
\end{frame}

\section{Nächste Woche}

\begin{frame}
  {Vorbereitung}
  \onslide<+->
  \onslide<+->
  \centering 
  \large
  \alert{Übernächste} Woche reden wir über Semantik, genauer Quantorenspeicher.\\
  \onslide<+->
  \Zeile
  \rot{Sie sollten dringend vorher aus \citet{ps2}\\
  die Seiten 47--59 lesen (s.~Webseite)!}\\
  \onslide<+->
  \Viertelzeile
  Das sind \gruen{13} Seiten.\\
\end{frame}
