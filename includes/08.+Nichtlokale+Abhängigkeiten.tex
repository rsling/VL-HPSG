\section{Einleitung}

\begin{frame}
  {Vorfeldbestzung usw.}
  \onslide<+->
  \onslide<+->
  Es gibt zwei Arten von Bewegung im Deutschen (und anderen Sprachen).\\
  \Zeile
  \begin{itemize}[<+->]
    \item x
  \end{itemize}
  \Zeile
  \onslide<+->
  \centering 
  \grau{\citet[Abschnitt~10.1--10.2]{MuellerLehrbuch3}}\\
\end{frame}

\begin{frame}
  {Was macht Abhängigkeiten nicht-lokal?}
\end{frame}

\section{Extraktion mit Spur}

\section{Extraktion ohne Spur}

\section{Nächste Woche}

\begin{frame}
  {Vorbereitung}
  \onslide<+->
  \onslide<+->
  \centering 
  \large
  \alert{Übernächste} Woche reden wir über Semantik, genauer Quantorenspeicher.\\
  \onslide<+->
  \Zeile
  \rot{Sie sollten dringend vorher aus \citet{ps2}\\
  die Seiten 47--59 lesen (s.~Webseite)!}\\
  \onslide<+->
  \Viertelzeile
  Das sind \gruen{13} Seiten.\\
\end{frame}
