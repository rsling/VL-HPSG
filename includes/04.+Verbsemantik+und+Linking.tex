\section{Einleitung}

\begin{frame}
  {Valenz und Rollensemantik}
  \onslide<+->
  \onslide<+->
  Erster Entwurf einer Semantik für HPSG:\\
  \Zeile
  \begin{itemize}[<+->]
    \item x
  \end{itemize}
  \Zeile
  \onslide<+->
  \centering 
  \grau{\citet[Kapitel~5]{MuellerLehrbuch3}}
\end{frame}

\section{Semantische Rollen und Situationen}

\section{NP-Semantik}

\section{Verbsemantik}

\section{Kopfmerkmalprinzip}

\section{Nächste Woche}

\begin{frame}
  {Vorbereitung}
  \onslide<+->
  \onslide<+->
  \centering 
  \large
  Nächste Woche reden wir über Adjunkte und Spezifizierer.\\
  \onslide<+->
  \Zeile
  \rot{Sie sollten dringend vorher aus dem HPSG-Buch\\
  von Kapitel 6 die Seiten 73--84 lesen!}\\
  \onslide<+->
  \Viertelzeile
  Das sind gerade mal \gruen{11} Seiten.
\end{frame}
