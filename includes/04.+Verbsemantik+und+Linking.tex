\section{Einleitung}

\begin{frame}
  {Valenz und Rollensemantik}
  \onslide<+->
  \onslide<+->
  Erster Entwurf einer Semantik für HPSG:\\
  \Zeile
  \begin{itemize}[<+->]
    \item Was ist Valenz?
    \item Valenz und \alert{semantische Rollen}
    \item Auf Rollen basierende Semantik: \alert{Situationssemantik}
    \item Anpassung der Merkmalsgeometrie
    \item \alert{Semantikprinzip} für Phrasen mit Kopf
  \end{itemize}
  \Zeile
  \onslide<+->
  \centering 
  \grau{\citet[Kapitel~5]{MuellerLehrbuch3}\\
  \Halbzeile
  Einführung Valenz und Rollen auch: \citet{Schaefer2018a}\\
  \Halbzeile
  \footnotesize Situationssemantik: \citet{ps, GSag2000a-u}\\
    \citet{BP83a,CMP90,Devlin92}}
\end{frame}

\section{Valenz, semantische Rollen und Situationen}

\begin{frame}
  {Ergänzungen und Angaben}
  \onslide<+->
  \onslide<+->
  \begin{exe}
    \ex\label{ex:valenz034}
    \begin{xlist}
      \ex{Gabriele malt \alert{[ein Bild]}.}
      \onslide<+->
      \ex{Gabriele malt \orongsch{[gerne]}.}
      \onslide<+->
      \ex{Gabriele malt \orongsch{[den ganzen Tag]}.}
      \onslide<+->
      \ex{Gabriele malt \orongsch{[ihrem Mann]} \rot{[zu figürlich]}.}
    \end{xlist}
  \end{exe}
  \Halbzeile
  \begin{itemize}[<+->]
    \item \alert{[ein Bild]} mit besonderer Relation zum Verb | \alert{Objekt\slash Ergänzung}
    \item keine solche Relation bei den anderen | \orongsch{Adverbial\slash Angaben}
    \item "`Weglassbarkeit"' (Optionalität) nicht entscheidend
  \end{itemize}
\end{frame}

\begin{frame}
  {Lizenzierung}
  \pause
  \begin{exe}
    \ex 
    \begin{xlist}
      \ex[ ]{Gabriele isst \orongsch{[den ganzen Tag]} Walnüsse.}
    \pause
      \ex[ ]{Gabriele läuft \orongsch{[den ganzen Tag]}.}
      \pause
      \ex[ ]{Gabriele backt ihrer Schwester \orongsch{[den ganzen Tag]} Stollen.}
      \pause
      \ex[ ]{Gabriele litt \orongsch{[den ganzen Tag]} unter Sonnenbrand.}
    \end{xlist}
    \pause\Halbzeile
    \ex 
    \begin{xlist}
      \ex[*]{Gabriele isst \alert{[ein Bild]} Walnüsse.}
      \pause
      \ex[*]{Gabriele läuft \alert{[ein Bild]}.}
      \pause
      \ex[*]{Gabriele backt ihrer Schwester \alert{[ein Bild]} Stollen.}
      \pause
      \ex[*]{Gabriele litt \alert{[ein Bild]} unter Sonnenbrand. }
      \pause
    \end{xlist}
  \end{exe}
  \pause\Halbzeile
  \begin{itemize}[<+->]
    \item \orongsch{Angaben} sind verb-unspezifisch lizenziert
    \item \alert{Ergänzungen} sind verb(klassen)spezifisch lizenziert
    \item \gruen{Valenz = Liste der Ergänzungen eines lexikalischen Worts}
  \end{itemize}
\end{frame}


\begin{frame}
  {Weitere Eigenschaften von Ergänzungen und Angaben}
  \pause
  \alert{Iterierbarkeit} (= Wiederholbarkeit) von Angaben, nicht Ergänzungen\\
  \pause
  \Halbzeile
  \begin{exe}
    \ex[ ]{Wir müssen den Wagen \orongsch{[jetzt]} \orongsch{[mit aller Kraft]} \orongsch{[vorsichtig]} anschieben.}
    \pause
    \ex[ ]{Wir essen \orongsch{[schnell]} \orongsch{[mit Appetit]} \orongsch{[an einem Tisch]}\\
      \orongsch{[mit der Gabel]} \alert{[einen Salat]}.}
    \pause
    \ex[*]{Wir essen \orongsch{[schnell]} \rot{[ein Tofugericht]} \orongsch{[mit Appetit]} \orongsch{[an einem Tisch]}\\
      \orongsch{[mit der Gabel]} \alert{[einen Salat]}.}
  \end{exe}
\end{frame}


\begin{frame}
  {Ergänzungen | Schnittstelle von Syntax und Semantik}
  \onslide<+->
  \onslide<+->
  Verbsemantik | Welche \alert{Rolle} spielen die von den Satzgliedern bezeichneten Dinge\\
  in der vom Verb beschriebenen Situation?\\
  \Zeile
  \onslide<+->
  Semantik (\alert{Rolle}) von \alert{Ergänzungen} | \alert{abhängig} vom Verb\\
  \onslide<+->
  \Viertelzeile
  Semantik (\alert{Rolle}) von \gruen{Angaben} | \gruen{unabhängig} vom Verb\\
  \Halbzeile
  \pause
  \begin{exe}
    \ex\label{ex:valenz071}
    \begin{xlist}
      \ex{\label{ex:valenz072}Ich lösche \alert{[den Ordner]} \gruen{[während der Hausdurchsuchung]}.}
      \pause
      \ex{\label{ex:valenz073}Ich mähe \alert{[den Rasen]} \gruen{[während der Ferien]}.}
      \pause
      \ex{\label{ex:valenz074}Ich fürchte \alert{[den Sturm]} \gruen{[während des Sommers]}.}
    \end{xlist}
  \end{exe}
\end{frame}

\begin{frame}
  {Valenz}
  \onslide<+->
  \onslide<+->
  \begin{block}{Angaben}
    \alert{Angaben} sind grammatisch immer lizenziert und bringen\\
    ihre eigene semantische Rolle mit.
  \end{block}
  \Zeile
  \onslide<+->
  \begin{block}{Ergänzungen}
    \gruen{Ergänzungen} werden spezifisch vom Verb lizenziert und in ihrer semantischen Rolle\\
    vom Verb festgelegt. Jede dieser Rollen kann nur einmal vergeben werden.
  \end{block}
\end{frame}

\section{Situationssemantik und Linking}

\begin{frame}
  {Situationen in Situationssemantik}
  \onslide<+->
  \onslide<+->
  Uns interessieren \alert{Situationen wie sie vom Verb beschrieben werden}.\\
  \Zeile
  \begin{itemize}[<+->]
    \item \textit{sehen} beschreibt \alert{sehen-Situationen} mit \gruen{zwei Mitspielern}
    \item \textit{schlafen} beschreibt \alert{schlafen-Situationen} mit \gruen{einem Mitspieler}
    \item \textit{schenken} beschreibt \alert{schenken-Situationen} mit \gruen{drei Mitspielern}
      \Halbzeile
    \item Unabhängig vom verbkodierten Situationstyp (=~Angabenmaterial):
      \begin{itemize}[<+->]
        \item Ort (\textit{auf dem Bett})
        \item Zeit (\textit{am letzten Dienstag})
        \item Geschwindigkeit (\textit{schnell})
        \item Zustand der Beteiligten (\textit{total groggy}, \textit{dicht})
        \item usw.
      \end{itemize}
  \end{itemize}
\end{frame}

\begin{frame}
  {PSOA | Parametrised State of Affairs}
  \onslide<+->
  \onslide<+->
  Die Verbsemantik muss angeben, welche Objekte\slash Mitspieler an Situationen\\
  beteiligt sind, und was über sie gesagt wird. Die Beschreibung erfolgt als PSOAs.\\
  \onslide<+->
  \Zeile
  \alert{Ein Kollege liest ein Buch.}\\
  \Halbzeile
  \begin{itemize}[<+->]
    \item Situationstyp: \alert{lesen} (V-Beitrag)
    \item Beteiligt: \alert{Objekt $x$ mit Eigenschaft \textit{Kollege}} (NP-Beitrag)
    \item Beteiligt: \alert{Objekt $y$ mit Eigenschaft \textit{Buch}} (NP-Beitrag)
    \item Rolle: \alert{Agens: $x$} (V-Beitrag\slash Linking)
    \item Rolle: \alert{Patiens: $y$} (V-Beitrag\slash Linking)
      \Halbzeile
    \item \grau{\footnotesize\textit{Objekt} ist hier im weitesten Sinn zu verstehen: Alles, über das man individualisiert sprechen kann.}
    \item \grau{\footnotesize In Konstruktionsgrammatik sind die Rollen der Beitrag einer Konstruktion. Aha.}
  \end{itemize}
\end{frame}

\begin{frame}
  {Semantischer Beitrag einer NP}
  \onslide<+->
  \onslide<+->
  \alert{Nomina bzw.\ NPs (\textit{ein}) \textit{Kollege} und \textit{ein Buch}}\\
  Ein Objekt wird in den Diskurs eingeführt (\textsc{index}) und als Kollege deklariert (\textsc{restr}).\\
  \Zeile 
  \onslide<+->
  \scalebox{0.8}{\begin{avm}
    \[ phon & \phon{ein, Kollege} \\
      head & \[ \asort{noun} \] \\
      \alert{content} & \alert{\[ index & \@1 \\
        restrictions & \< \[ \asort{colleague-rel} instance & \@1 \]\>
      \]}
    \]
  \end{avm}}\onslide<+->\hspace{2em}%
  \scalebox{0.8}{\begin{avm}
    \[ phon & \phon{ein, Buch} \\
      head & \[ \asort{noun} \] \\
      \alert{content} & \alert{\[ index & \@1 \\
        restrictions & \< \[ \asort{book-rel} instance & \@1 \]\>
      \]}
    \]
  \end{avm}}\\
  \Halbzeile
  \onslide<+->
  \footnotesize Achtung! Die beiden Strukturteilungen \mybox{1} bei \textit{ein Kollege} und \textit{ein Buch}\\
  stehen in unabhängigen Merkmalstrukturen und sind daher voneinander verschieden.
\end{frame}

\begin{frame}
  {Semantischer Beitrag des Verbs}
  \onslide<+->
  \onslide<+->
  \alert{Linking} | Verknüpfung von grammatischer Valenz und Verbsemantik\\
  \onslide<+->
  \Zeile
  \begin{tabular}{cc}
    \begin{minipage}{0.5\textwidth}
      \scalebox{0.7}{\begin{avm}
        \[ phon & \phon{liest} \\
          head & \[ \asort{verb} \] \\
          subcat & \<
            \[ head & \[ \asort{noun} cas & nom \] \\
               subcat & \<\> \\
               \gruen{cont} & \gruen{\[ ind & \@1 \]}
            \], 
            \[ head & \[ \asort{noun} cas & acc \] \\
               subcat & \<\> \\
               \gruen{cont} & \gruen{\[ ind & \@2 \]}
            \]
          \> \\
          \alert{cont} & \alert{\[
            restr & \< \[ \asort{read-rel} agent & \@1\\
              patient & \@2 \]\> 
        \]}
        \]
      \end{avm}}
    \end{minipage} & %
    \begin{minipage}{0.4\textwidth}
      \begin{itemize}[<+->]\footnotesize
        \item Es stehen zwei valenzgebundene NPs auf der \textsc{subcat}.
        \item Diese bringen \gruen{je einen Index mit (\mybox{1} und \mybox{2})}, auf die das Verb über die \textsc{subcat} "`zugreift"'.
        \item Diese Indizes werden durch den Beitrag der NPs als Kollegen, Bücher usw. spezifiziert (hier nicht zu erkennen).
        \item Das Verb fügt die \alert{Information hinzu, dass sie in einer lesen-Relation stehen (\mybox{1} liest \mybox{2})}.
      \end{itemize}
    \end{minipage} \\
  \end{tabular}
\end{frame}

\begin{frame}
  {Valenztypen \slash\ Verbtypen}
  \onslide<+->
  \onslide<+->
  Denkbare \alert{Hierarchie für Verb-Relationen} | Ziel: Generalisierungen abbilden!\\
  \centering
  \Zeile
  \scalebox{0.7}{\begin{forest}
    typehierarchy
    [rel
      [agent-rel
        [run-rel]
        [yell-rel]
        [\ldots]
      ]
      [agent-patient-rel
        [behold-rel]
        [read-rel]
        [\ldots]
      ]
      [patient-rel
        [sleep-rel]
        [cough-rel]
        [\ldots]
      ]
      [agent-theme-goal-rel
        [give-rel]
        [send-rel]
        [\ldots]
      ]
      [\ldots]
    ]
  \end{forest}}
\end{frame}


\section{Modifikation der Merkmalgeometrie}

\begin{frame}
  {Finale Merkmalgeometrie}
  \onslide<+->
  \onslide<+->
  Weil einige es lieber "`gleich in richtig"' hätten, hier einmal die Geometrie,\\
  auf die es hinausläuft für \alert{Phonologie (\textsc{phon})}, \gruen{Syntax (\textsc{cat})} und \orongsch{Semantik (\textsc{cont})}\\
  \onslide<+->
  \Zeile
  \centering 
  \scalebox{0.7}{%
    \begin{avm}
      \[ \asort{word} 
        \alert{phon} & list \\
        \rot{synsem} & \[
          \rot{local} & \[ 
            \gruen{cat} & \[ head & head \\
              subcat & list \] \\
              \orongsch{cont} & cont
          \] \\
          \rot{nonlocal} & \[ slash & list \]
        \]
      \]
    \end{avm}
  }\onslide<+->\hspace{2em}\scalebox{0.7}{%
    \begin{avm}
      \[ \asort{headed-phrase} 
        \alert{phon} & list \\
        \rot{synsem} & \[
          \rot{local} & \[ 
            \gruen{cat} & \[ head & head \\
              subcat & list \] \\
              \orongsch{cont} & cont
          \] \\
          \rot{nonlocal} & \[ slash & list \] \\
        \] \\
        hd-dtr & sign \\
        nhd-dtr & sign \\
      \]
    \end{avm}
  }\\
  \Zeile
  \onslide<+->
  \rot{Die Knoten \textsc{synsem}, \textsc{local} und \textsc{nonlocal} jetzt schon einzuführen,\\
  wäre nicht zielführend. Wir brauchen sie erst für einen Typ von Bewegung.}\\
  \grau{\footnotesize Außerdem gibt es bei Bedarf auch Geometrien mit noch mehr Struktur.}
\end{frame}

\begin{frame}
  {Syntax und Semantik trennen}
  \onslide<+->
  \onslide<+->
  Trennung von Syntax und Semantik | \textsc{head} und \textsc{subcat} (= Syntax) bündeln\\
  \onslide<+->
  \Zeile
  \centering 
  \scalebox{0.7}{\begin{avm}
    \[ phon & \phon{liest} \\
      \alert{cat} & \[
        \alert{head} & \[ \asort{verb} \] \\
        \alert{subcat} & \<
          \[ head & \[ \asort{noun} cas & nom \] \\
             subcat & \<\> \\
             cont & \[ ind & \@1 \]
          \], 
          \[ head & \[ \asort{noun} cas & acc \] \\
             subcat & \<\> \\
             cont & \[ ind & \@2 \]
          \]
        \> \\
      \]\\
      cont & \[
        restr & \< \[ \asort{read-rel} agent & \@1\\
          patient & \@2 \]\> 
    \]
    \]
  \end{avm}}
\end{frame}

\begin{frame}
  {Kongruenzmerkmale im \textsc{index}}
  \onslide<+->
  \onslide<+->
  \alert{Anaphern} | Kongruieren in Person, Numerus, Genus.\\
  Da dies über Satzgrenzen hinaus geschieht, sollten es \alert{Semantik-Merkmale} sein.\\
  \onslide<+->
  \Halbzeile
  \begin{exe}
    \ex[ ]{\alert{Die Kollegin\Sub{1}} liest \gruen{das Buch\Sub{2}}. \alert{Sie\Sub{1}} findet \gruen{es\Sub{2}} Schrott.}
    \ex[*]{\alert{Die Kollegin\Sub{1}} liest \gruen{das Buch\Sub{2}}. \alert{Er\Sub{1}} findet \gruen{euch\Sub{2}} Schrott.}
  \end{exe}
  \Halbzeile
  \onslide<+->
  \centering 
  \scalebox{0.8}{\begin{avm}
    \[ phon & \phon{eine, Kollegin} \\
      cat & \[ head & \[ \asort{noun} cas & nom $\vee$ acc $\vee$ dat $\vee$ gen\] \\
               subcat & \<\> 
    \] \\
      cont & \[ ind & \alert{\@1 \[ per & 3\\
                                                  num & sg\\
                                                gen & f\]} \\
        restr & \< \[ \asort{colleague-rel} inst & \@1 \]\>
      \]
    \]
  \end{avm}}
\end{frame}

\section{Semantikprinzip}

\begin{frame}
  {Semantik von Phrasen}
  \onslide<+->
  \onslide<+->
  Semantikprinzip für Phrasen mit Kopf wie Kopf-Komplement-Phrasen:\\
  \onslide<+->
  \Zeile
  \centering 
  \alert{\raisebox{-0.5\baselineskip}{\textit{head-non-adjunct-phrase} $\Rightarrow$\ }%
  \begin{avm}
    \[ cont & \@1 \\
      head-dtr|cont & \@1
    \]
  \end{avm}}\\
  \onslide<+->
  \Zeile
  Es gibt auch andere Phrasen. Tentative Typhierarchie für \textit{sign}:\\
  \onslide<+->
  \Halbzeile
  \centering 
  \scalebox{0.7}{\begin{forest}
    typehierarchy
    [sign
      [word]
      [phrase
        [non-headed-phrase]
        [headed-phrase
          [\alert{head-non-adjunct-phrase}
            [head-argument-phrase]
          ]
          [\ldots]
        ]
      ]
    ]
  \end{forest}}
\end{frame}

\begin{frame}
  {Die gesamte Kombinatorik bisher}
  \onslide<+->
  \onslide<+->
  Kopf-Komplement-Schema\\
  \Viertelzeile
  \raggedleft
  \scalebox{0.8}{\raisebox{-1.5\baselineskip}{\textit{head-argument-phrase}\ $\Rightarrow$\ }%
  \begin{avm}
    \[ cat|subcat & \@1 \\
       hd-dtr|cat|subcat & \@1\ $\oplus$\ \<\@2\> \\
       non-hd-dtr & \@2 
     \]
   \end{avm}}\\
  \onslide<+->
  \Zeile
  \raggedright
  Kopf-Merkmalprinzip\\
  \Viertelzeile
  \raggedleft
  \scalebox{0.8}{\raisebox{-0.5\baselineskip}{\textit{headed-phrase}\ $\Rightarrow$\ }%
  \begin{avm}
    \[ cat|head & \@1 \\
       hd-dtr|cat|head & \@1 \\
     \]
   \end{avm}}\\
  \onslide<+->
  \Zeile
  \raggedright
  Semantikprinzip\\
  \Viertelzeile
  \raggedleft
  \scalebox{0.8}{\raisebox{-0.5\baselineskip}{\textit{head-non-adjunct-phrase} $\Rightarrow$\ }%
  \begin{avm}
    \[ cont & \@1 \\
      head-dtr|cont & \@1
    \]
  \end{avm}}\\
\end{frame}

\begin{frame}
  {Zusammenspiel der bisherigen Prinzipien und Schemata}
  \onslide<+->
  \onslide<+->
  \centering 
  \begin{tabular}[h]{cc}
    \begin{minipage}{0.28\textwidth}
      \scalebox{0.37}{%
        \begin{avm}
          \[ \asort{\alert<4,6,8>{head-argument-phrase}}
            phon & \phon{ein, Ball, fliegt} \\
            cat & \[ \alert<6>{head} & \alert<6>{\@1} \\
              \alert<4>{subcat} & \alert<4>{\<\>}
            \]\\
            \alert<8>{cont} & \alert<8>{\@2} \\
            \alert<4>{nhd-dtr} & \alert<4>{\@5 \[ \asort{\alert<3,5,7>{head-argument-phrase}}
             phon & \phon{ein, Ball} \\
             cat & \[ \alert<5>{head} & \alert<5,10>{\@8 \[ \asort{noun} cas & \@9 nom \]} \\
               \alert<3>{subcat} & \alert<3>{\<\>}
             \] \\
             \alert<7>{cont} & \alert<7>{\@6} \\
             \alert<3>{nhd-dtr} & \alert<3>{\@7 \[ \asort{word}  
               phon & \phon{ein} \\
               cat & \[ head & \[ \asort{det} \alert<10>{cas} & \alert<10>{\@9} \] \] \\
             \]} \\
             hd-dtr & \[ \asort{word} 
               phon & \phon{Ball} \\
               cat & \[ \alert<5,10>{head} & \alert<5,10>{\@8} \\
               \alert<3>{subcat} & \alert<3>{\<\@7\>} \] \\
               \alert<7>{cont} & \alert<7>{\@6 \[ \alert<9>{ind} & \alert<9>{\@3 \[ per & 3 \\ num & sg \\ gen & m \]} \\
                 restr & \<\[ \asort{ball-rel} theme & \@3 \] \>
               \]}
             \]
           \]} \\
            hd-dtr & \[ \asort{word} 
             phon & \phon{fliegt} \\
             cat & \[ \alert<6>{head} & \alert<6>{\@1 \[ \asort{verb} vform & fin \]} \\ 
             \alert<4>{subcat} & \alert<4>{\< \@5 \>}
             \] \\
             \alert<8>{cont} & \alert<8>{\@2 \[ restr & \alert<9>{\<\[ \asort{fly-rel} agent & \@3 \]\>} \]} \\
            \] \\
          \]
        \end{avm}
      }
    \end{minipage} & %
    \begin{minipage}{0.68\textwidth}
      \begin{itemize}[<+->]
        \item \alert<3>{Kopf-Komplement-Schema für die NP \textit{ein Ball}}
        \item \alert<4>{Kopf-Komplement-Schema für die VP \textit{ein Ball fliegt}}
        \item \alert<5>{Kopfmerkmalprinzip} für die NP \textit{ein Ball}
        \item \alert<6>{Kopfmerkmalprinzip} für die VP \textit{ein Ball fliegt}
        \item \alert<7>{Semantikprinzip} für die NP \textit{ein Ball}
        \item \alert<8>{Semantikprinzip} für die VP \textit{ein Ball fliegt}
          \Zeile
        \item \alert<9>{Linking durch den Lexikoneintrag von \textit{fliegt}}
        \item \alert<10>{Kongruenz durch den Lexikoneintrag von \textit{Ball}}
      \end{itemize}
    \end{minipage} \\
  \end{tabular}
  \onslide<+->
\end{frame}

\section{Nächste Woche}

\begin{frame}
  {Vorbereitung}
  \onslide<+->
  \onslide<+->
  \centering 
  \large
  Nächste Woche reden wir über Adjunkte und Spezifizierer.\\
  \onslide<+->
  \Zeile
  \rot{Sie sollten dringend vorher aus dem HPSG-Buch\\
  von Kapitel 6 die Seiten 73--84 lesen!}\\
  \onslide<+->
  \Viertelzeile
  Das sind gerade mal \gruen{11} Seiten.
\end{frame}
