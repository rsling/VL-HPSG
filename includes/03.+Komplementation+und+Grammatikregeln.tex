\section{Einleitung}

\begin{frame}
  {Konzepte von letzter Woche}
  \onslide<+->
  \onslide<+->
  Wir systematisieren jetzt folgende Konzepte weiter:\\
  \Zeile
  \begin{itemize}[<+->]
    \item \alert{Merkmalbündel} gemäß Anforderungen aus den Daten (\textsc{head}-Features)
    \item \alert{Getypte Merkmalstrukturen} zur Kodierung von Generalisierungen
    \item \alert{Typenhierarchien} als Wortarten auf Steroiden
    \item \alert{Listen von Merkmalstrukturen} zur Repräsentation von \alert{Valenz}
    \item \alert{Strukturteilung} zur Modellierung von Kongruenz und Valenz
  \end{itemize}
  \Zeile
  \onslide<+->
  \centering 
  \grau{\citet[Kapitel~3.1~und~4]{MuellerLehrbuch3}}
\end{frame}

\section{Ohne Bäume, ohne Transformationen}

\begin{frame}
  {Status von Phrasenstrukturbäumen}
  \onslide<+->
  \onslide<+->
  Bäume als anschauliche Darstellung von Konstituenz\\
  \onslide<+->
  \Halbzeile
  \centering
  \begin{forest}
    [CP
      [C
        [\it dass]
      ]
      [VP
        [NP
          [\it Matthias]
        ]
        [V$'$
          [NP
            [\it Doro]
          ]
          [V
            [\it besucht]
          ]
        ]
      ]
    ]
  \end{forest}\\
  \onslide<+->
  \Zeile
  Sprache besteht aber \rot{immer nur aus Oberfläche}!\\
  \onslide<+->
  \Viertelzeile
  \grau{Natürlich kann man beliebige Behauptungen über Bäume im Gehirn hinzuerfinden.}
\end{frame}

\begin{frame}
  {Theorien ohne zusätzliche Strukturartefakte}
  \onslide<+->
  \onslide<+->
  HPSG | \alert{Struktur von Wörtern und Wortsequenzen}\\
  \onslide<+->
  \Zeile
  \centering 
  \scalebox{0.6}{\begin{tabular}[h]{cccc}
    \multicolumn{4}{c}{ %
    \visible<7->{%
      \resizebox{0.62\textwidth}{!}{
        \begin{avm}
          \[ phon & dass~Matthias~Doro~besucht \\
            head & \[ \asort{comp} \]
          \]
        \end{avm}} %
      }} \\
    \visible<6->{%
    & \multicolumn{3}{c}{ %
      \resizebox{0.455\textwidth}{!}{\begin{avm}
        \[ phon & Matthias~Doro~besucht \\
          head & \[ \asort{verb} \]
        \]
      \end{avm}} %
    } \\}
    \visible<5->{%
    & & \multicolumn{2}{c}{ %
      \resizebox{0.29\textwidth}{!}{\begin{avm}
        \[ phon & Doro~besucht \\
          head & \[ \asort{verb} \]
        \]
      \end{avm}} %
    }\\}
    \visible<4->{%
    \resizebox{0.13\textwidth}{!}{\begin{avm}
      \[ phon & dass \\
        head & \[ \asort{comp} \]
      \]
    \end{avm}} & %
    \resizebox{0.13\textwidth}{!}{\begin{avm}
      \[ phon & Matthias \\
        head & \[ \asort{noun} \]

      \]
    \end{avm}} & %
    \resizebox{0.13\textwidth}{!}{\begin{avm}
      \[ phon & Doro  \\
        head & \[ \asort{noun} \]

      \]
    \end{avm}} & %
    \resizebox{0.13\textwidth}{!}{\begin{avm}
      \[ phon & besucht  \\
        head & \[ \asort{verb} \]

      \]
    \end{avm}} \\}
    \visible<3->{\textit{dass} & \textit{Matthias} & \textit{Doro} & \textit{besucht} \\}
  \end{tabular}}\\
  \onslide<8->
  \Zeile
  Die größeren Strukturen sind die \alert{direkten Repräsentationen der Wortketten}.\\
  \onslide<9->
  \Viertelzeile
  Die Grammatik muss spezifizieren, unter welchen Bedingungen sie \alert{wohlgeformt} sind.
\end{frame}

\begin{frame}
  {Phrasen in HPSG}
  \onslide<+->
  \onslide<+->
  Strukturen mit \alert{Kopf- und Nicht-Kopf-Bündeln}\\
  \onslide<+->
  \Zeile
  \centering
  \begin{tabular}[h]{p{0.3\textwidth}p{0.05\textwidth}c}
    %
    \only<3-4|handout:0>{\scalebox{0.8}{\begin{forest}
      [ , whitetree 
        [
          [
            [ dass Matthias Doro besucht, blacktree ]
          ]
        ]
      ]
    \end{forest}}} %
    %
    \only<5-6|handout:0>{\scalebox{0.8}{\begin{forest}
      [  , whitetree
        [ , blacktree
          [dass, tier=t]
          [
            [Matthias Doro besucht, narroof, tier=t]
          ]
        ]
      ]
    \end{forest}}} %
    %
    \only<7-8|handout:0>{\scalebox{0.8}{\begin{forest}
      [
        [dass, tier=t]
        [
          [Matthias, tier=t]
          [
            [Doro besucht, narroof, tier=t]
          ]
        ]
      ]
    \end{forest}}} %
    %
    \only<9->{\scalebox{0.8}{\begin{forest}
      [
        [dass, tier=t]
        [
          [Matthias, tier=t]
          [
            [Doro, tier=t]
            [besucht, tier=t]
          ]
        ]
      ]
    \end{forest}}} %
    && %
    \scalebox{0.7}{\begin{avm}
      \[ phon & dass~Matthias~Doro~besucht \\
        \whyte<1-3|handout:0>{hd-dtr} & \whyte<1-3|handout:0>{\[ phon & dass \]} \\
        \whyte<1-4|handout:0>{non-hd-dtr} & \whyte<1-4|handout:0>{\[ phon & Matthias~Doro~besucht \\
                         \whyte<1-5|handout:0>{non-hd-dtr} & \whyte<1-5|handout:0>{\[ phon & Matthias \]} \\
                         \whyte<1-6|handout:0>{hd-dtr} & \whyte<1-6|handout:0>{\[ phon & Doro~besucht \\
                           \whyte<1-7|handout:0>{non-hd-dtr} & \whyte<1-7|handout:0>{\[ phon & Doro \]} \\
                           \whyte<1-8|handout:0>{hd-dtr} & \whyte<1-8|handout:0>{\[ phon & besucht \]} \\
                         \]} \\
                       \]}
      \]
    \end{avm}} \\
  \end{tabular} \\
  \onslide<+->
  \onslide<+->
  \onslide<+->
  \onslide<+->
  \onslide<+->
  \onslide<+->
  \onslide<+->
  \Zeile
  \grau{\footnotesize Wir tun erst einmal so, als wäre die Wortstellung bei der Verbindung der Wörter egal.}
\end{frame}

\begin{frame}
  {Bewegungstransformationen}
  \onslide<+->
  \onslide<+->
  \alert{Bewegung} | Erklärt \alert{Abhängigkeiten} zwischen Positionen in Strukturen.\\
  \grau{\footnotesize\textit{Transformationen} sagt man seit der GB-Theorie nicht mehr. Technisch gesehen sind es Transformationen.}\\
  \onslide<+->
  \Zeile
  \centering
  \begin{tabular}{cc}
    \scalebox{0.8}{\begin{forest}
      [CP
        [C$'$
          [C
            [\it dass, rottree]
          ]
          [VP
            [NP
              [\it Matthias, bluetree]
            ]
            [V$'$
              [NP
                [\it Doro]
              ]
              [V
                [\it besucht, gruentree]
              ]
            ]
          ]
        ]
      ]
    \end{forest}} & %
    \visible<4->{\scalebox{0.8}{\begin{forest}
      [CP
        [NP
            [\it Matthias\Sub{2}, bluetree, name=Matthias]
        ]
        [C$'$
          [C
            [\it besucht\Sub{1}, gruentree, name=besucht]
          ]
          [VP
            [t\Sub{2}, bluetree, name=t2]
            [V$'$
              [NP
                [\it Doro]
              ]
              [V
                [t\Sub{1}, gruentree, name=t1]
              ]
            ]
          ]
        ]
      ]
      {\draw [<->, bend left=70, gruen, thick] (t1.south) to (besucht.south);}
      {\draw [<->, bend left=70, trueblue, thick] (t2.south) to (Matthias.south);}
    \end{forest}}} \\
  \end{tabular}
\end{frame}


\begin{frame}
  {Theorien ohne Transformationen im weiteren Sinn}
  \onslide<+->
  \onslide<+->
  HPSG | \alert{Die gleichen Abhängigkeiten ohne Bewegung}, dafür mit \alert{Strukturteilung}\\
  \grau{\footnotesize Aber nicht unbedingt ohne leere Elemente!}\\
  \onslide<+->
  \Zeile
  \centering 
  \begin{tabular}[h]{cp{0.05\textwidth}c}
    \scalebox{0.6}{\begin{forest}
      [ VP
        [NP
          [\it \alt<1-4|handout:0>{Matthias}{\alert{Matthias\Sub{1}}}]
        ]
        [VP
          [V
            [\it \alt<1-4|handout:0>{besucht}{\gruen{besucht\Sub{2}}}]
          ]
          [VP
            [NP
              [\it Doro]
            ]
            [V
              [\alt<1-4|handout:0>{$\emptyset$}{\alert{t\Sub{1}}}]
              [\alt<1-4|handout:0>{$\emptyset$}{\gruen{t\Sub{2}}}]
            ]
          ]
        ]
      ]
    \end{forest}} && % 
    \onslide<+->\alt<1-4|handout:0>{%
    \scalebox{0.55}{\begin{avm}
      \[ phon & Matthias~besucht~Doro \\
        non-hd-dtr & \[ phon & Matthias \] \\
        hd-dtr & \[ phon & besucht~Doro \\
          hd-dtr & \[ phon & besucht \] \\
          non-hd-dtr & \[ phon & Doro \\
            non-hd-dtr & \[ phon & Doro \] \\
            hd-dtr & \[ phon & \<\> \\
              non-hd-dtr & \[ phon & \<\> \] \\
              hd-dtr & \[ phon & \<\> \]
            \]
          \]
        \]
      \]
    \end{avm}}}{%
    \scalebox{0.55}{\begin{avm}
      \[ phon & Matthias~besucht~Doro \\
        non-hd-dtr & \alert{\[ phon & Matthias \\ 
          some-features & \@1
        \]} \\
        hd-dtr & \[ phon & besucht~Doro \\
          hd-dtr & \gruen{\[ phon & besucht \\
          some-features & \@2 \]} \\
          non-hd-dtr & \[ phon & Doro \\
            non-hd-dtr & \[ phon & Doro \] \\
            hd-dtr & \[ phon & \<\> \\
              non-hd-dtr & \alert{\[ phon & \<\> \\
              some-features & \@1 \]} \\
              hd-dtr & \gruen{\[ phon & \<\> \\
              some-features & \@2 \]}
            \]
          \]
        \]
      \]
    \end{avm}}%
    }\\
  \end{tabular}  \\
  \Zeile
  \onslide<+->
  \onslide<+->
  \rot{Wenn die Spuren an den richtigen Positionen sind, braucht man keine Transformation!}\\
  \grau{\footnotesize Die Bewegung ins Vorfeld geht ohne Spur. Das kommt alles noch und sieht dann deutlich anders aus.}
\end{frame}

\section{Valenz}

\begin{frame}
  {Valenz als Liste von Merkmalbeschreibungen | Präpositionen}
  \onslide<+->
  \onslide<+->
  Valenz bzw.\ \gruen{\textsc{subcat(egorisation)}} einer Präposition\\
  \onslide<+->
  \Zeile
  \begin{avm}
    \[
      phon & \alert{wegen} \\
    head & \[ \asort{prep} \] \\
    \gruen{subcat} & \gruen{\< \[ 
      head & \[ \asort{noun} 
      case & gen \] \] \>}
  \]
  \end{avm}\\
  \onslide<+->
  \Zeile
  Die Präposition \textit{wegen} verbindet sich mit \alert{einem nominalen Element im Genitiv}.
\end{frame}

\begin{frame}
  {Valenz von Nomina}
  \onslide<+->
  \onslide<+->
  Zur Erinnerung | \alert{NP-Analyse} (nicht DP)\\
  \onslide<+->
  \centering 
  \Zeile
  \scalebox{0.6}{\begin{avm}
    \[ phon & \alert{Telefon} \\
      head & \[ \asort{\orongsch{count-noun}}
        gen & \@1 neut \\
        num & \@2 sg \\
        case & \@3 nom $\vee$ acc $\vee$ dat\\
      \] \\
      \gruen{subcat} & \gruen{\< %
        \[ head & \[ \asort{det} %
          gen & \@1 \\
          num & \@2 \\
          case & \@3 \\
        \] \] %
      \>}
    \]
  \end{avm}}\onslide<+->\hspace{3em}\scalebox{0.6}{\begin{avm}
    \[ phon & \alert{Saft} \\
      head & \[ \asort{\orongsch{mass-noun}}
        gen & \@1 masc \\
        num & \@2 sg \\
        case & \@3 nom $\vee$ acc $\vee$ dat\\
      \] \\
      \gruen{subcat} & \gruen{\<\>} \\
    \]
  \end{avm}}\onslide<+->\hspace{3em}\scalebox{0.6}{\begin{avm}
    \[ phon & \alert{Saft} \\
      head & \[ \asort{\orongsch{sort-noun}} 
        gen & \@1 masc \\
        num & \@2 sg \\
        case & \@3 nom $\vee$ acc $\vee$ dat\\
      \] \\
      \gruen{subcat} & \gruen{\< %
        \[ head & \[ \asort{det} %
          gen & \@1 \\
          num & \@2 \\
          case & \@3 \\
        \] \] %
      \>}
    \]
  \end{avm}}\\
  \onslide<+->
  \Zeile
  \footnotesize Idealerweise möchte man das Stoffnomen \textit{Saft} mit dem sortalen Nomen lexikalisch in Beziehung setzen.\\
  \grau{Das können sogenannte \textit{Lexikonregeln}. Kommt alles noch.}
\end{frame}

\begin{frame}
  {Valenz von Verben}
  \onslide<+->
  \onslide<+->
  Beispiele für verbale Valenz\\
  \onslide<+->
  \Zeile
  \centering 
  \scalebox{0.7}{%
    \begin{avm}
      \[ phon & \alert{schläft} \\
        head & \[ \asort{verb} 
          vform & fin \\
          per & \@1 3 \\
          num & \@2 sg \\
        \] \\
        \gruen{subcat} & \gruen{\<
          \[ head & \[ \asort{noun} 
            per & \@1 \\
            num & \@2 \\
            case & nom \\
          \] \]
        \>} \\
      \]
    \end{avm}%
  }%
  \onslide<+->\hspace{2em}%
  \scalebox{0.7}{%
    \begin{avm}
      \[ phon & \alert{testet} \\
        head & \[ \asort{verb} 
          vform & fin \\
          per & \@1 3 \\
          num & \@2 sg \\
        \] \\
        \gruen{subcat} & \gruen{\<
          \[ head & \[ \asort{noun} 
            per & \@1 \\
            num & \@2 \\
            case & nom \\
          \] \],
          \[ head & \[ \asort{noun} 
            case & acc \\
          \] \]
        \>} \\
      \]
    \end{avm}%
  }\\
  \Zeile
  \onslide<+->
  \footnotesize Übrigens: \alert{Kongruenz} ist Strukturteilung zwischen \textsc{head}-Merkmalen von Kopf und Nicht-Kopf,\\
  \alert{Valenz} ist Strukturteilung zwischen der \textsc{subcat} des Kopfs und \textsc{head} des Nicht-Kopfs.\\
  \grau{Diese Formulierung dient vor allem der Veranschaulichung.}
\end{frame}

\section{Komplementation}

\newcommand{\AvmDreiA}{%
  \begin{avm}
    \[ phon & testet \\
      head & \[ \asort{verb} 
        vform & fin \\
      \] \\
      subcat & \<
       \[ head & \[ \asort{noun} 
          case & nom \\
        \] \],
        \alert{\@1 \[ head & \[ \asort{noun} 
          case & acc \\
        \] \]}
      \> \\
    \]
  \end{avm}%
}

\newcommand{\AvmDreiB}{%
  \begin{avm}
    \alert{\@1 \[ phon & einen Verstärker \\
      head & \[ \asort{noun} 
        case & acc \]
      \]}
  \end{avm}
}

\newcommand{\AvmDreiC}{%
  \begin{avm}
    \gruen{\@2 \[ phon & Jörg \\
      head & \[ \asort{noun} 
       case & nom $\vee$ acc $\vee$ dat \]
     \]}
  \end{avm}
}

\newcommand{\AvmDreiAa}{%
  \begin{avm}
    \[ phon & testet einen Verstärker \\
      head & \[ \asort{verb} 
        vform & fin \\
      \] \\
      subcat & \<
      \gruen{\@2 \[ head & \[ \asort{noun} 
          case & nom \\
        \] \]}
      \> \\
    \]
  \end{avm}%
}

\newcommand{\AvmDreiAb}{%
  \begin{avm}
    \[ phon & Jörg testet einen Verstärker \\
      head & \[ \asort{verb} 
        vform & fin \\
      \] \\
      subcat & \<\> \\
    \]
  \end{avm}%
}

\begin{frame}
  {Wie steuert Valenz den Phrasenaufbau?}
  \onslide<+->
  \onslide<+->
  Die \textsc{subcat}-Liste wird bei Kombination mit Komplementen \alert{reduziert}.\\
  \grau{\footnotesize Die Bäume dienen nur der Veranschaulichung. Kongruenz wird aus Platzgründen nicht dargestellt.}\\
  \onslide<+->
  \Halbzeile
  \centering
  \only<3|handout:0>{\begin{forest}
    [VP
      [NP, gruentree [\it Jörg, narroof]]
      [{V$'$}
        [NP, bluetree [\it einen Verstärker, narroof]]
        [V [\it testet]]
      ]
    ]
  \end{forest}}%
  \only<4|handout:0>{\begin{forest}
    [{V$\langle\rangle$}
      [NP, gruentree [\it Jörg, narroof] ]
      [{V$\langle$\gruen{NP\Sub{nom}}$\rangle$}
        [NP, bluetree [\it einen Verstärker, narroof]]
        [{V$\langle$\gruen{NP\Sub{nom}}, \alert{NP\Sub{acc}}$\rangle$} [\it testet]]
      ]
    ]
  \end{forest}}%
  \only<5|handout:0>{\begin{forest}
    [{V$\langle\rangle$}
      [NP, gruentree [\it Jörg, narroof] ]
      [{V$\langle$\gruen{NP\Sub{nom}}$\rangle$}
        [NP, bluetree [\it einen Verstärker, narroof]]
        [\scalebox{0.45}{\AvmDreiA}]
      ]
    ]
  \end{forest}}
  \only<6|handout:0>{\begin{forest}
    [{V$\langle\rangle$}
      [NP, gruentree [\it Jörg, narroof] ]
      [{V$\langle$\gruen{NP\Sub{nom}}$\rangle$}
        [\scalebox{0.45}{\AvmDreiB}]
        [\scalebox{0.45}{\AvmDreiA}]
      ]
    ]
  \end{forest}}
  \only<7|handout:0>{\begin{forest}
    [{V$\langle\rangle$}
      [NP, gruentree [\it Jörg, narroof] ]
      [\scalebox{0.45}{\AvmDreiAa}
        [\scalebox{0.45}{\AvmDreiB}]
        [\scalebox{0.45}{\AvmDreiA}]
      ]
    ]
  \end{forest}}
  \only<8|handout:0>{\begin{forest}
    [{V$\langle\rangle$}
      [\scalebox{0.45}{\AvmDreiC}]
      [\scalebox{0.45}{\AvmDreiAa}
        [\scalebox{0.45}{\AvmDreiB}]
        [\scalebox{0.45}{\AvmDreiA}]
      ]
    ]
  \end{forest}}
  \only<9>{\begin{forest}
    [\scalebox{0.45}{\AvmDreiAb}
      [\scalebox{0.45}{\AvmDreiC}]
      [\scalebox{0.45}{\AvmDreiAa}
        [\scalebox{0.45}{\AvmDreiB}]
        [\scalebox{0.45}{\AvmDreiA}]
      ]
    ]
  \end{forest}}
\end{frame}

\begin{frame}
  {Derselbe Beispielsatz als Merkmalbeschreibung}
  \onslide<+->
  \onslide<+->
  Die Bäume sind nur ein Konstrukt, die Merkmalstrukturen real.\\
  \onslide<+->
  \Halbzeile
  \centering 
  \scalebox{0.6}{\begin{avm}
    \[ phon & Jörg einen Verstärker testet \\
      \rot{head} & \rot{\@3} \\
      subcat & \<\> \\
      non-hd-dtr & \gruen{\@2 \[ phon & Jörg \\
        head & \[ \asort{noun} case & nom \]
      \]} \\
      \rot{hd-dtr} & \[ phon & einen Verstärker testet \\
        \rot{head} & \rot{\@3} \\
        subcat & \<\gruen{\@2}\> \\
        non-hd-dtr & \alert{\@1 \[ phon & einen Verstärker \\
          head & \[ \asort{noun} case & acc \]
        \]} \\
        \rot{hd-dtr} & \[ phon & testet \\
          \rot{head} & \rot{\@3 \[ \asort{verb} vform & fin \]} \\
          subcat & \<\gruen{\@2}, \alert{\@1}\> \\
        \] \\
      \] \\
    \]
  \end{avm}}
\end{frame}

\begin{frame}
  {Projektionsstatus}
  \onslide<+->
  \onslide<+->
  Was macht eine \alert{Phrase zu einer Phrase}?\\
  \grau{\footnotesize Betrachtet im Gegensatz zu Kopf und Bar-Ebene \ldots}\\
  \Zeile
  \begin{itemize}[<+->]
    \item Köpfe X\Up{0} | \alert{Volle Valenz}
    \item Bar-Ebene X\Prm | \alert{Reduzierte Valenz}
    \item Phrase XP | \alert{Vollständig abgebundene Valenz}
      \Halbzeile
    \item Verhindert \rot{*\textit{dass Jörg Auto repariert}} usw.
  \end{itemize}
  \onslide<+->
  \Zeile
  \begin{block}
    {Maximalprojektionen in HPSG}
    Strukturen mit leerer \textsc{subcat}-Liste sind Maximalprojektionen.
  \end{block}
\end{frame}

\begin{frame}
  {Phrasenstatus anzeigen}
  \onslide<+->
  \onslide<+->
  Auch die NPs müssen \orongsch{\textsc{subcat}-empty} sein.\\
  \onslide<+->
  \Halbzeile
  \centering 
  \scalebox{0.6}{\begin{avm}
    \[ phon & Jörg einen Verstärker testet \\
      head & \@3 \\
      subcat & \<\> \\
      non-hd-dtr & \@2 \[ phon & Jörg \\
        head & \[ \asort{noun} case & nom \] \\
        \orongsch{subcat} & \orongsch{\<\>} \\
      \] \\
      hd-dtr & \[ phon & einen Verstärker testet \\
        head & \@3 \\
        subcat & \<\@2\> \\
        non-hd-dtr & \@1 \[ phon & einen Verstärker \\
          head & \[ \asort{noun} case & acc \] \\
          \orongsch{subcat} & \orongsch{\<\>} \\
        \] \\
        hd-dtr & \[ phon & testet \\
          head & \@3 \[ \asort{verb} vform & fin \] \\
          subcat & \<\@2, \@1\> \\
        \] \\
      \] \\
    \]
  \end{avm}}
\end{frame}

\begin{frame}
  {Einige Punkte zur Beachtung}
  \onslide<+->
  \onslide<+->
  \centering 
  \begin{tabular}[h]{cc}
    \begin{minipage}{0.35\textwidth}
      \scalebox{0.5}{\begin{avm}
        \[ phon & Jörg einen Verstärker testet \\
          \rot{head} & \rot{\@3} \\
          subcat & \<\> \\
          non-hd-dtr & \gruen{\@2 \[ phon & Jörg \\
            head & \[ \asort{noun} case & nom \] \\
            \orongsch{subcat} & \orongsch{\<\>} \\
          \]} \\
          \rot{hd-dtr} & \[ phon & einen Verstärker testet \\
            \rot{head} & \rot{\@3} \\
            subcat & \<\gruen{\@2}\> \\
            non-hd-dtr & \alert{\@1 \[ phon & einen Verstärker \\
              head & \[ \asort{noun} case & acc \] \\
              \orongsch{subcat} & \orongsch{\<\>} \\
            \]} \\
            \rot{hd-dtr} & \[ phon & testet \\
              \rot{head} & \rot{\@3 \[ \asort{verb} vform & fin \]} \\
              subcat & \<\gruen{\@2}, \alert{\@1}\> \\
            \] \\
          \] \\
        \]
      \end{avm}}
    \end{minipage} & %
    \begin{minipage}{0.55\textwidth}
      \begin{itemize}[<+->]
        \item Einträge auf der lexikalischen \textsc{subcat}\\
          des Verbs | Minimale Spezifikation\\
          der Komplemente (\textsc{case}, evtl.\ Kongruenz)
        \item Über \textsc{phon} zum Beispiel keine Vorgaben
        \item Konkrete NPs | \alert{Unifikation} mit\\
          dieser Information
          \Viertelzeile
        \item In der großen Struktur | Unter \alert{\mybox{1}} und \gruen{\mybox{2}}\\
          \alert{überall die volle Information}
          \Viertelzeile
        \item Falls nicht unifizierbar | Keine\\
          größere Struktur bzw. kein grammatischer\\
          Satz, keine grammatische VP usw.
      \end{itemize} 
    \end{minipage} \\
  \end{tabular}
\end{frame}

\section{Grammatikregeln}

\begin{frame}
  {Listenverknüpfung}
  \onslide<+->
  \onslide<+->
  Wir haben immer noch keine \alert{Regel} für die Komplementabbindung!\\
  \Zeile
  \begin{itemize}[<+->]
    \item Bei der Verbindung von Kopf \mybox{1} Komplement \mybox{2}
      \begin{itemize}[<+->]
        \item Unifikation des \alert{letzten Elements der \textsc{subcat} von \mybox{1}} mit \mybox{2}
        \item Resultierende Phrase | Kopfmerkmale kopiert von \textsc{hd-dtr}
        \item Resultierende Phrase | \textsc{subcat} von \mybox{1} \alert{ohne das letzte Element}
        \item \textsc{phon}-Werte der Phrase | \alert{Aneinandergehängte \textsc{phon}-Werte} der Töchter
      \end{itemize}
      \Halbzeile
    \item Teilung der \textsc{subcat} in \alert{letztes Element} und \alert{Rest der Liste davor}
    \item \alert{"`Rest der Liste"' möglicherweise leer} (z.\,B.\ bei Abbindung des Subjekts)
      \Halbzeile
    \item \alert{Konkatenationsoperator $\oplus$}
        \begin{itemize}[<+->]
          \item Verknüpft zwei Listen L\Sub{1} und L\Sub{2} zu neuer Liste L\Sub{3}: \alert{$L_3=L_1\oplus L_2$}
          \item L\Sub{3} enthält alle Elemente von L\Sub{1} gefolgt von allen Elementen von L\Sub{2}
          \item L\Sub{1} und\slash oder L\Sub{2} möglicherweise leer
        \end{itemize}
  \end{itemize}
\end{frame}

\begin{frame}
  {Zusammenbau von \textsc{phon}-Listen}
  \onslide<+->
  \onslide<+->
  Listen von Phonemketten\slash Segmentketten können wir konkatenieren.\\
  \onslide<+->
  \Halbzeile
      \scalebox{0.5}{\begin{avm}
        \[ phon & \alert{\@{12}} $\oplus$ \orongsch{\@{11}} $\oplus$ \gruen{\@{10}} \\
          head & \@3 \\
          subcat & \<\> \\
          non-hd-dtr & \@2 \[ phon & \alert{\@{12} \<\rm\it Jörg\>} \\
            head & \[ \asort{noun} case & nom \] \\
            subcat & \<\> \\
          \] \\
          hd-dtr & \[ phon & \orongsch{\@{11}} $\oplus$ \gruen{\@{10}} \\
            head & \@3 \\
            subcat & \<\@2\> \\
            non-hd-dtr & \@1 \[ phon & \orongsch{\@{11} \<\rm\it einen, Verstärker\>} \\
              head & \[ \asort{noun} case & acc \] \\
              subcat & \<\> \\
            \] \\
            hd-dtr & \[ phon & \gruen{\@{10} \<\rm\it testet\>} \\
              head & \@3 \[ \asort{verb} vform & fin \] \\
              subcat & \<\@2, \@1\> \\
            \] \\
          \] \\
        \]
      \end{avm}}\\
      \Halbzeile
      \centering 
      \grau{\footnotesize Darüber sprechen wir in Zusammenhang mit Wortstellung nochmal.}
\end{frame}


\begin{frame}
  {Reduktion von \textsc{subcat}-Listen}
  \onslide<+->
  \onslide<+->
  Im Ergebnis ist die untenstehende Beschreibung äquivalent zur vorherigen.\\
  \onslide<+->
  \Halbzeile
      \scalebox{0.5}{\begin{avm}
        \[ phon & \@{12} $\oplus$ \@{11} $\oplus$ \@{10} \\
          head & \@3 \\
          subcat & \orongsch{\<\>} \\
          non-hd-dtr & \gruen{\@2 \[ phon & Jörg \\
            head & \[ \asort{noun} case & nom \] \\
            subcat & \<\> \\
          \]} \\
          hd-dtr & \[ phon & \@{11} $\oplus$ \@{10} \\
            head & \@3 \\
            subcat & \orongsch{\<\>}\ $\oplus$ \<\gruen{\@2}\> \\
            non-hd-dtr & \alert{\@1 \[ phon & \@{11} \<\rm\it einen, Verstärker\> \\
              head & \[ \asort{noun} case & acc \] \\
              subcat & \<\> \\
            \]} \\
            hd-dtr & \[ phon & \@{10} \<\rm\it testet\> \\
              head & \@3 \[ \asort{verb} vform & fin \] \\
              subcat & \<\gruen{\@2}\>\ $\oplus$ \<\alert{\@1}\> \\
            \] \\
          \] \\
        \]
      \end{avm}}
\end{frame}

\begin{frame}
  {Dominanzregel für Komplementierung}
  \onslide<+->
  \onslide<+->
  \Zeile
  \centering 
  \raisebox{-1.5\baselineskip}{\textit{head-argument-phrase}\ $\Rightarrow$\ }%
  \begin{avm}
    \[ subcat & \@1 \\
       hd-dtr|subcat & \@1\ $\oplus$\ \<\@2\> \\
       non-hd-dtr & \@2 
     \]
  \end{avm}\\
  \Zeile
  \raggedright
  \begin{itemize}[<+->]
    \item \alert{Implikationsregel} | Für alle Zeichen vom Typ \textit{hd-arg-phrase} gilt \ldots
    \item Wichtig: \mybox{1} ist die "`restliche"' Valenzliste, \rot{\mybox{2} ist keine Liste}!
    \item \alert{Wenn \mybox{1} leer ist}, ist die betreffende \textit{hd-arg-phrase} eine Maximalprojektion.
    \item \alert{Das Pipe-Zeichen |} kürzt Pfade durch Merkmalsbeschreibungen ab.\\
      \Viertelzeile
      \grau{\raisebox{-0.25\baselineskip}{\begin{avm}
        \[ hd-dtr|subcat & \@1\ $\oplus$\ \<\@2\> \]
      \end{avm}}~\raisebox{-0.5\baselineskip}{=}~%
      \begin{avm}
        \[ hd-dtr & \[ subcat & \@1\ $\oplus$\ \<\@2\> \] \]
      \end{avm}}
      \Viertelzeile
    \item Achtung: \rot{Normalerweise (auch bei \citealt{MuellerLehrbuch3}) ist \textsc{non-hd-dtrs} eine Liste},\\
      wir brauchen aber immer \alert{nur eine Nicht-Kopf-Tochter}.
  \end{itemize}
\end{frame}

\begin{frame}
  {Regel für Weitergabe der Kopfmerkmale}
  \onslide<+->
  \onslide<+->
  Das \alert{Kopfmerkmalprinzip}\\
  \grau{\footnotesize Es werden noch andere Phrasentypen mit Kopf eingeführt werden.}\\
  \onslide<+->
  \Zeile
  \centering 
  \raisebox{-0.5\baselineskip}{\textit{headed-phrase}\ $\Rightarrow$\ }%
  \begin{avm}
    \[ head & \@1 \\
       hd-dtr|head & \@1 \\
     \]
  \end{avm}\\
  \Zeile
  \raggedright
  \begin{itemize}[<+->]
    \item Das gilt für alle \textit{headed-phrases} \alert{inkl.\ aller Untertypen}.
    \item Wichtig: Es darf nichts in die \textsc{head}-Merkmale, das nicht\\
      an die Projektion nach oben weitergegeben werden darf\slash soll.
    \item Die Valenz bzw.\ \rot{\textsc{subcat} darf also kein Kopfmerkmal sein}.\\
      \grau{Sonst hätte jede Projektionsstufe dieselbe Valenz wie der Kopf.}
      \Halbzeile
    \item Konsequenz | \alert{Die Kopfmerkmale von Nicht-Kopf-Töchtern werden nicht weitergegeben!}
    \item \small Das entspricht der Generalisierung, dass die syntaktischen Eigenschaften von Nicht-Köpfen\\
      für die Syntax über die direkt einschließende Phrase hinaus irrelevant sind.
  \end{itemize}
\end{frame}

\begin{frame}
  {Typhierarchie für \textit{sign}}
  \onslide<+->
  \onslide<+->
  Die Typenhierarchie wird beim Grammatikschreiben immer komplexer.\\
  \onslide<+->
  \Zeile
  \centering 
  \begin{forest}
    typehierarchy
  [sign
    [word]
    [phrase
      [non-headed-phrase]
      [headed-phrase
        [\ldots]
        [head-argument-phrase]
        [\ldots]]]]
  \end{forest}
\end{frame}

\section{Nächste Woche}

\begin{frame}
  {Vorbereitung}
  \onslide<+->
  \onslide<+->
  \centering 
  \large
  Nächste Woche reden wir über Verbsemantik und thematische Rollen.\\
  \onslide<+->
  \Zeile
  \rot{Sie sollten dringend vorher aus dem HPSG-Buch\\
  Kapitel 5 lesen!}\\
  \onslide<+->
  \Viertelzeile
  Das sind gerade mal \gruen{neun} Seiten.
\end{frame}
