\section{Einleitung}

\begin{frame}
  {Adjunkte und Spezifikatoren in HPSG}
  \onslide<+->
  \onslide<+->
  Kopf-Adjunkt-Phrasen und Kopf-Determinierer-Konstruktionen\\
  \Zeile
  \begin{itemize}[<+->]
    \item Was ist Modifikation?
    \item Intersektive und nicht-intersektive Adjektive
    \item NP-modifizierende PPs (PP-Attribute)
    \item Wozu braucht man ein gesondertes Spezifikatorprinzip?
    \item Genitivattribute
  \end{itemize}
  \Zeile
  \onslide<+->
  \centering 
  \grau{\citet[Kapitel~6]{MuellerLehrbuch3}}\\
\end{frame}

\section{Daten: Adjunkte und Spezifikatoren}

\begin{frame}
  {Ein Beispiel aus \textit{Alles klar!} 7\slash 8}
  Hier soll der Gebrauch von \alert{Adjektiven} geübt werden\ldots\\
  \Zeile
  \begin{center}
    \begin{minipage}{0.15\textwidth}\footnotesize
      \orongsch{\it\textbf{traumhaft}\\
      unvergesslich\\
      besten\\
      bunt\\
      spannend\\
      atemberaubend\\
      toll\\
      gemütlich\\
      riesig\\
      beheizt\\
      nächtlich\\
      groß\\
      interessant}
    \end{minipage}\hspace{0.05\textwidth}\begin{minipage}{0.7\textwidth}\footnotesize\setstretch{1.25}
      \alert{Lies die Anzeige eines Veranstalters für Jugendreisen. Überlege, wohin die Wörter aus der Randspalte passen könnten, und setze sie mit der richtigen Endung ein.}\\

      \textbf{\ul{\textit{Traumhafte}} Reisen mit den \ul{\ \ \ \ \ \ \ \ \ \ \ } Freunden!}\\

      \setstretch{1.25}In der \ul{\ \ \ \ \ \ \ \ \ \ \ }
      Natur der Alpen erwartet euch ein \ul{\ \ \ \ \ \ \ \ \ \ \ }
      Freizeitprogramm: \ul{\ \ \ \ \ \ \ \ \ \ \ }
      Sportturniere, \ul{\ \ \ \ \ \ \ \ \ \ \ }
      Reitausflüge übers Land, \ul{\ \ \ \ \ \ \ \ \ \ \ }
      Wanderungen mit Fackeln, \ul{\ \ \ \ \ \ \ \ \ \ \ }
      Partys in unserer Disko. Wir bieten ein \ul{\ \ \ \ \ \ \ \ \ \ \ }
      Sportgelände mit \ul{\ \ \ \ \ \ \ \ \ \ \ }
      Swimmingpool, einen \ul{\ \ \ \ \ \ \ \ \ \ \ }
      Kletterturm, einen Computerraum und ein eigenes Kino. Das ist doch wesentlich \ul{\ \ \ \ \ \ \ \ \ \ \ }
      , als mit den Eltern in den Urlaub zu fahren, oder? Dieser Urlaub wird bestimmt ein \ul{\ \ \ \ \ \ \ \ \ \ \ }
      Erlebnis!
    \end{minipage}
  \end{center}
  \Viertelzeile
  \tiny \grau{Maempel, Oppenländer \& Scholz. 2012. \textit{Alles klar!} 7\slash 8. Lern- und Übungsheft Grammatik und Zeichensetzung. Berlin: Cornelsen. (Layout ungefähr nachgebaut.)}
\end{frame}


\begin{frame}
  {Warum fehlen hier viele bildungssprachliche Arten von Adjektiven?}
  \onslide<+->
  \onslide<+->
  \small
  Diese Adjektivklassen fehlen nahezu vollständig in der Aufgabe
  \Halbzeile
  \begin{itemize}[<+->]
    \item \alert{temporal} | \textit{der \alert{gestrige} Vorfall}
    \item \alert{quantifizierend} (relativ, Zählsubstantiv) | \textit{die \alert{zahlreichen} Äpfel}
    \item \alert{quantifizierend} (relativ, Stoffsubstantiv) | \textit{\alert{reichlich} Apfelkompott}
    \item \alert{quantifizierend} (absolut) | \textit{die \alert{drei} Bienen}
    \item \alert{intensional} | \textit{der \alert{ehemalige} Präsident} \slash\ \textit{die \alert{fiktive} Gestalt}
    \item \alert{phorisch} | \textit{die \alert{obigen}}/\textit{\alert{weiteren}}/\textit{\alert{anderen} Ausführungen}
  \end{itemize}
  \Halbzeile
  \onslide<+->
  Fällt Ihnen was auf?
  \begin{itemize}[<+->]
    \item Das sind im Wesentlichen die, die \rot{nicht prädikativ verwendbar} sind.
    \item Der Wie-Wort-Test basiert aber auf prädikativer Verwendbarkeit.
    \item Aber viele Adjektive sind eben nicht prädikativ verwendbar. 
  \end{itemize}
\end{frame}



\begin{frame}
  {Intersektiv oder nicht}
  \onslide<+->
  \onslide<+->
  Man kann nicht alle Adjektivmodifikationen als Schnittmengenbildung auffassen.\\
  \grau{\footnotesize Schnittmenge im Sinn von: "`x hat die N-Eigenschaft und x hat die Adj-Eigenschaft"'}\\
  \Zeile
  \begin{itemize}[<+->]\small
    \item \alert{das türkise Buch} | Objekt x: x \gruen{ist} Buch \gruen{und} x \gruen{ist} türkis
      \Halbzeile
    \item \alert{der ehemalige Kanzler} | Objekt x: x \orongsch{war} Kanzler \orongsch{vor dem jetzigen Zeitpunkt}
    \item \alert{das fiktive Pferd} | Objekt x: x \orongsch{existiert nur in einer fiktiven Welt als Pferd}
    \item \alert{der gestrige Vorfall} | Objekt x: x \gruen{ist} Vorfall, \orongsch{der Zeitunkt von x liegt im Intervall "`gestern"'}
    \item \alert{die zahlreichen Äpfel} | \orongsch{große Menge} M von Objekten: alle x in M \gruen{sind} Äpfel
    \item \alert{die drei Äpfel} | \orongsch{dreielementige Menge} M von Objekten: alle x in M sind Äpfel
    \item \alert{reichlich Apfelkompott} | \orongsch{eine große Portion} x: Material von x \gruen{ist} Apfelkompott
    \item \alert{meine obige Ausführung} | Objekt x: x \gruen{ist} Ausführung \gruen{und} \orongsch{x steht von der aktuellen Texposition aus weiter oben} und \rot{x ist "`von mir"'}
  \end{itemize}
  \onslide<+->
  \Halbzeile
  \centering 
  Alle \orongsch{orange markierten} semantischen Beiträge kann man nicht als Eigenschaftsaussagen\\
  über Objekte in der aktuellen und aktualen Welt analysieren.
\end{frame}

\begin{frame}
  {Präpositionen als NP-Modifikatoren}
  \onslide<+->
  \onslide<+->
  Doppelter semantischer und syntaktischer Bezug \onslide<+->\ | \alert{\textit{das Buch auf dem Tisch}}\\
  \Halbzeile
  \begin{itemize}[<+->]
    \item Semantik
      \Viertelzeile
      \begin{itemize}[<+->]
        \item Objekt x: x \gruen{ist} Buch
        \item Objekt y: y \gruen{ist} Tisch
        \item Lokale Relation: \orongsch{x befindet sich auf y}
        \item %
          \scalebox{0.7}{\begin{avm}
            \[restr & \<\[ \asort{ontop-loc-rel} positioned-obj & \[ \asort{index} \] \\ reference-obj & \[ \asort{index} \] \]\>\]
          \end{avm}}
      \end{itemize}
      \Halbzeile
    \item Syntax
      \begin{itemize}[<+->]
        \item Valenz von \textit{auf}: %
          \raisebox{0.4\baselineskip}{\begin{avm}
            \[ cat|subcat & \< \textnormal{NP\Sub{Dat}} \>\]
          \end{avm}}
        \item PP \textit{auf dem Tisch}: \orongsch{Adjunkt} zu N$'$ \textit{Buch}
        \item Viele Adjunkte müssen aber die Semantik des N-Kopfs komplett umbauen.
        \item Wie geht das angesichts des Semantikprinzips für Phrasen mit Kopf?
      \end{itemize}
  \end{itemize}
\end{frame}

\begin{frame}
  {Pränominale Genitive und Possessiva}
  \onslide<+->
  \onslide<+->
  Zum Beispiel \alert{\textit{mein Buch}} oder \alert{\textit{Doros Wohnung}}\\
  \Halbzeile
  \begin{itemize}[<+->]
    \item Semantik
      \Viertelzeile
      \begin{itemize}[<+->]
        \item Objekt x: x \gruen{ist} Wohnung
        \item Objekt y: y \gruen{ist das Objekt mit Namen Doro}
        \item Besitzrelation: \orongsch{x gehört (zu) y}
        \item %
          \scalebox{0.7}{\begin{avm}
            \[restr & \<\[ \asort{possess-rel} possessor-obj & \[ \asort{index} \] \\ possessed-obj & \[ \asort{index} \] \]\>\]
          \end{avm}}
      \end{itemize}
      \Halbzeile
    \item Syntax
      \begin{itemize}[<+->]
        \item Valenz von \textit{Wohnung}: %
          \raisebox{0.4\baselineskip}{\begin{avm}
            \[ cat|subcat & \< \textnormal{Det $\vee$ NP\Sub{Gen}} \>\]
          \end{avm}}
        \item Dass die NP oder der Det eine \textit{possess-rel} einführt, wissen sie nur selbst.
        \item Wie kann angesichts des \alert{Semantikprinzips} die Semantik des N-Kopfs\\
          entsprechend modifiziert werden?
      \end{itemize}
  \end{itemize}
\end{frame}


\section{Adjektiv-Modifikation in HPSG}

\begin{frame}
  {Lexikoneintrag eines intersektiven Adjektivs}
  \onslide<+->
  \onslide<+->
  Einführung einer \textsc{restr} \ldots\ und sonst?\\
  \onslide<+->
  \Halbzeile
  \centering 
  \scalebox{0.8}{\begin{avm}
    \[ \asort{word} 
      phon & \phon{türkisem } \\
      cat & \[
        head & \[ \asort{adj} 
          cas & dat \\
          \only<4->{
            \alert{mod} & \alert{\[ cat & \[
              head & \[ \asort{noun} \] \\
              subcat & \< \[ cat|head & \[ \asort{det} \] \]\> \\
            \]
          \]}
          }
        \] \\
        subcat & \<\>
      \] \\
      cont|restr & \< \[ \asort{cyan-rel} theme & \@1 \] \> \\
    \]
  \end{avm}}\\
  \Halbzeile
  \onslide<5->
  \alert{Der Wert des \textsc{mod}-Merkmals entspricht einem N$'$!}
\end{frame}

\begin{frame}
  {Kopf-Adjunkt-Schema}
  \onslide<+->
  \onslide<+->
  Wie verbindet sich so ein Adjektiv mit dem N$'$?\\
  \onslide<+->
  \Zeile
  \centering 
  \raisebox{-1.25\baselineskip}{\textit{head-adjunct-phrase} $\Rightarrow$} \scalebox{0.8}{%
    \begin{avm}
      \[ head-dtr & \alert{\@1} \\
        non-hd-dtr & \< \[ cat & \[ head|mod & \alert{\@1} \\
          subcat & \<\>
        \]
      \]\> 
      \]
    \end{avm}
  }\\
  \Zeile
  \begin{itemize}[<+->]
    \item Das \alert{Adjunkt (z.\,B.\ ein Adjektiv) selegiert den Kopf (z.\,B.\ das N$'$)}.
    \item Dadurch können wir dem Adjektiv Zugriff auf die Semantik von N$'$ geben.
    \item Außerdem ist es so: Adjunkte legen ihre Kompatibilität zum Kopf fest.
    \item Es ist nicht zielführend, Köpfen eine Liste der kompatiblen Adjunkte mitzugeben.
  \end{itemize}
\end{frame}

\begin{frame}
  {Eine einfache \textit{head-adjunct-phrase}}
  \onslide<+->
  \onslide<+->
  \centering 
  \scalebox{0.8}{\begin{avm}
    \[ \asort{head-adjunct-phrase} 
      phon & \<\textnormal{\it türkisem,Buch}\> \\
       cat & \[
         head & \@1 \\
         subcat & \@2 \\
       \] \\
       hd-dtr & \@3 \[ 
         phon & \< \textnormal{\it Buch} \> \\ 
         cat & \[ 
           head & \@1 \[ \asort{noun} cas & dat \] \\
           subcat & \@2 \< \[ cat|hd & \[ \asort{det} \] \] \> \\
         \]
       \] \\
       non-hd-dtr & \[ 
         phon & \< \textnormal{\it türkisem} \> \\
          cat & \[
            head & \[ \asort{adj} 
              mod & \@3 \\
            \] \\
            subcat & \<\> \\
          \]
       \] \\
    \]
  \end{avm}}
\end{frame}

\begin{frame}
  {Erweiterter Lexikoneintrag eines attributiven Adjektivs}
  \onslide<+->
  \onslide<+->
  \centering 
  \scalebox{0.8}{\begin{avm}
    \[ \asort{word} 
      phon & \phon{türkisem } \\
      cat & \[
        head & \[ \asort{adj} 
          cas & dat \\
          \only<3->{
            mod & \[ cat & \[
                head & \[ \asort{noun} \] \\
                subcat & \< \[ cat|head & \[ \asort{det} \] \]\> \\
              \] \\
              cont & \[
                ind & \@1 \\
                restr & \@2 \\
              \] \\
          \]
          }
        \] \\
        subcat & \<\>
      \] \\
      cont & \[
        ind  & \@1 \[ gen & mask \\ num & sg \] \\
        restr & \< \[ \asort{cyan-rel} theme & \@1 \] \> $\oplus$ \@2 \\
      \]
    \]
  \end{avm}}\\
\end{frame}


\begin{frame}
  {Eine \textit{head-adjunct-phrase} mit Semantik}
  \onslide<+->
  \onslide<+->
  \centering 
  \scalebox{0.7}{\begin{avm}
    \[
%      \asort{head-adjunct-phrase} 
%      phon & \<\textnormal{\it türkisem,Buch}\> \\
       cat & \[
         head & \@1 \\
         subcat & \@2 \\
       \] \\
       \only<3->{\alert{cont} & \alert{\@6} \\}
       hd-dtr & \@3 \[ 
%         phon & \< \textnormal{\it Buch} \> \\ 
         cat & \[ 
           head & \@1 \[ \asort{noun} cas & dat \] \\
           subcat & \@2 \< \[ cat|hd & \[ \asort{det} \] \] \> \\
         \] \\
         cont & \[
           ind & \@4 \\
           restr & \@5 \< \[ \asort{book-rel} instance & \@4 \]\> \\
         \]
       \] \\
       non-hd-dtr & \[ 
%         phon & \< \textnormal{\it türkisem} \> \\
          cat & \[
            head & \[ \asort{adj} 
              mod & \@3 \\
            \] \\
%            subcat & \<\> \\
          \] \\
          cont & \alert{\@6 \[
            ind & \@4 \[ gen & masc \\ num & sg \] \\
            restr & \< \[ \asort{cyan-rel} theme & \@4 \]\> $\oplus$\ \@5\\
          \]} \\
       \] \\
    \]
  \end{avm}}
\end{frame}

\begin{frame}
  {Regeln, die wir dafür brauchen}
  \onslide<+->
  \onslide<+->
  \alert{Schema} für \textit{head-adjunct-phrase}\\
  \grau{\footnotesize In Kopf-Adjunkt-Strukturen wird der Kopf über \textsc{hd|mod} vom Adjunkt selegiert.}\\
  \onslide<+->
  \centering 
  \Viertelzeile
  \raisebox{-1.25\baselineskip}{\textit{head-adjunct-phrase} $\Rightarrow$} \scalebox{0.7}{%
    \begin{avm}
      \[ 
%         cont & \gruen{\@2} \\
         head-dtr & \gruen{\@1} \\
         non-hd-dtr & \[ cat & \[ head|mod & \gruen{\@1} \\
          subcat & \<\>
        \] \\
%        cont & \gruen{\@2}
      \] 
      \]
    \end{avm}}\\
    \raggedright
    \onslide<+->
    \Zeile
    Ergänzung (zweiter Teil) des \alert{Semantikprinzips}\\
    \grau{\footnotesize In Kopf-Adjunkt-Strukturen wird die Semantik des Adjunkts an der Phrase realisiert.}\\
    \onslide<+->
    \Viertelzeile
    \centering 
    \raisebox{-0.75\baselineskip}{\textit{head-adjunct-phrase} $\Rightarrow$} \scalebox{0.7}{%
      \begin{avm}
        \[ cont & \gruen{\@1} \\
           non-hd-dtr & \[ 
             cont & \gruen{\@1}
        \] 
        \]
    \end{avm}}\\
    \onslide<+->
    \raggedright
    \Zeile
    Ergänzung des \alert{Subkategorisierungsprinzips}\\
    \grau{\footnotesize In Kopf-Nichtargument-Strukturen wird die \textsc{subcat} des Kopfs unverändert an der Phrase realisiert.}\\
    \onslide<+->
    \centering 
    \Viertelzeile
    \raisebox{-0.5\baselineskip}{\textit{head-non-argument-phrase} $\Rightarrow$} \scalebox{0.7}{%
      \begin{avm}
        \[ cat|subcat & \gruen{\@1} \\
          hd-dtr|cat|subcat & \gruen{\@1} \\
        \]
    \end{avm}} \\
\end{frame}


\begin{frame}
  {Zusammenfassung bisher}
  \onslide<+->
  \onslide<+->
  Wie funktioniert Modifikation in HPSG?\\
 \Viertelzeile 
  \begin{itemize}[<+->]
    \item Das \alert{Adjunkt selegiert den Kopf} über ein Kopfmerkmal \textsc{mod}.\\
      \grau{\footnotesize Das entspricht der Intuition, dass Adjunkte ihre Kompatibilität zum Kopf bestimmen.}
    \item Das Adjunkt bekommt dadurch \alert{Zugriff auf die Semantik} des Kopfs.
    \item Das Adjunkt kann die \textsc{restr} des Kopf einfach aufsammeln (intersektiv),\\
          oder es modifiziert die Semantik des Kopfs (intensional), s.\,u.
    \item Die \alert{\textsc{subcat} des Kopfs} wird unverändert weitergegeben.\\
      \grau{\footnotesize \textit{Buch} hat dieselbe Valenz wie \textit{türkisem Buch}.}
    \item Wie in jeder Kopf-Struktur werden die Kopfmerkmale des Kopfs weitergegeben.\\
      \grau{\footnotesize Ein N$'$ mit einer attributiven AP ist immer noch ein N$'$.}
    \item Ein attributives Adjektiv erzwingt \alert{\textsc{Per-Num-Gen}-Kongruenz innerhalb der NP},\\
      indem es seinen Index mit dem des Kopfs identifiziert.
      \Halbzeile
    \item \orongsch{Aber wie geht das mit intensionalen Adjektiven?}
    \item \orongsch{Und warum ist \textsc{mod} ein Kopfmerkmal?}
  \end{itemize}
\end{frame}

\begin{frame}
  {Lexikoneintrag eines intensionalen Adjektivs}
  \onslide<+->
  \onslide<+->
  Es ist nicht adäquat, einfach die \textsc{restr} aufzusammeln.\\
  \onslide<+->
  Die \textsc{restr} des via \textsc{mod} selegierten Kopfs muss modifiziert werden.\\
  \onslide<+->
  \Halbzeile
  \centering 
  \scalebox{0.8}{%
    \begin{avm}
      \[ \asort{word} 
        phon & \< \textnormal{fiktiv} \> \\ 
        cat & \[
          head & \[ \asort{adj} 
            mod & \[
              cat & \[ head & \[ \asort{noun} \] \\
                       subcat & \<\> \] \\
              cont & \[ 
                ind & \@1 \\
                restr & \alert<4->{\@2} \\
              \]
            \]\\
          \]\\
          subcat & \<\> \\
        \]\\
        cont & \[ 
          ind & \@1 \\
          restr & \alert<4->{\<\[ \asort{fictional-psoa-op} psoa-arg & \@2 \]\>}
        \]\\
      \]
    \end{avm}
  }
\end{frame}


\section{PP-Modifkation in HPSG}

\begin{frame}
  {Lexikoneintrag einer NP-modifizierenden Präposition}
  \onslide<+->
  \onslide<+->
  Beispiel: \textit{ein Buch \alert{auf} dem Tisch}\\
  \onslide<+->
  \Halbzeile
  \begin{minipage}{0.45\textwidth}
    \scalebox{0.575}{%
      \begin{avm}
        \[ \asort{word} 
          phon & \<\textnormal{\it auf}\> \\
          cat &  \[ 
            head & \[ \asort{prep} 
              \alert<6>{mod} & \alert<6>{\[
                cat & \[
                  head & \[ \asort{n} \] \\
                  subcat & \< \[ cat|hd & \[ \asort{det} \] \]\> \\
                \]\\
                cont & \[
                  \alert<9>{ind} & \alert<9>{\@1} \\
                  \alert<7>{restr} & \alert<7>{\@5} \\
                \] \\
              \]} \\
            \] \\
            \alert<5>{subcat} & \alert<5>{\< \[ 
              cat & \[
                head & \[ \asort{n} cas & dat \]\\
                subcat & \<\> \\
              \] \\
              cont & \[
                \alert<9>{ind} & \alert<9>{\@2} \\
                \alert<7>{restr} & \alert<7>{\@3} \\
              \]
            \]\>} \\
          \]\\
          cont & \[ 
            ind & \@2 \\
            \alert<8>{restr} & \alert<8>{\< \[ \asort{ontop-loc-rel} 
              \alert<9>{pos-obj} & \alert<9>{\@1} \\
              \alert<9>{ref-obj} & \alert<9>{\@2} \\
            \] \>}\ \alert<7>{$\oplus$\ \@3\ $\oplus$\ \@5} \\
          \]\\
        \]
      \end{avm}%
    }
  \end{minipage}~\onslide<+->%
  \begin{minipage}{0.525\textwidth}
    Die vielen Aufgaben einer Präposition\\
    \Halbzeile
    \begin{itemize}[<+->]\small
      \item Die Präposition \alert<5>{regiert eine NP als ihr Komplement in einem bestimmten Kasus}.
      \item Außerdem \alert<6>{möchte sie ein N$'$ modifizieren}.
      \item Sie \alert<7>{sammelt die \textsc{restr} von Komplement und Modifikatum auf}.
      \item Sie \alert<8>{führt eine lokale Relation ein}.
      \item Die Relation besteht \alert<9>{zwischen den Objekten, die vom Komplement und Modifikatum eingeführt werden}.
    \end{itemize}
  \end{minipage}
\end{frame}

\begin{frame}
  {Kombination der Präposition mit ihrem Komplement}
  \onslide<+->
  \onslide<+->
  Diese beiden \textit{signs} können eine \textit{hd-arg-phr} bilden.\\
  \grau{\footnotesize Wir teilen ein Struktur in der Darstellung auf: \mybox{100} deutet die potenzielle Phrasenbildung an.}\\
  \onslide<+->
  \centering 
    \scalebox{0.5}{%
      \begin{avm}
        \[ \asort{word} 
          phon & \<\textnormal{\it auf}\> \\
          cat &  \[ 
            head & \[ \asort{prep} 
              mod & \[
                cat & \[
                  head & \[ \asort{n} \] \\
                  subcat & \< \[ cat|hd & \[ \asort{det} \] \]\> \\
                \]\\
                cont & \[
                  ind & \@1 \\
                  restr & \@5 \\
                \] \\
              \] \\
            \] \\
            subcat & \< \@{100} \[ 
              cat & \[
                head & \[ \asort{n} cas & dat \]\\
                subcat & \<\> \\
              \] \\
              cont & \[
                ind & \@2 \\
                restr & \@3 \\
              \]
            \]\> \\
          \]\\
          cont & \[ 
            ind & \@2 \\
            restr & \< \[ \asort{ontop-loc-rel} 
              pos-obj & \@1 \\
              ref-obj & \@2 \\
            \] \>\ $\oplus$\ \@3\ $\oplus$\ \@5 \\
          \]\\
        \]
      \end{avm}%
    }\hspace{2em}%
    \visible<4->{\raisebox{-3\baselineskip}{\scalebox{0.5}{%
      \begin{avm}
        \@{100} \[
          \asort{hd-arg-phr} 
          phon & \<\textnormal{\it dem,Tisch}\> \\
          cat & \[
            hd & \[ \asort{n}
              cas & dat \\
              mod & none \\
            \]\\
            subcat & \<\> \\
          \]\\
          cont & \[
            ind & \@2 \[ per & 3 \\ num & sg \\ gen & masc \]\\
            restr & \@3 \<\[ \asort{table-rel} instance & \@2 \]\> \\
          \]\\
        \]
      \end{avm}
    }}}
\end{frame}

\begin{frame}
  {Kombination der attributiven PP mit dem Kopf-N$'$}
  \centering 
    \visible<2->{\raisebox{-0\baselineskip}{\scalebox{0.5}{%
      \begin{avm}
        \[ \asort{hd-adj-phr}
          phon & \<\textnormal{\it Buch,auf,dem,Tisch}\> \\
          cat & \[
            head & \@9 \[ \asort{n} \] \\
            subcat & \@{10} \< \[ cat|hd & \[ \asort{det} \] \]\> \\
          \]\\
          cont & \@7 \\
          hd-dtr &\@{200}  \[
            phon & \<\textnormal{\it Buch}\> \\
            cat & \[ 
              head & \@9 \\
              subcat & \@{10} \\
            \] \\
            cont & \[
              ind & \@1 \[ per & 3 \\ num & sg  \\ gen & neut \] \\
              restr & \@5 \< \[ \asort{book-rel} instance & \@1 \] \> \\
            \]
          \]\\
          nhd-dtr & \@{300} \\
        \]
      \end{avm}%
    }}}\hspace{0.5em}%
    \visible<3->{\raisebox{-0.25\baselineskip}{\scalebox{0.5}{%
      \begin{avm}
        \@{300} \[ \asort{hd-arg-phr} 
          phon & \<\textnormal{\it auf,dem,Tisch}\> \\
          cat &  \[ 
            head & \@6 \[ \asort{prep} 
              mod & \@{200}
            \] \\
            subcat & \<\> \\
          \]\\
          cont & \@7 \[ 
            ind & \@2 \\
            restr & \< \[ \asort{ontop-loc-rel} 
              pos-obj & \@1 \\
              ref-obj & \@2 \\
            \] \>\ $\oplus$\ \@3\ $\oplus$\ \@5 \\
          \]\\
          hd-dtr & \[
            phon & \< \textnormal{\it auf} \> \\
            cat & \[
              head & \@6 \\
              subcat & \< \@{100} \> \\
            \] \\
            cont & \@7 \\
          \]\\
          nhd-dtr & \@{100} \\
        \]
      \end{avm}%
    }}}\hspace{0.5em}%
    \visible<4>{\raisebox{-1.35\baselineskip}{\scalebox{0.5}{%
      \begin{avm}
        \@{100} \[ \asort{hd-arg-phr} 
        phon & \<\textnormal{\it dem,Tisch}\> \\
        cat & \[
          hd & \[ \asort{n}
            cas & dat \\
            mod & none \\
          \]\\
          subcat & \<\> \\
        \]\\
        cont & \[
          ind & \@2 \[ per & 3 \\ num & sg \\ gen & masc \]\\
          restr & \@3 \<\[ \asort{table-rel} instance & \@2 \]\> \\
        \]\\
        \]
      \end{avm}
    }}}
\end{frame}

\begin{frame}
  {\textsc{mod} als \textsc{head}-Merkmal}
  \onslide<+->
  \onslide<+->
  \textsc{mod} muss ein \textsc{head}-Merkmal sein\\
  \Zeile
  \begin{itemize}[<+->]
    \item Die Präposition ist \alert{lexikalisch für ihr \textsc{mod} spezifiziert}.
    \item Sie \alert{bildet aber zunächst eine Phrase} (PP) mit einem Komplement (NP).
    \item Die \alert{volle PP modifiziert} dann das N$'$.
    \item Die \textsc{mod}-Spezifikation muss also \alert{an der PP} realisiert werden.
    \item \alert{Die \textsc{head}-Merkmale werden sowieso unverändert von P an PP weitergegeben.}
    \item \grau{Sonst bräuchten wir zusätzliche Mechanismen, um \textsc{mod} an der PP zu realisieren.}
      \Halbzeile
    \item Paralleles gilt für attributive NPs oder Relativsätze.
  \end{itemize}
\end{frame}

\section{Specifier in HPSG}

\begin{frame}
  {Lexikoneintrag eines Possessivartikels}
  \onslide<+->
  \onslide<+->
  Das Nomen bleibt der Kopf, aber \alert{der Spezifikator muss dessen Index erreichen}.\\
  \onslide<+->
  \Zeile
  \begin{minipage}{0.45\textwidth}
    \scalebox{0.8}{\begin{avm}
      \[ \asort{word}
        phon & \<\textnormal{\it sein} \> \\
        cat & \[ 
          head & \[ \asort{det} 
            \alert{spec} & \alert{\[
              cat|head & \[ \asort{n} \] \\
              cont|ind & \@1 \\
            \]} \\
          \]\\
          subcat & \<\> \\
        \]\\
        cont & \[
          ind & \@2 \[ per & 3 \\ num & sg \\ gen & m $\vee$ n \] \\
          restr & \< \[ \asort{possess-rel} possessor-obj & \@2 \\ \alert<5>{possessed-obj} & \alert{\@1} \]\> \\
        \]\\
      \]
    \end{avm}}
  \end{minipage}~\onslide<+->%
  \begin{minipage}{0.5\textwidth}
    \alert{Spezifikator-Prinzip}\\
    \Halbzeile 
    \begin{itemize}[<+->]
      \item Wenn eine Nicht-Kopf-Tochter für \textsc{cat|head|spec} nicht \textit{none} ist, \ldots
      \item \ldots\ ist der Wert ihres \textsc{spec}-Merkmals token-identisch zur Kopftochter.\\
        \grau{\footnotesize Erinnerung: token-identisch = strukturgeteilt}
    \end{itemize}
  \end{minipage}
\end{frame}

\begin{frame}
  {Kombination des Possessivartikels mit N$'$}
  \onslide<+->
  \onslide<+->
  \centering 
  \raisebox{-2\baselineskip}{\scalebox{0.55}{%
    \begin{avm}
      \[ \asort{ hd-arg-phr}
        phon & \<\textnormal{\it sein,Buch}\> \\
        cat & \[ 
          head & \@6 \\
          subcat \<\> \\
        \]\\
        cont & \@5 \\
        hd-dtr & \@{100} \\
        nhd-dtr & \@{200} \\
      \]
    \end{avm}
  }}\hspace{1em}%
  \visible<3->{%
    \raisebox{0\baselineskip}{\scalebox{0.55}{%
      \begin{avm}
        \@{200} \[ \asort{word} 
          phon & \<\textnormal{\it sein} \> \\
          cat & \[ 
            head & \[ \asort{det} 
              spec & \@{100}\[
                cat|head & \[ \asort{n} \] \\
                cont|ind & \@1 \\
              \] \\
            \]\\
            subcat & \<\> \\
          \]\\
          cont & \[
            ind & \@4 \[ per & 3 \\ num & sg \\ gen & m $\vee$ n \] \\
            restr & \< \[ \asort{possess-rel} possessor-obj & \@4 \\ possessed-obj & \@1 \]\> \\
          \]\\
        \]
      \end{avm}
    }}
  }\hspace{1em}%
  \visible<4->{%
    \raisebox{-0.25\baselineskip}{\scalebox{0.55}{%
      \begin{avm}
        \@{100} \[ \asort{word}
          phon & \<\textnormal{\it Buch}\> \\
          cat & \[
            head & \@6 \[ \asort{n} \] \\
            subcat & \<\@{200}\[
              cat & \[
                head & \[ \asort{det} \] \\
                subcat & \<\> \\
              \]
            \]\>
          \] \\
          cont & \@5 \[ 
            ind & \@1 \[ per & 3 \\ num & sg \\ gen & n \]\\
            restr & \< \[ \asort{book-rel} instance & \@1 \]\> \\
          \]
        \]
      \end{avm}
    }}
  }
\end{frame}

\section{Zusammenfassung der Grammatik}

\begin{frame}
  {Kopf-Adjunkt-Schema}
  \onslide<+->
  \onslide<+->
  \centering 
  \raisebox{-1.25\baselineskip}{\textit{head-adjunct-phrase} $\Rightarrow$} \scalebox{0.8}{%
    \begin{avm}
      \[ head-dtr & \@1 \\
        non-hd-dtr & \< \[ cat & \[ head|mod & \@1 \\
          subcat & \<\>
        \]
      \]\> 
      \]
    \end{avm}
  }\\
  \Zeile
  \onslide<+->
  In Kopf-Adjunkt-Strukturen ist das \textsc{mod}-Merkmal des Nicht-Kopfs\\
  token-identisch mit der Kopftochter.\\
  \onslide<+->
  \Halbzeile
  \grau{So selegiert das Adjunkt seinen Kopf und kann dessen Semantik modifizieren.}
\end{frame}

\begin{frame}
  {Semantikprinzip (zweiteilig)}
  \onslide<+->
  \onslide<+->
  \centering 
    \raisebox{-0.75\baselineskip}{\textit{head-non-adjunct-phrase} $\Rightarrow$} \scalebox{0.7}{%
      \begin{avm}
        \[ cont & \@1 \\
           hd-dtr & \[ 
             cont & \@1
        \] 
        \]
    \end{avm}}\\
    \onslide<+->
    \Zeile
    \raisebox{-0.75\baselineskip}{\textit{head-adjunct-phrase} $\Rightarrow$} \scalebox{0.7}{%
      \begin{avm}
        \[ cont & \@1 \\
           non-hd-dtr & \[ 
             cont & \@1
        \] 
        \]
    \end{avm}}\\
    \onslide<+->
    \Zeile
    In Kopf-Nichtadjunkt-Strukturen wird die Semantik des Kopfs an der Phrase repräsentiert, in Kopf-Adjunkt-Strukturen die Semantik des Nicht-Kopfs (Adjunkts).\\
    \onslide<+->
    \Halbzeile
    \grau{Das erlaubt dem Adjunkt die Kontrolle über die Semantik der Phrase.}
\end{frame}

\begin{frame}
  {Subkategorisierungsprinzip}
  \onslide<+->
  \onslide<+->
  \centering 
    \raisebox{-0.5\baselineskip}{\textit{head-argument-phrase} $\Rightarrow$} \scalebox{0.7}{%
      \begin{avm}
        \[ cat|subcat & \@1 \\
          hd-dtr|cat|subcat & \@1 $\oplus$ \<\@2\> \\
          nhd-dtr & \@2 \\
        \]
    \end{avm}} \\
    \onslide<+->
    \Zeile
    \raisebox{-0.5\baselineskip}{\textit{head-non-argument-phrase} $\Rightarrow$} \scalebox{0.7}{%
      \begin{avm}
        \[ cat|subcat & \@1 \\
          hd-dtr|cat|subcat & \@1 \\
        \]
    \end{avm}} \\
    \onslide<+->
    \Zeile
    In einer Kopf-Argument-Struktur ist das letzte Element der \textsc{subcat} des Kopfs\\
    token-identisch zur Nicht-Kopf-Tochter und die \textsc{subcat} der Phrase\\
    ist die \textsc{subcat} der Kopftochter ohne deren letztes Element.\\
    \Halbzeile
    \onslide<+->
    In Kopf-Nichtargument-Strukturen ist die \textsc{subcat} des Kopfs an der Phrase repräsentiert.
\end{frame}

\begin{frame}
  {Specifier-Prinzip}
  \onslide<+->
  \onslide<+->
  \centering 
  Falls eine Nicht-Kopf-Tochter in einer Kopf-Struktur einen Wert für \textsc{spec} anders als \textit{none} hat, ist dieser token-identisch zur Kopftochter.\\
  \onslide<+->
  \Halbzeile
  \grau{Das erlaubt es pränominalen Possessiva, auf die Semantik des N$'$ zuzugreifen.}
\end{frame}

\begin{frame}
  {Typenhierachie (Ausschnitt)}
  \onslide<+->
  \onslide<+->
  \centering 
  \begin{forest}
    typehierarchy,
    for tree={
      calign=fixed angles,
      calign angle=60
    } 
    [sign
      [word]
      [phrase
        [non-headed-phrase]
        [headed-phrase, for tree={l sep+=\baselineskip}
          [head-non-adjunct-phrase, calign=first
            [head-argument-phrase]
            [\ldots]
          ]
          [head-non-argument-phrase, calign=last
            [,identify=!r2212]
            [head-adjunct-phrase]
          ]
        ]
      ]
    ]
  \end{forest}
\end{frame}

\begin{frame}
  {Bemerkungen zur Grammatik}
  \onslide<+->
  \onslide<+->
  Die Grammatik im engeren Sinn (Kombinatorik) ist damit weitgehend beschrieben.\\
  \Zeile
  \begin{itemize}[<+->]
    \item Sie merken: Die meiste Arbeit leistet das Lexikon.
    \item Das Lexikon besprechen wir nächste Woche, und dabei kommen noch Regeln hinzu.
    \item Es ist wichtig, die wenigen echten Regeln zu verinnerlichen.
      \Halbzeile
    \item "`Phrase"' bedeutet in HPSG zunächst mal "`komplexes Zeichen"'.
    \item Die "`Phrase"' traditioneller Ansätze ist eine \textsc{subcat}-empty Struktur mit Kopf.
  \end{itemize}
\end{frame}

\section{Nächste Woche}

\begin{frame}
  {Vorbereitung}
  \centering 
  \large
  Nächste Woche reden wir über das Lexikon und Lexikonregeln.\\
  \Zeile
  \rot{Sie sollten dringend vorher aus dem HPSG-Buch\\
  von Kapitel 7 die Seiten 91--98 lesen!}\\
  \Halbzeile 
  Das sind gerade mal \gruen{7} Seiten.\\
  \Halbzeile
  Ein zusätzlicher Blick in Kapitel~19 kann nicht schaden.\\
  \grau{\footnotesize Achtung: Das ist etwas anspruchsvoller und setzt noch mehr Syntaxwissen voraus.}\\
  \Zeile
  \rot{Achtung! In der Woche darauf sind die Seiten~129--148 dran.\\Das ist mehr als sonst. Lesen Sie ggf.\ im Voraus!}
\end{frame}
