\section{Einleitung}

\begin{frame}
  {Adjunkte und Spezifikatoren in HPSG}
  \onslide<+->
  \onslide<+->
  Kopf-Adjunkt-Phrasen und Kopf-Determinierer-Konstruktionen\\
  \Zeile
  \begin{itemize}[<+->]
    \item Was ist Modifikation?
    \item Intersektive und nicht-intersektive Adjektive
    \item Modifizierende PPs
    \item Wozu braucht man ein gesondertes Spezifikatorprinzip?
    \item Genitivattribute
  \end{itemize}
  \Zeile
  \onslide<+->
  \centering 
  \grau{\citet[Kapitel~6]{MuellerLehrbuch3}}\\
\end{frame}

\section{Adjunkte}

\begin{frame}
  {Adjektive}
  \onslide<+->
  \onslide<+->
  Adjektive sind mehr als \alert{Eigenschaftswörter}
\end{frame}

\begin{frame}
  {Nicht-intersektive Modifikation}
  \onslide<+->
  \onslide<+->
  Manche Adjektive modifizieren die Bedeutung (Intension) von Adjektiven.
\end{frame}

\begin{frame}
  {Präspositionen als NP-Modifikatoren}
  \onslide<+->
  \onslide<+->
  Doppelter semantischer und syntaktischer Bezug
\end{frame}


\section{Modifikation in HPSG}

\begin{frame}
  {Lexikoneintrag eines Adjektivs}
  \onslide<+->
  \onslide<+->
  Einführung einer \textsc{restr}
\end{frame}

\begin{frame}
  {Lexikoneintrag einer Präposition}
  \onslide<+->
  \onslide<+->
  Präpositionen, die NP-modifizierende PPs bilden
\end{frame}

\begin{frame}
  {Kopf-Adjunkt-Schema}
  \onslide<+->
  \onslide<+->
  Ein neuer Phrasentyp
\end{frame}

\begin{frame}
  {Semantikprinzip für Kopf-Adjunkt-Phrasen}
  \onslide<+->
  \onslide<+->
  Die Adjunkttochter bestimmt die Semantik!
\end{frame}

\begin{frame}
  {Subkategorisierungsprinzip}
  \onslide<+->
  \onslide<+->
  Kopf-Adjunkt-Phrasen müssen die \textsc{subcat} des Kopfs weitergeben.
\end{frame}

\begin{frame}
  {Typhierarchie}
  \onslide<+->
  \onslide<+->
  Der Typ \textit{sign} und einige seiner Untertypen:
\end{frame}

\begin{frame}
  {Beispiel 1}
  \onslide<+->
  \onslide<+->
  Intersektive Adjektivmodifikation
\end{frame}

\begin{frame}
  {Beispiel 2}
  \onslide<+->
  \onslide<+->
  Nicht-intersektive Adjektivmodifikation
\end{frame}

\begin{frame}
  {Beispiel 3}
  \onslide<+->
  \onslide<+->
  PP als Attribut in einer NP
\end{frame}

\begin{frame}
  {Beispiel 4}
  \onslide<+->
  \onslide<+->
  PP als Adjunkt in der VP
\end{frame}

\section{Spezifikation}

\begin{frame}
  {Spezifikatoren in HPSG}
  \onslide<+->
  \onslide<+->
  Genitiv-Attribute und ihre Semantik
\end{frame}

\begin{frame}
  {\textsc{spec}-Prinzip}
  \onslide<+->
  \onslide<+->
  Nur eine zusätzliche Strukturteilung
\end{frame}

\begin{frame}
  {Beispiel}
  \onslide<+->
  \onslide<+->
  NP mit Genitivattribut
\end{frame}

\section{Nächste Woche}

\begin{frame}
  {Vorbereitung}
  \onslide<+->
  \onslide<+->
  \centering 
  \large
  Nächste Woche reden wir über das Lexikon und Lexikonregeln.\\
  \onslide<+->
  \Zeile
  \rot{Sie sollten dringend vorher aus dem HPSG-Buch\\
  von Kapitel 7 die Seiten 91--98 lesen!}\\
  \onslide<+->
  \Viertelzeile
  Das sind gerade mal \gruen{7} Seiten.\\
  \onslide<+->
  \Zeile
  \rot{Achtung! In der Woche darauf sind die Seiten~129--148 dran.\\Das ist mehr als sonst. Lesen Sie ggf.\ im Voraus!}
\end{frame}
