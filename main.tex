\def\GRAPHPATH{localgraphics}

\ifdefined\HANDOUT
  \documentclass[handout,aspectratio=1610,dvipsnames]{beamer}
  \def\GRAPHPATH{graphics}
\else
  \documentclass[aspectratio=1610,dvipsnames]{beamer}
\fi

\usepackage[ngerman]{babel}
\usepackage{ifthen}
\usepackage{color}
\usepackage{colortbl}
\usepackage{textcomp}
\usepackage{multirow}
\usepackage{nicefrac}
\usepackage{multicol}
\usepackage{langsci-gb4e}
\usepackage{verbatim}
\usepackage{cancel}
\usepackage{graphicx}
\usepackage{hyperref}
\usepackage{verbatim}
\usepackage{boxedminipage}
\usepackage{adjustbox}
\usepackage{rotating}
\usepackage{booktabs}
\usepackage{bbding}
\usepackage{pifont}
\usepackage{multicol}
\usepackage{stmaryrd}
\usepackage{FiraSans}
\usepackage{soul}
\usepackage{tikz}
\usepackage{array}
\usepackage{xstring}
\usepackage{avm}

\usetikzlibrary{calc,decorations.pathmorphing,tikzmark,positioning,chains,trees,graphs,shapes,shadows,arrows}

\renewcommand\tikzmark[2]{%
  \tikz[remember picture,baseline=(chain-1.base),start chain] \node[on chain,inner sep=2pt,outer sep=0] (#1){#2};%
}

\newcommand\link[2]{%
  \begin{tikzpicture}[remember picture, overlay, >=stealth, shift={(0,0)}]
    \draw[-] (#1) --++(0,-12pt) -| (#2);
   \end{tikzpicture}%
}


\makeatletter
\g@addto@macro{\endtabular}{\rowfont{}}% Clear row font
\makeatother
\newcommand{\rowfonttype}{}% Current row font
\newcommand{\rowfont}[1]{% Set current row font
   \gdef\rowfonttype{#1}#1%
}
\newcolumntype{L}{>{\rowfonttype}l}

\usepackage{tikz}
\usepackage{tikz-qtree}
\usepackage{forest}
\useforestlibrary{linguistics}
\forestapplylibrarydefaults{linguistics}



\usepackage[maxbibnames=99,
  maxcitenames=2,
  uniquelist=false,
  backend=biber,
  doi=false,
  url=false,
  isbn=false,
  bibstyle=biblatex-sp-unified,
  citestyle=sp-authoryear-comp]{biblatex}

% Biblatex ============================================================

%\addbibresource{rs.bib}
\addbibresource{../biblio.bib}

% Colors ==============================================================

% \ifdefined\HANDOUT
  \definecolor{grau}{rgb}{0.5,0.5,0.5}
  \definecolor{lg}{rgb}{0.8,0.8,0.8}
  \definecolor{trueblue}{rgb}{0.3,0.3,1}
  \definecolor{ltb}{rgb}{0.8,0.8,1}
  \definecolor{lgr}{rgb}{0.5,1,0.5}
  \definecolor{orongsch}{RGB}{255,165,0}
  \definecolor{gruen}{rgb}{0,0.4,0}
  \definecolor{rot}{rgb}{0.7,0.2,0.0}
  \definecolor{tuerkis}{RGB}{63,136,143}
  \definecolor{braun}{RGB}{108,71,65}
  \definecolor{blaw}{rgb}{0,0,0.9}
% \else
%   \definecolor{grau}{rgb}{0.7,0.7,0.7}
%   \definecolor{lg}{rgb}{0.9,0.9,0.9}
%   \definecolor{trueblue}{rgb}{0.8,0.8,1}
%   \definecolor{ltb}{rgb}{0.9,0.9,1}
%   \definecolor{lgr}{rgb}{0.7,1,0.7}
%   \definecolor{orongsch}{RGB}{255,200,100}
%   \definecolor{gruen}{RGB}{0,230,0}
%   \definecolor{rot}{RGB}{255,155,100}
%   \definecolor{tuerkis}{RGB}{150,205,205}
%   \definecolor{braun}{RGB}{140,120,115}
%   \definecolor{blaw}{rgb}{0,0,0.9}
% \fi

\newcommand{\gruen}[1]{\textcolor{gruen}{#1}}
\newcommand{\blaw}[1]{\textcolor{blaw}{#1}}
\newcommand{\rot}[1]{\textcolor{rot}{#1}}
\newcommand{\blau}[1]{\textcolor{trueblue}{#1}}
\newcommand{\orongsch}[1]{\textcolor{orongsch}{#1}}
\newcommand{\grau}[1]{\textcolor{grau}{#1}}
\newcommand{\whyte}[1]{\textcolor{white}{#1}}
\newcommand{\tuerkis}[1]{\textcolor{tuerkis}{#1}}
\newcommand{\braun}[1]{\textcolor{braun}{#1}}

% Newcommands =========================================================

\newcommand{\Dim}{\cellcolor{lg}}
\newcommand{\Dimblue}{\cellcolor{ltb}}
\newcommand{\Dimgreen}{\cellcolor{lgr}}
\newcommand{\Sub}[1]{\ensuremath{_{\text{#1}}}}
\newcommand{\Up}[1]{\ensuremath{^{\text{#1}}}}
\newcommand{\UpSub}[2]{\ensuremath{^{\text{#1}}_{\text{#2}}}}
\newcommand{\Spur}[1]{t\Sub{#1}}
\newcommand{\Ti}{\Spur{1}}
\newcommand{\Tii}{\Spur{2}}
\newcommand{\Tiii}{\Spur{3}}
\newcommand{\Tiv}{\Spur{4}}
\newcommand{\Ck}{\CheckmarkBold}
\newcommand{\Fl}{\XSolidBrush}
\newcommand{\xxx}{\hspaceThis{[}}
\newcommand{\zB}{z.\,B.\ }
\newcommand{\down}[1]{\ensuremath{\mathrm{#1}}}
\newcommand{\Doppelzeile}{\vspace{2\baselineskip}}
\newcommand{\Zeile}{\vspace{\baselineskip}}
\newcommand{\Halbzeile}{\vspace{0.5\baselineskip}}
\newcommand{\Viertelzeile}{\vspace{0.25\baselineskip}}
\newcommand{\KTArr}[1]{\ding{222}~\fbox{#1}~\ding{222}}
\newcommand{\Ast}{*}
\newcommand{\SL}{\ensuremath{\llbracket}}
\newcommand{\SR}{\ensuremath{\rrbracket}}
\def\lspbottomrule{\bottomrule}
\def\lsptoprule{\toprule}
\newcommand{\Sw}[1]{\begin{sideways}#1\end{sideways}}
\newcommand{\Lab}{\ensuremath{\langle}}
\newcommand{\Rab}{\ensuremath{\rangle}}
\newcommand{\AbUmlautBreaker}{}
\ifdefined\HANDOUT
  \renewcommand{\AbUmlautBreaker}{\ /}
\fi
\newcommand{\LocStrutGrph}{\hspace{0.1\textwidth}}
\newcommand{\Nono}{---}

\newcommand{\Bewegtes}[1]{\ensuremath{_{\textrm{#1}}}}
\newcommand{\ORi}{\Bewegtes{1}}
\newcommand{\ORii}{\Bewegtes{2}}
\newcommand{\ORiii}{\Bewegtes{3}}
\newcommand{\ORiv}{\Bewegtes{4}}
\newcommand{\ORv}{\Bewegtes{5}}
\newcommand{\goesto}{→~}

% Beamer ==============================================================

\usetheme[hideothersubsections]{Boadilla}

% \ifdefined\HANDOUT
  \usecolortheme{whale}
% \else
%   \usecolortheme{magpie}
% \fi

\renewcommand<>{\rot}[1]{%
  \alt#2{\beameroriginal{\rot}{#1}}{#1}%
}
\renewcommand<>{\blau}[1]{%
  \alt#2{\beameroriginal{\blau}{#1}}{#1}%
}
\renewcommand<>{\orongsch}[1]{%
  \alt#2{\beameroriginal{\orongsch}{#1}}{#1}%
}
\renewcommand<>{\gruen}[1]{%
  \alt#2{\beameroriginal{\gruen}{#1}}{#1}%
}

\setbeamercolor{alerted text}{fg=trueblue}

\newcounter{lastpagemainpart}

\resetcounteronoverlays{exx}

\AtBeginSection[]{
  \begingroup
  \setbeamertemplate{navigation symbols}{}
  \begin{frame}[noframenumbering,plain]
  \vfill
  \centering
  \begin{beamercolorbox}[sep=8pt,center,shadow=true,rounded=true]{title}
    \usebeamerfont{title}\insertsectionhead\par%
  \end{beamercolorbox}
  \vfill
  \end{frame}
  \endgroup
}

\setbeamertemplate{itemize item}[circle]
\setbeamertemplate{enumerate item}[square]


\makeatother
\setbeamertemplate{footline}
{
  \leavevmode%
  \hbox{%
  \begin{beamercolorbox}[wd=.4\paperwidth,ht=2.25ex,dp=1ex,center]{author in head/foot}%
    \usebeamerfont{author in head/foot}\insertshortauthor
  \end{beamercolorbox}%
  \begin{beamercolorbox}[wd=.6\paperwidth,ht=2.25ex,dp=1ex,center]{title in head/foot}%
    \usebeamerfont{title in head/foot}\insertshorttitle\hspace*{3em}
    \insertframenumber{} / \inserttotalframenumber\hspace*{1ex}
  \end{beamercolorbox}}%
  \vskip0pt%
}
\makeatletter
\setbeamertemplate{navigation symbols}{}


% Tikz ================================================================

\usetikzlibrary{positioning,arrows,cd}
\tikzset{>=latex}

% Forest

\forestset{
  Ephr/.style={draw, ellipse, thick, inner sep=2pt},
  Eobl/.style={draw, rounded corners, inner sep=5pt},
  Eopt/.style={draw, rounded corners, densely dashed, inner sep=5pt},
  Erec/.style={draw, rounded corners, double, inner sep=5pt},
  Eoptrec/.style={draw, rounded corners, densely dashed, double, inner sep=5pt},
  Ehd/.style={rounded corners, fill=gray, inner sep=5pt,
    delay={content=\whyte{##1}}
  },
  Emult/.style={for children={no edge}, for tree={l sep=0pt}},
  phrasenschema/.style={for tree={l sep=2em, s sep=2em}},
  decide/.style={draw, chamfered rectangle, inner sep=2pt},
  finall/.style={rounded corners, fill=gray, text=white},
  intrme/.style={draw, rounded corners},
  yes/.style={edge label={node[near end, above, sloped, font=\scriptsize]{Ja}}},
  no/.style={edge label={node[near end, above, sloped, font=\scriptsize]{Nein}}},
  sake/.style={tier=preterminal},
  ake/.style={
    tier=preterminal
    },
}

\tikzset{
    invisible/.style={opacity=0,text opacity=0},
    visible on/.style={alt=#1{}{invisible}},
    alt/.code args={<#1>#2#3}{%
      \alt<#1>{\pgfkeysalso{#2}}{\pgfkeysalso{#3}}
    },
}

\forestset{
  visible on/.style={
    for tree={
      /tikz/visible on={#1},
      edge+={/tikz/visible on={#1}}}}}

\useforestlibrary{edges}

\forestset{
  narroof/.style={roof, inner xsep=-0.25em, rounded corners},
  forky/.style={forked edge, fork sep-=7.5pt},
  bluetree/.style={for tree={trueblue}, for descendants={edge=trueblue}},
  orongschtree/.style={for tree={orongsch}, for children={edge=orongsch}},
  rottree/.style={for tree={rot}, for children={edge=rot}},
  gruentree/.style={for tree={gruen}, for children={edge=gruen}},
  tuerkistree/.style={for tree={tuerkis}, for children={edge=tuerkis}},
  brauntree/.style={for tree={braun}, for children={edge=braun}},
  grautree/.style={for tree={grau}, for children={edge=grau}}, 
  whitetree/.style={for tree={white}, for descendants={edge=white}}, 
  blacktree/.style={for tree={black}, for descendants={edge=black}}, 
  gruennode/.style={gruen, edge=gruen},
  graunode/.style={grau, edge=grau},
}

\forestset{
  invisible/.style={opacity=0,text opacity=0},
  visible on/.style={alt={#1{}{invisible}}},
  alt/.code args={<#1>#2#3}{%
    \alt<#1>{\pgfkeysalso{#2}}{\pgfkeysalso{#3}}
  },
  highlight on/.style={{alt=#1{highlight}{}}},
  white on/.style={{alt=#1{whitetree}{}}},
  black on/.style={{alt=#1{blacktree}{}}},
  blue on/.style={{alt=#1{bluetree}{}}},
}



% Drawing sonority diagrams =========================================== 

\makeatletter

\long\def\ifnodedefined#1#2#3{%
  \@ifundefined{pgf@sh@ns@#1}{#3}{#2}}

\newcommand\aeundefinenode[1]{%
  \expandafter\ifx\csname pgf@sh@ns@#1\endcsname\relax
  \else
    \typeout{Undefining node "#1"}%
    \global\expandafter\let\csname pgf@sh@ns@#1\endcsname\relax
  \fi
}

\newcommand\aeundefinethesenodes[1]{%
  \foreach \myn  in {#1}
    {%
      \ifnodedefined{\myn}{%
      \expandafter\aeundefinenode\expandafter{\myn}%
    }{}
    }%
}

\newcommand\aeundefinenumericnodes{%
  \foreach \myn in {1,2,...,50}
    {%
      \ifnodedefined{\myn}{%
      \expandafter\aeundefinenode\expandafter{\myn}%
    }{}
    }%
}
\makeatother

\newcommand{\plo}{0}
\newcommand{\fri}{0.5}
\newcommand{\nas}{1}
\newcommand{\liq}{1.5}
\newcommand{\vok}{2}

% Save text.
\newcommand{\lastsaved}{}
\newcommand{\textsave}[1]{\gdef\lastsaved{#1}#1}

\newcommand{\SonDiag}[2][0]{%
  \begin{tikzpicture}
    \textsave{.}
    \tikzset{
      normalseg/.style={fill=white},
      extrasyll/.style={circle, draw, fill=white},
      sylljoint/.style={diamond, draw, fill=white}
    }
    \node at (0,\plo) {P};
    \node at (0,\fri) {F};
    \node at (0,\nas) {N};
    \node at (0,\liq) {L};
    \node at (0,\vok) {V};

    % Draw the helper lines if required.
    \ifthenelse{\equal{#1}{0}}{}{%
      \foreach \y in {\plo, \fri, \nas, \liq,\vok} {%
	\draw [dotted, |-|] (0.25, \y) -- (#1.75, \y);
      }
    }

    \foreach [count=\x from 1, remember=\x as \lastx] \p / \y / \g in #2 {
      \ifthenelse{\equal{\y}{-1}}{\textsave{.}}{%

	% Draw the node, either plain, as Silbenbgelenk, or as extrasyllabic.
        \ifthenelse{\equal{\g}{1}}{%
	  \node (\x) [sylljoint] at (\x, \y) {\p};
	}{%
	  \ifthenelse{\equal{\g}{2}}{%
	    \node (\x) [extrasyll] at (\x, \y) {\p};
	  }{%
	    \node (\x) [normalseg] at (\x, \y) {\p};
	  }
	}

	% Draw the connection unless the previous node was not or was empty.
	\ifthenelse{\NOT\equal{\lastsaved}{.}}{%
	  \draw [->] (\lastx) to (\x);
	}{}
	\textsave{1}
      }
    }
    \aeundefinenumericnodes
  \end{tikzpicture}
}


\setbeamertemplate{navigation symbols}{}
\setbeamertemplate{section in toc}[circle]
\setbeamertemplate{subsection in toc}[square]
\setbeamertemplate{subsubsection in toc}[triangle]

\setbeamerfont{section in toc}{size=\tiny}
\setbeamerfont{subsection in toc}{size=\tiny}
\setbeamerfont{subsubsection in toc}{size=\tiny}

\avmfont{\sc}
\avmsortfont{\it}
\avmvalfont{\it}


\ifdefined\TITLE
  \title[Formale Syntax | \StrSubstitute{\TITLE}{+}{ }]{Formale Syntax\\\StrSubstitute{\TITLE}{+}{ }}
\else
  \title[Formale Syntax]{Formale Syntax}
\fi

\author{Roland Schäfer}
\institute[FSU Jena]{Institut für Germanistische Sprachwissenschaft\\Friedrich-Schiller-Universität Jena}
\date[HPSG]{\grau{\scriptsize Stets aktuelle Fassungen: \url{https://github.com/rsling/VL-Formale-Syntax}\\
Basiert teilweise auf Folien von Stefan Müller: \url{https://hpsg.hu-berlin.de/~stefan/Lehre/S2021/hpsg.html}\\
Stefan trägt natürlich keinerlei Verantwortung für meine Fehler und Missverständnisse!}}

\begin{document}

\begingroup
  \setbeamertemplate{navigation symbols}{}
  \begin{frame}[noframenumbering,plain]
   \titlepage
  \end{frame}
\endgroup

\ifdefined\TITLE
  \input{includes/\TITLE}
\else
  \section[Grammatik]{Grammatik}
  \let\woopsi\section\let\section\subsection\let\subsection\subsubsection
  
\begin{frame}
  {Ziele}
  \onslide<+->
  \onslide<+->
  Worum geht es heute?\\
  \Zeile
  \begin{itemize}[<+->]
    \item Vermittlung grundlegender Vorstellungen über deutsche Syntax
    \item Vorstellung für die Daten, Zusammenhänge und Komplexität
    \item Einführung in Grundannahmen in der HPSG
    \item Befähigung zum Schreiben formaler Grammatiken
  \end{itemize}
  \Zeile
  \centering
  \onslide<+->
  \grau{\citet[Kapitel~1]{MuellerLehrbuch} bzw.\ \citet[Kapitel~1]{MuellerGTBuch}\\
    Englische Version des Grammatiktheoriebuches: \citet[Kapitel~1]{MuellerGT-Eng}}
\end{frame}



\section{Wozu (formale) Syntax?}

\begin{frame}
  {Wozu Syntax?}
  \onslide<+->
  \begin{itemize}[<+->]
    \item \alert{Zeichen} | Form-Bedeutungs-Paare \citep{Saussure16a-Fr}
    \item Wörter, Wortgruppen, Sätze
    \item Sprache | \rot{keine} (endliche) \rot{Aufzählung} von Wortfolgen\\
      \grau{Endlichkeit von Sprache bei Annahme einer maximalen Satzlänge}
      \begin{exe}
        \ex Dieser Satz geht weiter und weiter und weiter und weiter \ldots
        \ex {}[Ein Satz ist ein Satz] ist ein Satz.
      \end{exe}
      \Halbzeile
    \item Auf jeden Fall \alert{sehr viele Sätze}, Unendlichkeitsproblem als Scheinfrage
    \item \alert{Kompetenz} | (implizites) Wissen um grammatische Regularitäten
    \item \alert{Performanz} | Nutzung des Wissens, Sprachproduktion
      \Halbzeile
    \item \alert{Kreativität} | Sätze bilden, die man nie zuvor gehört hat
  \end{itemize}
\end{frame}
 
 
\begin{frame}
  {Die Kinder im Randaledorf (Astrid Lindgren)}
  \onslide<+->
  \onslide<+->
  Schon Kindern kann man ein Spiel um Kompetenz und Performanz zumuten!\\
  \Zeile
  \onslide<+->
  \begin{quote}
    Und wir beeilten uns, den Jungen zu erzählen, wir hätten von Anfang an gewußt, daß es nur eine
    Erfindung von Lasse gewesen sei. Und da sagte Lasse, die Jungen hätten gewußt, daß wir gewußt
    hätten, es sei nur eine Erfindung von ihm. Das war natürlich gelogen, aber vorsichtshalber sagten
    wir, wir hätten gewußt, die Jungen hätten gewußt, daß wir gewußt hätten, es sei nur eine Erfindung
    von Lasse. Und da sagten die Jungen -- ja -- jetzt schaffe ich es nicht mehr aufzuzählen, aber es
    waren so viele "`gewußt"', daß man ganz verwirrt davon werden konnte, wenn man es hörte.
  \end{quote}
  \Zeile
  \begin{itemize}[<+->]
    \item \alert{Grammatikalität} der Sätze | Einwandfrei feststellbar
    \item \alert{Akzeptabilität} der Sätze | Vermindert durch \rot{Performanzeffekte}
  \end{itemize}
\end{frame}

\begin{frame}
  {Wozu Syntax? Bedeutung aus Bestandteilen ermitteln}
  \onslide<+->
  \onslide<+->
  Bedeutung einer Äußerung aus den Bedeutungen ihrer Teile bestimmen\\
  \Viertelzeile
  \onslide<+->
  \begin{exe}
    \ex Der Mann kennt den Kollegen.
  \end{exe}
  \Halbzeile
  \onslide<+->
  \alert{Syntax} | Art und Weise der Kombination, Strukturierung\\
  \Viertelzeile
  \onslide<+->
  \begin{exe}
    \ex
    \begin{xlist}
      \ex Die Frau kennt die Kolleginnen.
      \ex Die Frau kennen die Kolleginnen.
    \end{xlist}
    \onslide<+->
    \ex
    \begin{xlist}
      \ex Die Frau schläft.
      \ex Die Kolleginnen schlafen.
    \end{xlist}
  \end{exe}
  \Halbzeile
  \onslide<+->
  \begin{block}
    {Das Frege-Prinzip (Gottlob Frege, 1879)}
    Die Bedeutung eines Satzes ergibt sich aus der Bedeutung seiner Konstituenten\\
    und der Art ihrer Kombination.
  \end{block}
\end{frame}
 

\begin{frame}
  {Warum formal?}
  \onslide<+->
  \onslide<+->
  \begin{quote}\small
    Precisely constructed models for linguistic structure can play an
    important role, both negative and positive, in the process of discovery 
    itself. By pushing a precise but inadequate formulation to
    an unacceptable conclusion, we can often \alert{expose the exact source
    of this inadequacy and, consequently, gain a deeper understanding}
    of the linguistic data. More positively, a formalized theory may 
    \alert{automatically provide solutions for many problems other than those
    for which it was explicitly designed}. Obscure and intuition-bound
    notions can neither lead to absurd conclusions nor provide new and
    correct ones, and hence they fail to be useful in two important respects. 
    I think that some of those linguists who have questioned
    the value of precise and technical development of linguistic theory
    have failed to recognize the productive potential in the method
    of rigorously stating a proposed theory and applying it strictly to
    linguistic material with no attempt to avoid unacceptable conclusions by ad hoc adjustments or loose formulation.
\citep[S.\,5]{Chomsky57a}
  \end{quote}   
  \onslide<+->
  \Halbzeile
  \begin{quote}\small
    As is frequently pointed out but cannot be overemphasized, an important goal
    of formalization in linguistics is to \alert{enable subsequent researchers to see the defects
    of an analysis as clearly as its merits}; only then can progress be made efficiently.
    \citep[S.\,322]{Dowty79a}
  \end{quote}
\end{frame}

\begin{frame}
  {Sie studieren Deutsch auf Lehramt?}
  \onslide<+->
  \onslide<+->
  \centering 
  \rot{Das bringt mir doch nichts für den Unterricht in der 5.~oder 10.~Klasse!}\\
  \onslide<+->
  \Zeile
  \Large Erste Antwortmöglichkeit:\\
  \onslide<+->
  \Halbzeile
  \alert{Seien Sie froh!} Sie können jetzt im pessimistischsten Fall\\
  zum letzten Mal vor der Rente etwas machen, das \\
  Ihr Gehirn weiterbringt und nicht an die Zwecke \\
  der Arbeit gebunden ist.\\
  \onslide<+->
  \Halbzeile
  \normalsize
  \gruen{Diese Antwort stimmt aber in unserem Fall nicht ganz …}
\end{frame}


\begin{frame}
  {Sie studieren Deutsch auf Lehramt?}
  \onslide<+->
  \onslide<+->
  Sie möchten den \alert{Bildungsspracherwerb} von Kindern\slash Jugendlichen fördern.\\
  Die Anforderungen an Sie ergeben sich aus den \alert{Zielkompetenzen Ihrer Schüler}.\\
  \onslide<+->
  \Viertelzeile 
  \begin{block}
    {Zielkompetenzen \textit{Deutsch} 5.--11.~Klasse (Thüringer RLP 2019; S.~7)}
    \begin{enumerate}[<+->]
      \item Texte rezipieren
      \item Texte produzieren
      \item \alert{Über Sprache, Sprachverwendung und Sprachenlernen reflektieren}
    \end{enumerate}
  \end{block}
  \onslide<+->
  Aufgabenspektrum\\
  \Viertelzeile 
  \begin{itemize}[<+->]
    \item \alert{Bildungssprache\slash Sprachbewusstheit} unterrichten
    \item Sprachliche Leistungen \rot{fair} bewerten
    \item Bewertungen und Lösungsstrategien \alert{erklären}
    \item \alert{Deutsche Sprache} vermitteln (falls nicht L1)
    \item \rot{Wie soll das ohne fundierte Grammatikkenntnisse funkionieren?}
    \item \gruen{Nach Morphologie, Syntax-Vorlesung und Syntax-Seminar geht es hier weiter!}
  \end{itemize}
\end{frame}




\section{Konstituenz}

\begin{frame}
  {Einteilung in Einheiten}
  \onslide<+->
  \onslide<+->
  \alert{Parataxe} | Einbettung von ganzen Satzstrukturen\\
  \Viertelzeile
  \onslide<+->
  \begin{exe}
    \ex dass Max glaubt, [dass Julius weiß, [dass Otto behauptet, [dass Karl vermutet, [dass Richard bestätigt, [dass Friederike lacht]]]]]
  \end{exe}
  \onslide<+->
  \Zeile
  Parataxe als Spezialfall | \alert{Konstitueten in Konstituenten}\\
  \onslide<+->
  \Viertelzeile
  \begin{exe}
    \ex {}[das Haus [des Autors [von Zettel Traum [den ich 1993 gelesen habe]]]]
    \ex {}[[den][ich][1993][[gelesen]habe]]
  \end{exe}
\end{frame}

\begin{frame}
  {Naive Konstituenzanalyse}
  \onslide<+->
  \onslide<+->
  \centering
  \scalebox{0.8}{\begin{forest}
    [NP
      [Artikel
        [das]
      ]
      [N$'$
        [N
          [Haus]
        ]
        [NP
          [Artikel
            [des]
          ]
          [N$'$
            [N
              [Autors]
            ]
            [PP
              [P
                [von]
              ]
              [NP
                [NP
                  [Zettels Traum]
                ]
                [Relativsatz
                  [den \ldots\ habe]
                ]
              ]
            ]
          ]
        ]
      ]
    ]
  \end{forest}}
\end{frame}


\begin{frame}
  {Konstituententests}
  \onslide<+->
  \onslide<+->
  Welche \alert{Konstituententests} kennen Sie?\\
  \Zeile 
  \begin{itemize}[<+->]
    \item Substituierbarkeit\slash Pronominalisierungstest\slash Fragetest
    \item Weglaßtest
    \item Verschiebetest (Umstelltest)\slash Vorfeldtest
    \item Koordinationstest
  \end{itemize}
\end{frame}


\begin{frame}
  {Konstituententests I}
  \onslide<+->
  \onslide<+->
  \begin{description}
  \item[Substituierbarkeit]
    Ausstauschbare Wortfolgen als potenzielle Konstituenten
    \onslide<+->
    \begin{exe}
      \ex Er kennt \orongsch{den Mann}.
      \ex Er kennt \orongsch{eine Frau}.
    \end{exe}
    \Zeile
    \onslide<+->
  \item[Pronominalisierungstest]
    Dasselbe, aber spezifisch mit pronominalen Ein-Wort-Folgen
    \onslide<+->
    \begin{exe}
      \ex \orongsch{Der Mann} schläft.
      \ex \orongsch{Er} schläft.
    \end{exe}
  \end{description}
\end{frame}

\begin{frame}
  {Konstituententests II}
  \onslide<+->
  \onslide<+->
  \begin{description}
    \item[Fragetest]
    Erfragbarkeit von Konstituenten
    \onslide<+->
      \begin{exe}
        \ex \orongsch{Der Mann} arbeitet.
        \ex \orongsch{Wer} arbeitet?
      \end{exe}
      \onslide<+->
      \Halbzeile
  \item[Verschiebetest] 
   Umstellbarkeit von Konstituenten 
   \onslide<+->
      \begin{exe}
        \ex weil \orongsch{keiner} \gruen{diese Frau} kennt.
        \ex weil \gruen{diese Frau} \orongsch{keiner} kennt.
      \end{exe}
      \Halbzeile
      \onslide<+->
  \item[Koordinationstest]
    Konstituenten als koordinierbar
    \onslide<+->
      \begin{exe}
        \ex \alert{[}\orongsch{[Der Mann]} \alert{und} \orongsch{[die Frau]}\alert{]} arbeiten.
      \end{exe}
    \end{description}
\end{frame}


\section{Köpfe}



\begin{frame}
  {Köpfe}
  \onslide<+->
  \onslide<+->
  \alert{Kopf} | Festlegung der syntaktisch relevanten \alert{kategorialen Merkmale der Phrase}\\
  \onslide<+->
  \Halbzeile
  \begin{exe}
    \ex \alert{Träumt} er?
    \ex \alert{Erwartet} er einen dreiprozentigen Anstieg?
    \ex \alert{in} diesem Haus
    \ex ein \alert{Mann}
  \end{exe}
  \Halbzeile
  \begin{itemize}[<+->]
    \item \alert{Projektion} | Kombination eines Kopfes mit anderem Material
    \item \alert{Maximalprojektion} | Vollständige Projektion
    \item \alert{Satz} | Maximalprojektion eines finiten Verbs
  \end{itemize}
\end{frame}

\begin{frame}
  {Naive Konstituenzanalyse mit Markierung der Köpfe}
  \onslide<+->
  \onslide<+->
  \centering
  \scalebox{0.8}{\begin{forest}
    [NP, name=NP2, blue
      [Artikel
        [das]
      ]
      [N$'$, name=N2b, blue
        [N, blue, name=N2a
          [Haus, blue]
        ]
        [NP, name=NP1, green
          [Artikel
            [des]
          ]
          [N$'$, green, name=N1b
            [N, green, name=N1a
              [Autors, green]
            ]
            [PP, name=PP, orange
              [P, name=P, orange
                [von, orange]
              ]
              [NP, red, name=NP3b
                [NP, red, name=NP3a
                  [Zettels Traum, red]
                ]
                [Relativsatz
                  [den \ldots\ habe]
                ]
              ]
            ]
          ]
        ]
      ]
      {\draw [->, bend left=45, orange, thick] (P.north) to (PP.west);}
      {\draw [->, bend left=45, green, thick] (N1a.north) to (N1b.west);}
      {\draw [->, bend right=45, green, thick] (N1b.north) to (NP1.east);}
      {\draw [->, bend left=45, blue, thick] (N2a.north) to (N2b.west);}
      {\draw [->, bend right=45, blue, thick] (N2b.north) to (NP2.east);}
      {\draw [->, bend left=45, red, thick] (NP3a.north) to (NP3b.west);}
    ]
  \end{forest}}
\end{frame}


\begin{frame}
  {Generalisierung durch Phrasenbildung}
  \onslide<+->
  Der \alert{interne Aufbau} einer Phrase ist für den Kontext \alert{irrelevant}:\\
  \onslide<+->
  \Viertelzeile
  \begin{exe}
    \ex er
    \ex der Mann
    \ex der Mann aus Stuttgart
    \ex der Mann aus Stuttgart, den wir kennen
  \end{exe}
  \onslide<+->
  \Halbzeile
  Bestimmte \alert{Merkmale} des Kopfs sind aber \alert{kontextrelevant}:\\
  \onslide<+->
  \begin{exe}
   \ex[]{Der Kollege liest einen Aufsatz.} 
   \ex[*]{Die Kollegen liest einen Aufsatz.} 
   \ex[*]{ Des Kollegen liest einen Aufsatz.} 
  \end{exe} 
\end{frame}


\begin{frame}
  {Naive Konstituenzanalyse mit Projektion von Kopfmerkmalen}
  \onslide<+->
  \onslide<+->
  \centering
  \scalebox{0.6}{\begin{forest}
    [NP\\\footnotesize Nom Sg Neut, name=NP2, blue
      [Artikel
        [das]
      ]
      [N$'$\\\footnotesize Nom Sg Neut, name=N2b, blue
        [{N\\\footnotesize Nom Sg Neut}, blue, name=N2a
          [Haus, blue]
        ]
        [{NP\\\footnotesize Gen Sg Mask}, name=NP1, green
          [Artikel
            [des]
          ]
          [{N$'$\\\footnotesize Gen Sg Mask}, green, name=N1b
            [{N\\\footnotesize Gen Sg Mask}, green, name=N1a
              [Autors, green]
            ]
            [{PP\\\footnotesize ?}, name=PP, orange
              [{P\\\footnotesize ?}, name=P, orange
                [von, orange]
              ]
              [{NP\\\footnotesize Dat Sg Mask}, red, name=NP3b
                [{NP\\\footnotesize Dat Sg Mask}, red, name=NP3a
                  [Zettels Traum, red]
                ]
                [Relativsatz
                  [den \ldots\ habe]
                ]
              ]
            ]
          ]
        ]
      ]
      {\draw [->, bend left=45, orange, thick] (P.north) to (PP.west);}
      {\draw [->, bend left=45, green, thick] (N1a.north) to (N1b.west);}
      {\draw [->, bend right=45, green, thick] (N1b.north) to (NP1.east);}
      {\draw [->, bend left=45, blue, thick] (N2a.north) to (N2b.west);}
      {\draw [->, bend right=45, blue, thick] (N2b.north) to (NP2.east);}
      {\draw [->, bend left=45, red, thick] (NP3a.north) to (NP3b.west);}
    ]
  \end{forest}}
\end{frame}


\section{Argumente und Adjunkte}

\begin{frame}
  {Valenz und logische Argumente}
  \onslide<+->
  \onslide<+->
  Nicht alle Phrasen, die vom Verb abhängen, stehen in derselben Art Relation zu ihm.\\
  \Zeile
  \begin{itemize}[<+->]
    \item Konstituenten | Verschiedenartige Beziehungen zu ihrem Kopf
    \item Semantische Beteiligte -- \alert{Aktanten} -- als \alert{feste Teile der Verbbedeutung}
    \item Semantik von \textit{sehen} | Immer ein \alert{Sehender}, ein \alert{Gesehenes}
      \Viertelzeile
      \begin{exe}
       \ex Dani sieht den Chaoten.
      \end{exe}
    \item \alert{Logische Argumente von \textit{sehen}} | Dani und der Chaot
    \item Valenz | Abbildung logischer Argumente auf grammatische Argumente
  \end{itemize}
\end{frame}

\begin{frame}
  {Optionale Argumente}
  \onslide<+->
  \onslide<+->
  Semantische Argumente | Nicht immer syntaktisch erforderlich\\
  \onslide<+->
  \Halbzeile
    \begin{exe}
      \ex Er wartet \alert{auf den Installateur}.
      \ex Er wartet.
    \end{exe}
  \onslide<+->
  \Zeile
  Bei \alert{Nominalisierung} | Alle Argumente optional\\
  \onslide<+->
  \Halbzeile
    \begin{exe}
      \ex \gruen{Arno} \alert{liest} \orongsch{diese Bücher}.
      \ex \alert{das Lesen} \orongsch{dieser Bücher} \gruen{durch Arno}
      \ex \alert{das Lesen} \orongsch{dieser Bücher} 
      \ex \alert{das Lesen} 
    \end{exe}
\end{frame}



\begin{frame}
  {Syntaktische Argumente, die keine logischen sind}
  \onslide<+->
  \onslide<+->
  Oben waren alle \alert{syntaktischen Argumente} auch \gruen{logische Argumente}.\\
  \onslide<+->
  \Halbzeile
  \begin{exe}
    \ex \alert{Dani} sieht \alert{den Chaoten}.
  \end{exe}
  \onslide<+->
  \Zeile
  \orongsch{Syntaktische Argumente}, die keine logischen sind:\\
  \Halbzeile
  \onslide<+->
  \begin{exe}
    \ex \orongsch{Es} regnet.
    \ex Conny erholt \orongsch{sich}.
  \end{exe}
\end{frame}

\begin{frame}
  {Adjunkte}
  \onslide<+->
  \onslide<+->
  \alert{Adjunkte} | Keine verbgebundene, sondern \alert{selbst mitgebrachte} Rolle\\
  \onslide<+->
  \Halbzeile
  \begin{exe}
    \ex \alert{Dani} sieht \alert{den Chaoten} \orongsch{bellend} \rot{auf der Brücke}.
  \end{exe}
  \onslide<+->
  \Zeile
  Deutliche Unterschiede zwischen Argumenten und Adjunkten\\
  \Halbzeile
  \begin{itemize}[<+->]
    \item \alert{Sehende und Gesehener} | Fester Teil einer \textit{sehen}-Situation
    \item \rot{Ort} | Teil so ziemlich jedes Geschehens, nicht \textit{sehen}-spezifisch
    \item \orongsch{Verhalten des Beteiligten} | Erst recht nicht \textit{sehen}-spezifisch
  \end{itemize}
\end{frame}
 
\begin{frame}
  {Andere Bezeichnungen}
  \onslide<+->
  \onslide<+->
  Üblicher Terminologie-Wildwuchs in der Linguistik\\
  \Zeile
  \begin{itemize}[<+->]
    \item Argument = \alert{Ergänzung}
    \item Adjunkt = \alert{(freie) Angabe}
      \Halbzeile
    \item Argumente | Beim Verb aufgeteilt in \alert{Subjekte} und \alert{Komplemente} 
    \item \alert{Aktant} Subjekte und Objekte (nicht Prädikative und Adverbiale)
      \Halbzeile
    \item \alert{Adverbial} | Angabe beim Verb
      \begin{itemize}
        \item Raum (Lage, Richtung/Ziel, Herkunft, Weg)
        \item Zeit (Zeitpunkt, Anfang, Ende, Dauer)
        \item Grund (inkl.\ Gegengrund, Bedingung)
        \item Art und Weise
      \end{itemize}
  \end{itemize}
\end{frame}


\section{Grammatische Funktionen}

\begin{frame}
  {Grammatische Funktionen (eigentlich Relationen)}
  \onslide<+->
  \onslide<+->
  Grammatische Funktionen\slash Relationen sind oft nicht unabhängig definierbar!\\
  \Zeile
  \begin{itemize}[<+->]
    \item Typen von Argumenten\slash Adjunkten mit spezifischen Eigenschaften
      \Halbzeile
    \item \alert{Subjekt} | Siehe nächste Folien
    \item \alert{Objekt}\slash \alert{Komplement} | Nicht-Nominativ-Argumente
    \item \alert{Adverb}\slash \alert{Adverbiale Bestimmung} | Angabe des Verbs
  \end{itemize}
\end{frame}

\subsection{Subjekt}

\begin{frame}
  {Subjekt}
  \onslide<+->
  \onslide<+->
  Für \alert{deutsche Subjekte} benannte definitorische Kriterien:\\
  \Halbzeile
  \begin{enumerate}[<+->]
    \item \alert{Kongruenz} mit dem finiten Verb
    \item \alert{Nominativ} in nichtkopulativen Sätzen
    \item Weglassbarkeit in \alert{Infinitivkonstruktionen} (Kontrolle)
    \item Weglassbarkeit in \alert{Imperativsätzen}
  \end{enumerate}
  \Zeile
  \onslide<+->
  \citet{Reis82} | Nur (2) relevant!
\end{frame}


\begin{frame}
  {Dative sind keine Subjekte}
  \onslide<+->
  \onslide<+->
  Kongruenz:\\
  \onslide<+->
  \Viertelzeile
  \begin{exe}
    \ex[]{Er hilft den Männern.}
    \ex[]{Den Männern wurde geholfen.}
    \ex[*]{Den Männern wurden geholfen.}
  \end{exe}
  \onslide<+->
  \Halbzeile
  Keine Kontrolle in Infinitivkonstruktionen:\\
  \onslide<+->
  \Viertelzeile
  \begin{exe}
   \ex[]{Klaus behauptet, den Männern zu helfen.}
   \ex[]{Klaus behauptet, dass er den Männern hilft.}
   \ex[]{Klaus behauptet, seine Familie zu lieben.}
   \ex[]{Seine Familie behauptet, geliebt zu werden.}
   \ex[*]{Die Männer behaupten, geholfen zu werden.}
   \ex[*]{Die Männer behaupten, elegant getanzt zu werden.}
  \end{exe}
\end{frame}


\begin{frame}
  {Dative sind keine Subjekte}
  \onslide<+->
  \onslide<+->
  Weglassbarkeit in Imperativen:\\
  \Halbzeile
  \onslide<+->
  \begin{exe}
    \ex[]{Fürchte dich nicht!}
    \ex[*]{Graue nicht!}
    \ex[]{Werd einmal unterstützt und \ldots}
    \ex[*]{Werd einmal geholfen und \ldots}
  \end{exe}
\end{frame}

\section{Phrasenstrukturgrammatiken}

\begin{frame}
  {Phrasenstrukturen}
  \onslide<+->
  \onslide<+->
  \centering 
  \begin{tabular}{@{}l@{\hspace{1cm}}l@{}}
  \scalebox{.75}{%
  \begin{forest}
    [S
      [NP [er] ]
      [NP
        [Det [das] ]
        [N [Buch] ] 
      ]
      [NP
        [Det [dem] ]
        [N [Mann] ] 
      ]
      [V [gibt] ]
    ]
  \end{forest}} & \onslide<+->
  \scalebox{.75111111}{%
  \begin{forest}
    [V
      [NP [er] ]
      [V
        [NP
          [Det [das] ]
          [N [Buch] ] ]
        [V
          [NP
            [Det [dem] ]
            [N [Mann] ] ]
          [V [gibt] ] ] ] ]
  \end{forest}}
  \\
  \\
  \onslide<+->
    \begin{tabular}{@{~}l@{ }l@{}}
      \multicolumn{2}{l}{\grau{Grammatik}} \\
      NP & \goesto Det N            \\
      S  & \goesto NP NP NP V  \\
    \end{tabular} \onslide<+-> & \begin{tabular}{@{~}l@{ }l@{}}
      \multicolumn{2}{l}{\grau{Grammatik}} \\
      NP & \goesto Det N  \\
      V  & \goesto NP V\\
    \end{tabular}\\
  \end{tabular}
\end{frame}

\begin{frame}
  {Wie PSG-Regeln als Ersetzungsregeln funktionieren}
  \onslide<+->
  \onslide<+->
  Ersetzungsregeln und Bäume als Protokoll der Ersetzung\\
  \Zeile
  \onslide<+->
  \begin{tabular}[t]{@{}l@{ }l}
    \multicolumn{2}{l}{\grau{Grammatik}} \\
    \alert<8,11>{NP} & \alert<8,11>{\goesto Det N}\\
    \alert<13>{S}  & \alert<13>{\goesto NP NP NP V}
  \end{tabular}\hspace{2cm}%
  \begin{tabular}[t]{@{}l@{ }l}
    \multicolumn{2}{l}{\grau{Lexikon (gleiches Format)}} \\
    \alert<5>{NP} & \alert<5>{\goesto er}\\
    \alert<6>{Det}  & \alert<6>{\goesto das}\\
    \alert<9>{Det}  & \alert<9>{\goesto dem}\\
  \end{tabular}\hspace{8mm}
  \begin{tabular}[t]{@{}l@{ }l}
    &\\
    \alert<7>{N} & \alert<7>{\goesto Buch}\\
    \alert<10>{N} & \alert<10>{\goesto Mann}\\
    \alert<12>{V} & \alert<12>{\goesto gibt}\\
  \end{tabular}\\
  \onslide<+->
  \Zeile
  \begin{minipage}{0.48\textwidth}
    \begin{tabular}{@{}llllll@{\hspace{2.5cm}}l}
      \visible<4->{er            & das          & Buch          & dem          & Mann & gibt                }\\
      \visible<5->{\alert<5>{NP} & das          & Buch          & dem          & Mann & gibt &  }\\
      \visible<6->{NP            & \alert<6>{Det} & Buch          & dem          & Mann & gibt &   }\\
      \visible<7->{NP            & Det            & \alert<7>{N}  & dem          & Mann & gibt &  }\\
      \visible<8->{NP            &              & \alert<8>{NP} & dem          & Mann & gibt & }\\
      \visible<9->{NP            &              & NP            & \alert<9>{Det} & Mann & gibt &  }\\
      \visible<10->{NP            &              & NP            & Det            & \alert<10>{N}    & gibt  &  }\\
      \visible<11->{NP            &              & NP            &              & \alert<11>{NP}       & gibt & }\\
      \visible<12->{NP            &              & NP            &              & NP       & \alert<12>{V}   &   }\\
      \visible<13->{              &              &               &              &      & \alert<13>{S}      & }\\
    \end{tabular}
  \end{minipage}~\begin{minipage}{0.48\textwidth}
    \centering
    \begin{forest}
      [S, visible on=<4->, white on=<4->, blue on=<13->
        [NP, ake, blue on=<5>, black on=<6->
          [er, tier=term, visible on=<4->, black on=<4->]
        ]
        [NP, blue on=<8>, black on=<9->
          [Det, blue on=<6>, black on=<7->
            [das, tier=term, visible on=<4->, black on=<4->]
          ]
          [N, blue on=<7>, black on=<8->
            [Buch, tier=term, visible on=<4->, black on=<4->]
          ]
        ]
        [NP, ake, blue on=<11>, black on=<12->
          [Det, blue on=<9>, black on=<10->
            [dem, tier=term, visible on=<4->, black on=<4->]
          ]
          [N, blue on=<10>, black on=<11->
            [Mann, tier=term, visible on=<4->, black on=<4->]
          ]
        ]
        [V, ake, blue on=<12>, black on=<13->
          [gibt, tier=term, visible on=<4->, black on=<4->]
        ]
      ]
    \end{forest}
  \end{minipage}
\end{frame}


\begin{frame}
  {Phrasenstrukturschemata}
  \onslide<+->
  \onslide<+->
  Manche kennen die \alert{Phrasenschemata} aus \citet{Schaefer2018a}.\\
  \onslide<+->
  \Halbzeile
  \centering 
  \scalebox{0.7}{\begin{forest}
    phrasenschema
    [NP, Ephr
      [Art, Eopt, Emult, [NP\Sub{Genitiv}, Eopt]]
      [AP, Eopt, Erec]
      [N, Ehd, name=Nkopf]
      [innere Rechtsattribute, Eopt, Erec]
      {\draw [bend left=45, dashed,<-] (.south) to (Nkopf.south);}
      [RS, Eopt, Erec]
    ]
  \end{forest}}\\
  \onslide<+->
  \raggedright
  \Zeile
  Es handelt sich um \alert{abgekürzte Phrasenstrukturregeln}.\\
  \onslide<+->
  \Halbzeile
  \scalebox{0.8}{\begin{tabular}[h]{lll}
    NP \goesto N & NP \goesto Art N & NP \goesto NP\Sub{Gen} N \\
    \grau{\textit{Bücher}} & \grau{\textit{das Buch}} & \grau{\textit{Arnos Buch}}  \\
    \visible<6->{NP \goesto N Rechtsattribut\Up{n} & NP \goesto Art N Rechtsattribut\Up{n} & NP \goesto NP\Sub{Gen} N Rechtsattribut\Up{n} \\
    \grau{\textit{Bücher über Poe}} & \grau{\textit{das Buch über Poe}} & \grau{\textit{Arnos Buch über Poe}}  \\}
    \visible<7->{NP \goesto N RS\Up{n} & NP \goesto Art N RS\Up{n} & NP \goesto NP\Sub{Gen} N RS\Up{n} \\
    \grau{\textit{Bücher, die gefallen}} & \grau{\textit{das Buch, das gefällt}} & \grau{\textit{Arnos Buch, das gefällt}}  \\}
    & & \\
    \multicolumn{3}{l}{\visible<8->{\gruen{NP \goesto (Art | NP\Sub{Gen} ) (AP\Up{n}) N (Rechtsattribut\Up{n}) (RS\Up{n})}}} \\
    \multicolumn{3}{l}{\visible<9->{Rechtsattribut NP \goesto PP, NP\Sub{Gen}, CP, IP, \ldots}} \\
  \end{tabular}}
\end{frame}

\begin{frame}
  {Von der Grammatik beschriebene Sätze}
  \onslide<+->
  \onslide<+->
  Die folgende Grammatik \rot{übergeneriert}!\\
  \onslide<+->
  \Zeile
  \begin{tabular}{@{}l@{ }l}
    NP & \goesto Det N\\
    S  & \goesto NP NP NP V\\
  \end{tabular}
  \onslide<+->
  \Zeile
  \begin{exe}
    \ex[]{er das Buch dem Mann gibt}
    \onslide<+->
    \ex[*]{ich das Buch dem Mann gibt\\
      \onslide<+->
      \alert{Subjekt"=Verb"=Kongruenz} | \orongsch{{\em ich\/} -- {\em gibt\/}}}
      \onslide<+->
    \ex[*]{er das Buch den Mann gibt\\
      \onslide<+->
      \alert{Valenz}\slash\alert{Rektion} | \orongsch{{\em gibt\/} $+$ Dativ}}
      \onslide<+->
    \ex[*]{er den Buch dem Mann gibt\\
      \onslide<+->
      \alert{Determinator"=Nomen"=Kongruenz} | \orongsch{{\em den\/} -- {\em Buch\/}}}
  \end{exe}
\end{frame}


\begin{frame}
  {Subjekt"=Verb"=Kongruenz}
  \onslide<+->
  \onslide<+->
  Übereinstimmung in \alert{Person (1, 2, 3)} und Numerus \alert{(sg, pl)}\\
  \Zeile
  \onslide<+->
  \begin{exe}
    \ex Ich schlafe. (1, sg)
    \ex Du schläfst.  (2, sg)
    \ex Er schläft. (3, sg)
    \ex Wir schlafen. (1, pl)
    \ex Ihr schlaft.  (2, pl)
    \ex Sie schlafen. (3,pl)
  \end{exe}
  \Zeile
  \onslide<+->
  \centering 
  Wie drückt man das in Regeln aus?
\end{frame}
 
 
\begin{frame}
  {Regelinflation}
  \onslide<+->
  \onslide<+->
  Verfeinerung der verwedenten Symbole | Statt S \goesto NP NP NP V\\
  \onslide<+->
  \Zeile
  \centering 
  \begin{tabular}{@{}l@{ }l}
    S  & \goesto \alert{NP\_1\_sg} NP NP \alert{V\_1\_sg} \\\onslide<+-> 
    S  & \goesto \alert{NP\_2\_sg} NP NP \alert{V\_2\_sg} \\\onslide<+->
    S  & \goesto \alert{NP\_3\_sg} NP NP \alert{V\_3\_sg} \\\onslide<+->
    S  & \goesto \alert{NP\_1\_pl} NP NP \alert{V\_1\_pl} \\\onslide<+->
    S  & \goesto \alert{NP\_2\_pl} NP NP \alert{V\_2\_pl} \\\onslide<+->
    S  & \goesto \alert{NP\_3\_pl} NP NP \alert{V\_3\_pl} \\
  \end{tabular}\\
  \Zeile
  \onslide<+->
  \rot{Sechs Regeln} ($3\times 2$) statt einer!
\end{frame}
 
 
\begin{frame}
  {Kasuszuweisung durch das Verb}
  \onslide<+->
  \onslide<+->
  Hier für ein Valenzmuster (\alert{ditransitiv}) die Kongruenzkodierung.\\
  \onslide<+->
  \Zeile
  \centering 
  \begin{tabular}{@{}l@{ }l}
    S  & \goesto \alert{NP\_1\_sg\_nom} NP\_dat NP\_acc \alert{V\_1\_sg}\_ditransitiv\\\onslide<+-> 
    S  & \goesto \alert{NP\_2\_sg\_nom} NP\_dat NP\_acc \alert{V\_2\_sg}\_ditransitiv\\\onslide<+->
    S  & \goesto \alert{NP\_3\_sg\_nom} NP\_dat NP\_acc \alert{V\_3\_sg}\_ditransitiv\\\onslide<+->
    S  & \goesto \alert{NP\_1\_pl\_nom} NP\_dat NP\_acc \alert{V\_1\_pl}\_ditransitiv\\\onslide<+->
    S  & \goesto \alert{NP\_2\_pl\_nom} NP\_dat NP\_acc \alert{V\_2\_pl}\_ditransitiv\\\onslide<+->
    S  & \goesto \alert{NP\_3\_pl\_nom} NP\_dat NP\_acc \alert{V\_3\_pl}\_ditransitiv\\
  \end{tabular}\\
  \onslide<+->
  \Zeile
  \alert{NP} | \rot{$3\times2\times4=24$} neue Kategorien\\
    \onslide<+->
    \Viertelzeile 
    \alert{V} | Für $n$ Valenzmuster \rot{$3\times2\times n$} Kategorien 
\end{frame}


\begin{frame}
  {Determinator"=Nomen"=Kongruenz}
  \onslide<+->
  \onslide<+->
  Übereinstimmung in \alert{drei Genera}, \alert{zwei Numeri} und \alert{vier Kasus}!\\
  \Halbzeile
  \onslide<+->
  \begin{exe}
    \ex der Mann, die Frau, das Buch (Genus)
    \ex das Buch, die Bücher (Numerus)
    \ex des Buches, dem Buch (Kasus)
  \end{exe}
  \onslide<+->
  \Halbzeile
  \centering 
  \resizebox{0.8\linewidth}{!}{
  \begin{tabular}{@{}l@{ }l@{\hspace{4mm}}l@{ }l}
    NP\_3\_sg\_nom  & \goesto Det\_fem\_sg\_nom N\_fem\_sg\_nom & NP\_gen  & \goesto Det\_fem\_sg\_gen N\_fem\_sg\_gen\\ 
    NP\_3\_sg\_nom  & \goesto Det\_mas\_sg\_nom N\_mas\_sg\_nom & NP\_gen  & \goesto Det\_mas\_sg\_gen N\_mas\_sg\_gen\\
    NP\_3\_sg\_nom  & \goesto Det\_neu\_sg\_nom N\_neu\_sg\_nom & NP\_gen  & \goesto Det\_neu\_sg\_gen N\_neu\_sg\_gen\\
    NP\_3\_pl\_nom  & \goesto Det\_fem\_pl\_nom N\_fem\_pl\_nom & NP\_gen  & \goesto Det\_fem\_pl\_gen N\_fem\_pl\_gen\\
    NP\_3\_pl\_nom  & \goesto Det\_mas\_pl\_nom N\_mas\_pl\_nom & NP\_gen  & \goesto Det\_mas\_pl\_gen N\_mas\_pl\_gen\\
    NP\_3\_pl\_nom  & \goesto Det\_neu\_pl\_nom N\_neu\_pl\_nom & NP\_gen  & \goesto Det\_neu\_pl\_gen N\_neu\_pl\_gen\\
    \grau{\ldots} & \grau{\goesto Dativ}                                                             & \grau{\ldots} & \grau{\goesto Akkusativ} \\
  \end{tabular}
  }\\
  \Halbzeile
  \onslide<+->
  \rot{Je 24 Symbole} für Determinatoren und Substantive, \rot{24 Regeln}
\end{frame}
 
\begin{frame}
  {Das Problem sind nicht die vielen Regeln!}
  \onslide<+->
  \onslide<+->
  Syntaktische \rot{Generalisierungen werden nicht erfaßt}.\\
  \Zeile
  \begin{itemize}[<+->]
      \item Beispiel Generalisierung | \alert{Wo kann eine NP oder NP\_nom stehen?}
      \item Nicht: \rot{Wo kann eine NP\_3\_sg\_nom stehen?}
    \end{itemize}
  \onslide<+->
  \Zeile
  Lösung | \alert{Komplexe Kategorien} mit Merkmalen, Werten und Identität von Werten\\
  \Zeile
  \centering
  \onslide<+->
  \begin{tabular}{@{}l@{ }l}
    NP(3,sg,nom)  & \goesto Det(fem,sg,nom) N(fem,sg,nom)\\
    NP(3,sg,nom)  & \goesto Det(mas,sg,nom) N(mas,sg,nom)\\
  \end{tabular}
\end{frame}


\begin{frame}
  {Merkmale und Regelschemata}
  \onslide<+->
  \onslide<+->
  Regelschemata mit \alert{variablen Werten} und ggf.\ \rot{festen Werten}\\
  \onslide<+->
  \Zeile
  \centering 
  {\Large NP(\rot{3}, \alert{Num}, \orongsch{Kas}) \goesto Det(\gruen{Gen}, \alert{Num}, \orongsch{Kas}) N(\gruen{Gen}, \alert{Num}, \orongsch{Kas}) }
  \Zeile
  \begin{itemize}[<+->]
    \item \rot{Genus} | Festgelegt durch Regel (NP mit Appellativum)
    \item \alert{Numerus} und \orongsch{Kasus} | Müssen übereinstimmen, sind an Projektion sichtbar
    \item \gruen{Genus} | Muss übereinstimmen, an Projektion sichtbar
    \Zeile
    \item Wohlgeformte und nicht wohlgeformte NP nach dieser Regel:
      \Viertelzeile
      \begin{itemize}[<+->]
        \item des Baums \\
          NP(\rot{3}, \alert{sg}, \orongsch{gen}) \goesto Det(\gruen{mask}, \alert{sg}, \orongsch{dat}) N(\gruen{mask}, \alert{sg}, \orongsch{gen})
          \Viertelzeile
        \item des Bäumen \\
          NP(\rot{3}, \alert{?}, \orongsch{?}) \goesto Det(\gruen{mask}, \alert{sg}, \orongsch{gen}) N(\gruen{mask}, \alert{pl}, \orongsch{dat})
      \end{itemize}
  \end{itemize}
\end{frame}


\begin{frame}
  {Zusammenspiel von Regelschemata}
  \onslide<+->
  \onslide<+->
  Grammatik mit Kongruenz und rudimentärer Valenz\\
  \Halbzeile
  \begin{itemize}[<+->]
    \item[ ] NP(3, Num, Kas) \goesto Det(Gen, Num, Kas) N(Gen, Num, Kas)
    \item[ ] S \goesto NP(\rot{Per}, \rot{Num}, \alert{nom}) V\_itr(\rot{Per}, \rot{Num})
    \item[ ] S \goesto NP(\rot{Per1}, \rot{Num1}, \alert{nom}) NP(Per2, Num2, \gruen{akk}) V\_tr(\rot{Per1}, \rot{Num1})
    \item[ ] S \goesto NP(\rot{Per1}, \rot{Num1}, \alert{nom}) NP(Per2, Num2, \orongsch{dat}) NP(Per3, Num3, \gruen{akk}) V\_tr(\rot{Per1}, \rot{Num1})
  \end{itemize}
  \Zeile
  \begin{itemize}[<+->]
    \item \rot{Kongruenzmerkmale}
    \item \alert{Valenz noch in} \gruen{der Regel und einem} \orongsch{Verbsymbol kodiert}
  \end{itemize}
\end{frame}


\begin{frame}
  {Hinweis zu Merkmalen und Werten}
  \onslide<+->
  \onslide<+->
  Merkmalsmengen in den obigen Regeln müssen geordnet sein!\\
  \Halbzeile
  \begin{itemize}[<+->]
    \item N(mask, sg, nom) | \alert{Werte} in \alert{bestimmter Reihenfolge}: Genus, Numerus, Kasus
    \item N(Gen, Num, Kas) | \alert{Variablen} für Werte in dieser Reihenfolge
    \item N(Bim, Bam, Bum) | Genau so gute \alert{Variablennamen} (gleiche Reihenfolge!)
    \item N(V1, V2, V3) | \alert{Indizierte Variablennamen} (gleiche Reihenfolge!)
    \item N(\_, \_, \_) | Irrelevante Werte für Genus, Numerus, Kasus \alert{in dieser Reihenfolge}
  \end{itemize}
  \onslide<+->
  \Zeile
  Andere Möglichkeit | Trennung von Merkmal und Wert\\
  \Halbzeile
  \begin{itemize}[<+->]
    \item N\{\orongsch{Gen}=\gruen{mask}, \orongsch{Num}=\gruen{sg}, \orongsch{Kas}=\gruen{mask}\} | Benennung von \orongsch{Merkmal}, \gruen{Wert}
    \item N\{\orongsch{Kas}=\gruen{mask}, \orongsch{Gen}=\gruen{mask}, \orongsch{Num}=\gruen{sg}\} | Reihenfolge egal
  \end{itemize}
\end{frame}


  \let\subsection\section\let\section\woopsi

  \section[Grammatik]{Grammatik}
  \let\woopsi\section\let\section\subsection\let\subsection\subsubsection
  
\section{Einleitung}

\begin{frame}
  {Ziele}
  \onslide<+->
  \onslide<+->
  Worum geht es heute?\\
  \Zeile
  \begin{itemize}[<+->]
    \item Repräsentation von Merkmalen und ihren Werten in Grammatiken
    \item Strukturierte\slash hierarchische Merkmalstrukturen
    \item Unifikation von Merkmalstrukturen
    \item Merkmalstrukturen vs.\ Merkmalbeschreibungen
  \end{itemize}
  \Zeile
  \centering
  \onslide<+->
  \grau{\citet[Kapitel~2]{MuellerLehrbuch}}
\end{frame}

\begin{frame}
  {Warnung}
  \onslide<+->
  \onslide<+->
  \centering 
  \LARGE
  \rot{Merken Sie sich die Strukturen von heute\\nicht als "`korrekte Modellierung"'\\
  des Deutschen in HPSG!}\\
  \Zeile
  \onslide<+->
  \normalsize
  Wir nehmen heute einige Vereinfachungen und Didaktisierungen vor,\\
  denn es geht darum, grundlegende Repräsentationen\slash Prinzipien einzuführen.\\
  \Zeile
  \onslide<+->
  Völlig abwegig sind die Strukturen dieser Lektion aber auch nicht.\\
  \Zeile
  \onslide<+->
  Generell haben Sie mehr davon, wenn Sie in jeder Woche zu verstehen versuchen,\\
  warum sich bestimmte Repräsentationen wieder ändern, als wenn Sie von Anfang an\\
  nur wissen wollen, wie das Endergebnis in den Prüfungen aussehen wird.
\end{frame}

\section{Merkmalstrukturen}

\begin{frame}
  {Vorteil von Merkmalstrukturen}
  \onslide<+->
  \onslide<+->
  Problem mit einfachen \alert{Phrasenstrukturgrammatiken}\\
  \Halbzeile
  \begin{itemize}[<+->]
    \item \alert{Symbolinflation} | Selbst für einfachete Valenz-\slash Kongruenzphänomene
    \item Viele Regeln und viele Kategorien
  \end{itemize}
  \onslide<+->
  \Zeile
  \alert{Merkmalstrukturen} wie in HPSG\\
  \Halbzeile
  \begin{itemize}[<+->]
    \item \alert{Komplexe Symbole}, dadurch weniger Symbole
    \item Extrem einfache \alert{Regeln} (Kombinatorik)
  \end{itemize}
\end{frame}

\begin{frame}
  {Merkmalstrukturen und Merkmalbeschreibungen}
  \onslide<+->
  \onslide<+->
  \alert{Merkmalstrukturen} modellieren linguistische Objekte.\\
  \Halbzeile
  \begin{itemize}[<+->]
    \item Merkmal-Wert-Struktur
    \item Attribut-Wert-Struktur
    \item \emph{Feature structure}
  \end{itemize}
  \onslide<+->
  \Zeile
  Wir nutzen \alert{Merkmalsbeschreibungen}, um über Merkmalstrukturen zu sprechen.\\
  \Halbzeile
  \begin{itemize}[<+->]
    \item \emph{Attribute-value matrix}
    \item \emph{Feature matrix}
  \end{itemize}
  \onslide<+->
  \Zeile
  \centering 
  \grau{\footnotesize\citet{Shieber86a}, \citet{ps}, \citet{Johnson88},\citet{Carpenter92a}, \citet{King94a}, \citet{Richter2004a-u,Richter2021a}}
\end{frame}

\begin{frame}
  {AVM-Format}
  \onslide<+->
  \onslide<+->
  Einfache Merkmalbeschreibung\\
  \onslide<+->
  \Viertelzeile
  \alert{\begin{avm}
    \[ attribut & wert \]
  \end{avm}}\\
  \onslide<+->
  \Zeile
  Mehrere Attribut-Wert-Paare in einer Struktur\\
  \onslide<+->
  \Viertelzeile
  \alert{\begin{avm}
    \[ attribut1 & wert1 \\
      attribut2 & wert2 \\
      \ldots & \ldots
    \]
    \end{avm}}\\
    \onslide<+->
    \Zeile
    Komplexe Merkmale können Werte von Attributen sein!\\
    \alert{\begin{avm}
      \[ attribut1a & wert1a \\
         attribut1b & 
                     \[ attribut2a & wert2a \\
                        attribut2b & wert2b \]
      \]
    \end{avm}
    }
\end{frame}

\begin{frame}
  {Wörter in Merkmalen beschreiben | Phone und Graphen}
  \onslide<+->
  \onslide<+->
  \textsc{phone} oder \textsc{graphen} | Aussprache bzw. Schreibung\\
  \onslide<+->
  \Halbzeile
  \alert{\begin{avm}
    \[ graphen & \textit{Tisch} \]
  \end{avm}}\\
  \onslide<+->
  \Zeile
  Aber reicht diese Datenstruktur?\\
  \Viertelzeile
  \begin{itemize}[<+->]
    \item \textit{Tisch} | Sieht aus wie ein \alert{Symbol} ohne Struktur
    \item Phonetik\slash Phonologie | \alert{Ketten} Phonen\slash Phonemen\\
      \grau{Bei \citet{Schaefer2018a} und anderen: Segmente}
    \item Phonologische Grammatik | Zugriff auf einzelne Segmente\\
      \grau{Auslautverhärtung | Zugriff auf letztes Segment einer Silbe}
  \end{itemize}
\end{frame}

\begin{frame}
  {Listen}
  \onslide<+->
  \onslide<+->
  Lösung für \textsc{graph(en)} oder \textsc{phon(e)} | \alert{Geordnete Listen}\\
  \onslide<+->
  \Halbzeile
  \alert{\begin{avm}
    \[ graph & \<\it T,i,s,c,h\> \]
  \end{avm}}\\
  \onslide<+->
  \Zeile
  Auf einer Liste stehen eigentlich auch Merkmalbeschreibungen.\\
  \onslide<+->
  \Viertelzeile
  \scalebox{0.8}{\begin{avm}
    \[ phon &  \<\[ art & plosiv \\ ort & alveolar \],
    \[ art & vokal \\ position & vorn \\ höhe & hoch \\ rundung & nein \],
    \[ art & frikativ \\ ort & alveolar \]
  \> \]
  \end{avm}}\\
  \onslide<+->
  \Zeile
  Strenggenommen falsche Kurzschreibweisen für \textsc{phon} in typischer HPSG\\
  \onslide<+->
  \Viertelzeile
  \orongsch{\begin{avm}
    \[ phon & \textit{Tisch} \]
  \end{avm}}\\
  \Viertelzeile
  \onslide<+->
  \orongsch{\begin{avm}
    \[ phon & \<\rm\it Tisch\> \]
  \end{avm}}
\end{frame}

\begin{frame}
  {Morpholosyntaktische Merkmale}
  \onslide<+->
  \onslide<+->
  Lösung für Probleme mit Genus usw.\ in PSGs von letzter Woche\\
  \onslide<+->
  \Viertelzeile
  \scalebox{0.8}{\begin{avm}
    \[ graphen & \textit{Tisch} \\
      \alert{genus} & \alert{maskulin} \\
      \alert{numerus} & \alert{singular} \\
      \alert{kasus} & \alert{nominativ}
    \]
  \end{avm}}\\
  \onslide<+->
  \Halbzeile
  Andere Merkmalausstattungen = andere sprachliche Zeichen\\
  \onslide<+->
  \Viertelzeile
  \scalebox{0.6}{\begin{avm}
    \[ graphen & \textit{Tisch} \\
      wortart & nomen \\
      genus & maskulin \\
      numerus & singular \\
      \orongsch{kasus} & \orongsch{akkusativ}
    \]
  \end{avm}}\onslide<+->
  \scalebox{0.6}{\begin{avm}
    \[ graphen & \textit{Tisch} \\
      wortart & nomen \\
      genus & maskulin \\
      numerus & singular \\
      \orongsch{kasus} & \orongsch{dativ}
    \]
  \end{avm}}\onslide<+->
  \scalebox{0.6}{\begin{avm}
    \[ graphen & \textit{Tisch} \\
      wortart & nomen \\
      genus & maskulin \\
      numerus & singular \\
      \orongsch{kasus} & \orongsch{genitiv}
    \]
  \end{avm}}\\
  \Halbzeile
  \onslide<+->
  Abgekürzte Schreibweise mit \gruen{\textit{oder} bzw. $\vee$}\\
  \onslide<+->
  \Viertelzeile
  \scalebox{0.8}{\begin{avm}
    \[ graphen & \textit{Tisch} \\
      wortart & nomen \\
      genus & maskulin \\
      numerus & singular \\
      \gruen{kasus} & \gruen{nominativ $\vee$ akkusativ $\vee$ dativ $\vee$ genitiv}
    \]
  \end{avm}}
\end{frame}

\begin{frame}
  {Dasselbe für eine Verbform}
  \onslide<+->
  \onslide<+->
  Verben | Teilweise dieselben, teilweise andere Merkmale verglichen mit Nomina\\
  \onslide<+->
  \Viertelzeile
  \begin{avm}
    \[ graphen & \textit{sieht} \\
      wortart & verb \\
      person & dritte \\
      numerus & singular \\
    \]
  \end{avm}\\
  \onslide<+->
  \Zeile
  Syntaktisch relevant auch \alert{Finitheit} bzw.\ \gruen{Status} \\
  \Viertelzeile
  \onslide<+->
  \begin{avm}
    \[ graphen & \textit{sieht} \\
      wortart & verb \\
      person & dritte \\
      numerus & singular \\
      \alert{finit} & \alert{ja} \\
    \]
  \end{avm}
  \onslide<+->
  \begin{avm}
    \[ graphen & \textit{gesehen} \\
      wortart & verb \\
      \alert{finit} & \alert{nein} \\
      \gruen{status} & \gruen{3} \\
    \]
  \end{avm}
\end{frame}

\section{Typen}

\begin{frame}
  {Getypte Strukturen}
  \onslide<+->
  \onslide<+->
  Nicht alle Wörter haben alle Merkmale. | \alert{Typen} und \alert{Beschränkungen} über Typen
  \onslide<+->
  \Zeile
  \begin{avm}
    \[ \asort{\alert{nomen}}
    graphen & \textit{Tischs} \\
    genus & maskulin \\
    numerus & singular \\
    kasus & genitiv \\
  \]
  \end{avm}\\
  \onslide<+->
  \begin{avm}
    \[ \asort{\alert{finites-verb}}
    graphen & \textit{sieht} \\
    person & dritte \\
    numerus & singular \\
    tempus & präsens \\
    modus & indikativ
  \]
  \end{avm}
  \onslide<+->
  \begin{avm}
    \[ \asort{\alert{infinites-verb}}
    graphen & \textit{gesehen} \\
    status & 1 
  \]
  \end{avm}
\end{frame}

\begin{frame}
  {Typenhierarchien}
  \onslide<+->
  \onslide<+->
  Typen sind sehr wichtig in HPSG und bilden \alert{Hierachien}. Denkbares Beispiel:\\
  \onslide<+->
  \Zeile
  \centering 
  \scalebox{0.7}{\begin{forest}
    [ wort
      [nomen
        [eigenname]
        [appellativum
          [zählsubstantiv]
          [stoffsubstantiv]
        ]
      ]
      [verb
        [finites-verb]
        [infinites-verb]
      ]
    ]
  \end{forest}}

  \Zeile
  \raggedright
  \begin{itemize}[<+->]
    \item Typen sind die eigentlichen \alert{Wortarten} in HPSG.
    \item \alert{Monotonizität} | \alert{Untertypen} erbt alle Merkmale\slash Beschränkungen ihrer \alert{Obertypen}.
    \item \alert{Mehrfachvererbung} | Ein Typ kann \alert{mehrere Obertypen} haben.
    \item \grau{Keine Sorge! Dazu kommen wir noch im Detail.}
  \end{itemize}
\end{frame}

\section{Strukturteilung}

\begin{frame}
  {Valenz}
  \onslide<+->
  \onslide<+->
  Letzte Woche in PSGs | Valenz doppelt in \alert{Kategorien} und \orongsch{Regeln} kodiert\\
  \Halbzeile
  \begin{itemize}[<+->]
    \item[ ] \orongsch{Regel} für Satz mit intransitivem Verb\\
      \footnotesize S \goesto NP(Per, Num, nom) \alert{V\_itr}(Per, Num)
    \item[ ] \orongsch{Regel} für Satz mit transivitem Verb\\
      \footnotesize S \goesto NP(Per1, Num1, nom) NP(Per2, Num2, akk) \alert{V\_tr}(Per1, Num1)
    \item[ ] \orongsch{Regel} für Satz mit ditransitivem Verb\\
      \footnotesize S \goesto NP(Per1, Num1, nom) NP(Per2, Num2, dat) NP(Per3, Num3, akk) \alert{V\_dtr}(Per1, Num1)
  \end{itemize}
  \onslide<+->
  \Zeile
  \alert{Typische Definition von Valenz allerdings}\\
  \onslide<+->
  Die \alert{Liste} der Ergänzungen eines Worts.\\
\end{frame}

\begin{frame}
  {Valenz als Liste}
  \onslide<+->
  \onslide<+->
  Valenz | \alert{Liste von Merkmalsbeschreibungen}\\
  \onslide<+->
  \Halbzeile
  \begin{avm}
    \[ \asort{finites-verb}
      graphen & \textit{sieht} \\
      person & dritte \\
      numerus & singular \\
      tempus & präsens \\
      modus & indikativ \\
      \alert{valenz} & \alert{\< \[ \asort{nomen} \], \[ \asort{nomen} \]\>}
    \]
  \end{avm}
\end{frame}

\begin{frame}
  {Hinreichende Beschreibung}
  \onslide<+->
  \onslide<+->
  Valenzliste | Hinreichend eingrenzende Beschreibung der Ergänzungen des Verbs\\
  \onslide<+->
  \Halbzeile
  \begin{avm}
    \[ \asort{finites-verb}
      graphen & \textit{sieht} \\
      person & dritte \\
      numerus & singular \\
      tempus & präsens \\
      modus & indikativ \\
      \alert{valenz} & \alert{\< \[ \asort{nomen}
                                    person & dritte \\
                                    numerus & singular \\
                                    kasus & nom \], \[ \asort{nomen}
                                                        kasus & akkusativ \]\>}
    \]
  \end{avm}
\end{frame}

\begin{frame}
  {Subjekt-Verb-Kongruenz und Strukturteilung}
  \onslide<+->
  \onslide<+->
  \alert{Übereinstimmung von Merkmalen} | Hart verdrahtet durch \gruen{Strukturteilung}\\
  \onslide<+->
  \Halbzeile
  \scalebox{0.8}{\begin{avm}
    \[ \asort{finites-verb}
      graphen & \textit{sieht} \\
      person & \alt<1-3>{\alert{dritte}}{\gruen{\@1 dritte}} \\
      numerus & \alt<1-3>{\alert{singular}}{\gruen{\@2 singular}} \\
      tempus & präsens \\
      modus & indikativ \\
      valenz & \< \[ \asort{nomen}
        person & \alt<1-3>{\alert{dritte}}{\gruen{\@1}} \\
        numerus & \alt<1-3>{\alert{singular}}{\gruen{\@2}} \\
                     kasus & nom \], \[ \asort{nomen}
                                         kasus & akkusativ \]\>
    \]
  \end{avm}}\\
  \onslide<+->
  \Zeile 
  \gruen{Strukturteilung bedeutet Token-Identität von Werten,} \rot{nicht Kopie}!\\
  Man kann sich die Nummern als \alert{Zeiger} auf dieselbe Datenstruktur vorstellen.
\end{frame}

\begin{frame}
  {Beispiel für Valenz einer Präposition}
  \onslide<+->
  \onslide<+->
  Valenz von Präpositionen | NP in einem bestimmten Kasus\\
  \onslide<+->
  \Halbzeile
  \begin{avm}
    \[ \asort{präposition} 
    graphen & \textit{wegen} \\
    valenz & \< \[ \asort{nomen} 
                kasus & genitiv \] \>
  \]
  \end{avm}\\
  \Zeile
  \begin{itemize}[<+->]
    \item Was ist mit \alert{argumentmarkierenden Präpositionen}\slash Präpositionalobjekten?\\
      \grau{\textit{leiden unter}, \textit{abhängen von}, \textit{glauben an} usw.}
    \item Was ist mit \alert{Wechselpräpositionen} mit Akkusativ oder Dativ?\\
      \grau{\textit{unter}, \textit{neben}, \textit{über} usw.}
  \end{itemize}
\end{frame}

\section{Phrasen und Kopfmerkmale}

\begin{frame}
  {Beispieleintrag für einen Determinierer}
  \onslide<+->
  \onslide<+->
  Kongruenzmerkmale in der NP auch beim Determinierer\\
  \onslide<+->
  \Zeile 
  \begin{avm}
    \[ \asort{determinierer} 
    graphen & \textit{des} \\
    genus & maskulin \\
    numerus & singular \\
    kasus & genitiv \\
  \]
  \end{avm}
\end{frame}

\begin{frame}
  {Determinierer in der NP}
  \onslide<+->
  \onslide<+->
  DP oder NP? | \alert{Für Deutsch ist eine NP-Analyse näherliegend.}\\
  \onslide<+->
  \Zeile
  \begin{avm}
    \[ \asort{nomen}
    graphen & \textit{Tischs} \\
    person & dritte \\
    genus & \alt<1-5>{maskulin}{\gruen{\@1 maskulin}} \\
    numerus & \alt<1-5>{singular}{\gruen{\@2 singular}} \\
    kasus & \alt<1-5>{genitiv}{\gruen{\@3 genitiv}} \\
    \alt<5->{\gruen{valenz} & \gruen{\<\[ \asort{determinierer} 
                             genus & \ \@1 \\
                             numerus & \@2 \\
                             kasus & \ \@3
                           \]\>}}{}
  \]
  \end{avm}\\
  \onslide<+->
  \Zeile
  Wie kann man \alert{Notwendigkeit von} und \alert{Kongruenz mit Determinierern} kodieren?
  \onslide<+->
  \onslide<+->
\end{frame}

\newcommand{\AvmA}{%
  \begin{avm}
    \[ \asort{nomen}
    graphen & \textit{Tischs} \\
    person & \gruen{\@1 dritte} \\
    genus & \gruen{\@2 maskulin} \\
    numerus & \gruen{\@3 singular} \\
    kasus & \gruen{\@4 genitiv} \\
    valenz & \<\[ \asort{determinierer} 
      genus & \ \gruen{\@2} \\
      numerus & \gruen{\@3} \\
      kasus & \ \gruen{\@4}
                           \]\>
  \]
  \end{avm}%
}

\newcommand{\AvmB}{%
  \begin{avm}
    \[ \asort{determinierer}
      graphen & \textit{des} \\
       genus & \gruen{\@2} \\
       numerus & \gruen{\@3} \\
       kasus & \gruen{\@4} \\
       valenz & \<\>
    \]
  \end{avm}
}

\newcommand{\AvmC}{%
  \begin{avm}
    \[ \asort{nomen} 
      graphen & \textit{des Tischs} \\
      person & \gruen{\@1} \\ 
      genus & \gruen{\@2} \\
      numerus & \gruen{\@3} \\
      kasus & \gruen{\@4} \\ 
  \]
  \end{avm}
}

\begin{frame}
  {NP mit Kongruenz als Baum}
  \onslide<+->
  \onslide<+->
  \orongsch{In HPSG gibt es eigentlich keine Bäume.} Zur Illustration aber hilfreich:\\
  \onslide<+->
  \Zeile
  \centering
  \begin{forest}
    [ \scalebox{0.6}{\AvmC}
      [ \scalebox{0.6}{\AvmB} ]
      [ \scalebox{0.6}{\AvmA} ]
    ]
  \end{forest}
\end{frame}

\begin{frame}
  {Offene Probleme}
  \onslide<+->
  \onslide<+->
  Wir haben jetzt so getan, \orongsch{als hätten wir schon eine Syntax}!\\
  \Zeile 
  \begin{itemize}[<+->]
    \item Eigentlich \alert{nur Lexikoneinträge}
    \item Fehlende \alert{Regeln für Kombinationsmechanismus}
    \item NP auf der letzten Folie | Nur eine grobe Idee, wo wir hin wollen
    \item \alert{Projektionsebenen} (N vs.\ NP) nicht unterscheidbar
    \item Also auch keine Identifikation von \alert{Köpfen}
    \item Identifikation der \alert{Merkmale, die vom Kopf zur Phrase projizieren}
    \item Zusammenbau von \textit{des Tischs} aus \textit{des} und \textit{Tischs}
  \end{itemize}
\end{frame}

\newcommand{\AvmAb}{%
  \begin{avm}
    \[ \asort{nomen}
    graphen & \textit{Tischs} \\
    \alert{kopf} & \alert{\[ 
      person & \@1 dritte \\
      genus & \@2 maskulin \\
      numerus & \@3 singular \\
      kasus & \@4 genitiv
    \]} \\
    valenz & \<\[ \asort{determinierer}
      kopf & \[ genus & \ \@2 \\
        numerus & \@3 \\
        kasus & \ \@4
      \]
      \]\>
  \]
  \end{avm}%
}

\newcommand{\AvmBb}{%
  \begin{avm}
    \[ \asort{determinierer}
      graphen & \textit{des} \\
      \alert{kopf} & \alert{\[ 
         genus & \@2 \\
         numerus & \@3 \\
         kasus & \@4
       \]} \\
       valenz & \<\>
    \]
  \end{avm}
}

\newcommand{\AvmCb}{%
  \begin{avm}
    \[ \asort{nomen} 
      graphen & \textit{des Tischs} \\
      \alert{kopf} & \alert{\[
        person & \@1 \\
        genus & \@2 \\
        numerus & \@3 \\
        kasus & \@4 
      \]}
  \]
  \end{avm}
}


\begin{frame}
  {Kopfmerkmale}
  \onslide<+->
  \onslide<+->
  \alert{Bündel der Merkmale}, die vom Kopf zur Phrase projizieren\\
  \onslide<+->
  \Halbzeile
  \centering
  \begin{forest}
    [ \alt<1-3>{\scalebox{0.6}{\AvmC}}{\scalebox{0.6}{\AvmCb}}
      [\alt<1-3>{\scalebox{0.6}{\AvmB}}{\scalebox{0.6}{\AvmBb}}]
      [\alt<1-3>{\scalebox{0.6}{\AvmA}}{\scalebox{0.6}{\AvmAb}}]
    ]
  \end{forest}
  \onslide<+->
\end{frame}

\newcommand{\AvmAc}{%
  \begin{avm}
    \[ \asort{nomen}
    graphen & \textit{Tischs} \\
    kopf & \gruen{\@5} \[ 
      person & \@1 dritte \\
      genus & \@2 maskulin \\
      numerus & \@3 singular \\
      kasus & \@4 genitiv
    \] \\
    valenz & \<\[ \asort{determinierer}
      kopf & \gruen{\@6} 
      \]\>
  \]
  \end{avm}%
}

\newcommand{\AvmBc}{%
  \begin{avm}
    \[ \asort{determinierer}
      graphen & \textit{des} \\
      kopf & \gruen{\@6} \[ genus & \ \@2 \\
        numerus & \@3 \\
        kasus & \ \@4
      \]\\
       valenz & \<\>
    \]
  \end{avm}
}

\newcommand{\AvmCc}{%
  \begin{avm}
    \[ \asort{nomen} 
      graphen & \textit{des Tischs} \\
      kopf & \gruen{\@5} 
  \]
  \end{avm}
}

\begin{frame}
  {Projizierte Kopfmerkmale}
  \onslide<+->
  \onslide<+->
  Durch Merkmalbündel | Optimale Struktur finden\slash \alert{Generalisierungen abbilden}\\
  \onslide<+->
  \Halbzeile
  \centering
  \begin{forest}
    [ \alt<1-3>{\scalebox{0.6}{\AvmCb}}{\scalebox{0.6}{\AvmCc}}
      [\alt<1-3>{\scalebox{0.6}{\AvmBb}}{\scalebox{0.6}{\AvmBc}}]
      [\alt<1-3>{\scalebox{0.6}{\AvmAb}}{\scalebox{0.6}{\AvmAc}}]
    ]
  \end{forest}
  \onslide<+->
\end{frame}

\newcommand{\AvmAd}{%
  \begin{avm}
    \[ \asort{\orongsch{wort}}
    graphen & \textit{Tischs} \\
    kopf & \gruen{\@5} \[ 
      \asort{\orongsch{nomen}}
      person & \@1 dritte \\
      genus & \@2 maskulin \\
      numerus & \@3 singular \\
      kasus & \@4 genitiv
    \] \\
    valenz & \<\[ \asort{\orongsch{wort}}
      kopf & \gruen{\@6} 
      \]\>
  \]
  \end{avm}%
}

\newcommand{\AvmBd}{%
  \begin{avm}
    \[ \asort{\orongsch{wort}}
      graphen & \textit{des} \\
      kopf & \gruen{\@6} \[
        \asort{\orongsch{determinierer}}
        genus & \ \@2 \\
        numerus & \@3 \\
        kasus & \ \@4
      \]\\
       valenz & \<\>
    \]
  \end{avm}
}

\newcommand{\AvmCd}{%
  \begin{avm}
    \[ \asort{\orongsch{phrase}} 
      graphen & \textit{des Tischs} \\
      kopf & \gruen{\@5} 
  \]
  \end{avm}
}


\begin{frame}
  {\textsc{head}-Typen}
  \onslide<+->
  \onslide<+->
  Wortartenspezifisch sind die \orongsch{\textsc{head}-Bündel}, nicht die \orongsch{Wörter\slash Phrasen}.\\
  \onslide<+->
  \Halbzeile
  \centering
  \begin{forest}
    [ \alt<1-3>{\scalebox{0.6}{\AvmCc}}{\scalebox{0.6}{\AvmCd}}
      [\alt<1-3>{\scalebox{0.6}{\AvmBc}}{\scalebox{0.6}{\AvmBd}}]
      [\alt<1-3>{\scalebox{0.6}{\AvmAc}}{\scalebox{0.6}{\AvmAd}}]
    ]
  \end{forest}
  \onslide<+->
\end{frame}

\section{Unifikation}

\begin{frame}
  {Zusammenlegen von Informationen}
  \onslide<+->
  \onslide<+->
  Beispiel | Lexikalische Spezifikation der Valenz einer Präposition\\
  \Viertelzeile
  \onslide<+->
  \scalebox{0.5}{\begin{avm}
    \[ \asort{wort} 
    graphen & \textit{wegen} \\
    kopf & \[ \asort{präposition} \] \\
    valenz & \< \gruen{\[ 
      kopf & \@1 \[ \asort{nomen}
          kasus & genitiv 
        \]
      \]} \>
  \]
  \end{avm}}\\
  \onslide<+->
  \Zeile
  Die NP kommt mit viel mehr Information daher.\\
  \onslide<+->
  \Viertelzeile
  \scalebox{0.5}{\begin{avm}
    \[ \asort{phrase} 
      graphen & \textit{des Tischs} \\
      kopf    & \gruen{\@1} \[ \asort{\gruen{nomen}}
                   person & dritte \\
                   genus & maskulin \\
                   numerus & singular \\
                   \gruen{kasus} & \gruen{genitiv}
                 \] \\
      valenz & \<\> \\
    \]
  \end{avm}}\\
  \onslide<+->
  \Zeile
  Die Informationen unter \mybox{1} sind aber kompatibel und \alert{unifizieren} daher.
\end{frame}

\begin{frame}
  {Unifikation}
  \onslide<+->
  \onslide<+->
  Bedingungen für Unifikation von zwei Merkmalstrukturen A und B\\
  \Zeile
  \begin{itemize}[<+->]
    \item A und B enthalten \alert{keine widersprüchlichen Informationen}.\\
      \Viertelzeile
      \onslide<+->
      \grau{%
        \scalebox{0.8}{\begin{avm}
          \[ kasus & nominativ \]
        \end{avm}} und %
        \scalebox{0.8}{\begin{avm}
          \[ kasus & akkusativ \]
        \end{avm}} unifizieren nicht.}
      \Halbzeile
    \item Aus nicht widersprüchlichen Informationen wird die \alert{Vereinigungsmenge} gebildet.
      \Halbzeile
    \item A kann mehr Informationen enthalten als B oder umgekehrt.\\
      \Viertelzeile
      \onslide<+->
      \grau{%
        \scalebox{0.8}{\begin{avm}
          \[ kasus & nominativ \\ person & dritte \]
        \end{avm}} und %
        \scalebox{0.8}{\begin{avm}
          \[ kasus & nominativ \]
        \end{avm}} unifizieren zu %
        \scalebox{0.8}{\begin{avm}
          \[ kasus & nominativ \\ person & dritte \]
        \end{avm}}}
      \Halbzeile
    \item A und B können beide mehr Informationen enthalten als die jeweils andere.\\
      \Viertelzeile
      \onslide<+->
      \grau{%
        \scalebox{0.8}{\begin{avm}
          \[ kasus & nominativ \\ person & dritte \]
        \end{avm}} und %
        \scalebox{0.8}{\begin{avm}
          \[ kasus & nominativ \\ numerus & singular \]
        \end{avm}} unifizieren zu %
        \scalebox{0.8}{\begin{avm}
          \[ kasus & nominativ \\ person & dritte \\ numerus & singular \]
        \end{avm}}}
  \end{itemize}
\end{frame}

\section{Nächste Woche}

\begin{frame}
  {Vorbereitung}
  \onslide<+->
  \onslide<+->
  \centering 
  \large
  Nächste Woche geht es um Valenz und Valenzabbindung.\\
  \onslide<+->
  \Zeile
  \rot{Sie sollten dringend vorher aus dem HPSG-Buch\\
  Abschnitt 3.1 und Kapitel 4 lesen!}\\
    \onslide<+->
    \Viertelzeile
  Das sind gerade mal 15 Seiten.
\end{frame}

  \let\subsection\section\let\section\woopsi

  \section{Wo sind wir?}

  \begin{frame}
    {Formale Syntax | Plan}
    \begin{enumerate}
      \item Phrasenstruktur und Phrasenstrukturgrammatiken 
      \item Merkmalstrukturen und Merkmalbeschreibungen
    \end{enumerate}
    \Halbzeile
    \centering 
    \url{https://hpsg.hu-berlin.de/~stefan/Pub/hpsg-lehrbuch.html}
  \end{frame}
\fi


\makeatletter
\setcounter{lastpagemainpart}{\the\c@framenumber}
\makeatother

\appendix

\begin{frame}[allowframebreaks]
  {Literatur}
  \renewcommand*{\bibfont}{\footnotesize}
  \setbeamertemplate{bibliography item}{}
  \printbibliography
\end{frame}

\begin{frame}
  {Autor}
  \begin{block}{Kontakt}
    Prof.\ Dr.\ Roland Schäfer\\
    Institut für Germanistische Sprachwissenschaft\\
    Friedrich-Schiller-Universität Jena\\
    Fürstengraben 30\\
    07743 Jena\\[\baselineskip]
    \url{https://rolandschaefer.net}\\
    \texttt{roland.schaefer@uni-jena.de}
  \end{block}
\end{frame}

\begin{frame}
  {Lizenz}
  \begin{block}{Creative Commons BY-SA-3.0-DE}
    Dieses Werk ist unter einer Creative Commons Lizenz vom Typ \textit{Namensnennung - Weitergabe unter gleichen Bedingungen 3.0 Deutschland} zugänglich.
    Um eine Kopie dieser Lizenz einzusehen, konsultieren Sie \url{http://creativecommons.org/licenses/by-sa/3.0/de/} oder wenden Sie sich brieflich an Creative Commons, Postfach 1866, Mountain View, California, 94042, USA.
  \end{block}
\end{frame}

\mode<beamer>{\setcounter{framenumber}{\thelastpagemainpart}}

\end{document}
